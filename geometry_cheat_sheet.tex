\documentclass{article}
\usepackage{amsmath,amssymb}
\usepackage{graphicx}
\usepackage{geometry}
\usepackage{enumitem}
\usepackage{microtype}
\usepackage{titlesec}
\usepackage[normalem]{ulem}
\usepackage{tikz}
\usetikzlibrary{angles,quotes,arrows.meta,calc,intersections,positioning,decorations.markings}
\geometry{a4paper, margin=0.8in}

% Colors and typography theme
\usepackage{xcolor}
\definecolor{primary}{HTML}{0B5394}    % deep blue
\definecolor{secondary}{HTML}{38761D}  % deep green
\definecolor{accent}{HTML}{990000}     % dark red
\definecolor{darktext}{HTML}{222222}
\definecolor{highlight}{HTML}{FFF2CC}  % light yellow
\color{darktext}

% Section heading styles
\titleformat*{\section}{\Large\bfseries\color{primary}}
\titleformat*{\subsection}{\large\bfseries\color{secondary}}
\titleformat*{\subsubsection}{\normalsize\bfseries\color{accent}}

% Boxed content
\newcommand{\keybox}[1]{\fcolorbox{primary!60!black}{primary!5}{\parbox{\linewidth-4pt}{\strut#1\strut}}}
\newcommand{\theorbox}[1]{\fcolorbox{secondary!60!black}{secondary!5}{\parbox{\linewidth-4pt}{\strut#1\strut}}}
\newcommand{\formulabox}[1]{\fcolorbox{accent!60!black}{accent!5}{\parbox{\linewidth-4pt}{\strut#1\strut}}}

% Spacing
\setlength{\parskip}{6pt}
\setlength{\parindent}{0pt}
\linespread{1.05}

\begin{document}

\begin{center}
\textbf{\Huge GEOMETRY CHEAT SHEET} \\
\textbf{\large Triangle Similarity \& Related Theorems} \\
\textit{Quick Reference Guide for Exercises 15A, 15B, 15C}
\end{center}

\section*{🔑 SIMILARITY CRITERIA}

\subsection*{AAA (Angle-Angle-Angle)}
\keybox{
\textbf{Theorem:} If three angles of one triangle are equal to three angles of another triangle, the triangles are similar.

\textbf{Conditions:} $\angle A = \angle X$, $\angle B = \angle Y$, $\angle C = \angle Z$

\textbf{Result:} $\triangle ABC \sim \triangle XYZ$

\textbf{Consequence:} All corresponding sides are proportional.
}

\subsection*{SAS (Side-Angle-Side)}
\keybox{
\textbf{Theorem:} If two sides of one triangle are proportional to two sides of another triangle and the included angles are equal, the triangles are similar.

\textbf{Conditions:} $\dfrac{AB}{XY} = \dfrac{AC}{XZ}$ and $\angle A = \angle X$

\textbf{Result:} $\triangle ABC \sim \triangle XYZ$

\textbf{Proportionality:} $\dfrac{AB}{XY} = \dfrac{BC}{YZ} = \dfrac{CA}{ZX}$
}

\subsection*{SSS (Side-Side-Side)}
\keybox{
\textbf{Theorem:} If three sides of one triangle are proportional to three sides of another triangle, the triangles are similar.

\textbf{Conditions:} $\dfrac{AB}{XY} = \dfrac{BC}{YZ} = \dfrac{CA}{ZX}$

\textbf{Result:} $\triangle ABC \sim \triangle XYZ$

\textbf{Consequence:} All corresponding angles are equal.
}

\section*{📐 KEY THEOREMS}

\subsection*{Angle Bisector Theorem}
\theorbox{
\textbf{Theorem:} The angle bisector divides the opposite side in the ratio of the other two sides.

\textbf{Statement:} In $\triangle ABC$, if $AD$ bisects $\angle BAC$, then: \\
$\dfrac{AB}{AC} = \dfrac{BD}{DC}$ or equivalently $\dfrac{AB}{BD} = \dfrac{AC}{DC}$

\textbf{Extended:} For angle bisector from vertex $B$: $\dfrac{AC}{BC} = \dfrac{AB}{BD} = \dfrac{AP}{PD}$
}

\subsection*{Midpoint Theorem}
\theorbox{
\textbf{Theorem:} The line segment joining the midpoints of two sides of a triangle is parallel to the third side and half its length.

\textbf{Statement:} If $M$ is midpoint of $AB$ and $N$ is midpoint of $AC$, then $MN \parallel BC$ and $MN = \frac{1}{2}BC$
}

\subsection*{Geometric Mean Theorems}
\theorbox{
\textbf{Right Triangle Altitude Theorem:} In right $\triangle ABC$ with $\angle C = 90^\circ$ and $CD \perp AB$, then:
\begin{enumerate}
\item $CD^2 = AD \cdot DB$
\item $AC^2 = AD \cdot AB$
\item $BC^2 = DB \cdot AB$
\end{enumerate}
}

\subsection*{Trapezium Properties}
\theorbox{
\textbf{Diagonal Proportionality:} In trapezium $ABCD$ with $AB \parallel CD$, diagonals $AC$ and $BD$ intersect at $O$, then:
\begin{enumerate}
\item $\triangle AOB \sim \triangle COD$
\item $\dfrac{AO}{CO} = \dfrac{BO}{DO}$ or $AO \cdot OD = BO \cdot OC$
\end{enumerate}
}

\section*{🧮 ESSENTIAL FORMULAS}

\subsection*{Basic Proportionality}
\formulabox{
\textbf{Similar Triangles Ratio:} If $\triangle ABC \sim \triangle XYZ$, then:
\[\dfrac{AB}{XY} = \dfrac{BC}{YZ} = \dfrac{CA}{ZX} = k\]

\textbf{Area Ratio:} $\dfrac{\text{Area of }\triangle ABC}{\text{Area of }\triangle XYZ} = k^2$

\textbf{Perimeter Ratio:} $\dfrac{\text{Perimeter of }\triangle ABC}{\text{Perimeter of }\triangle XYZ} = k$
}

\subsection*{Median Properties}
\formulabox{
\textbf{Median Length:} In $\triangle ABC$, median $AM$ to side $BC$:
\[\dfrac{AM}{BM} = \dfrac{AM}{MC} = \dfrac{2}{3}\]

\textbf{Centroid Division:} Medians intersect at centroid $G$ where:
\[AG : GD = 2 : 1\]
}

\subsection*{Angle Relationships}
\formulabox{
\textbf{Isosceles Triangle:} If $AB = AC$, then $\angle B = \angle C$

\textbf{Exterior Angle:} $\angle ACD = \angle ABC + \angle BAC$

\textbf{Parallel Lines:} If $AB \parallel CD$, then:
\begin{itemize}
\item Alternate interior angles equal
\item Corresponding angles equal
\item Consecutive interior angles supplementary
\end{itemize}
}

\section*{🛠 PROBLEM-SOLVING STRATEGIES}

\subsection*{Step-by-Step Approach}
\begin{enumerate}
\item \textbf{Read carefully:} Identify given information and what needs to be proved
\item \textbf{Draw diagram:} Sketch the figure with all given information
\item \textbf{Identify similarity:} Look for parallel lines, equal angles, proportional sides
\item \textbf{Apply theorems:} Use appropriate similarity criteria (AAA, SAS, SSS)
\item \textbf{Set up proportions:} Write ratios of corresponding sides
\item \textbf{Solve:} Cross-multiply and solve for unknowns
\item \textbf{Verify:} Check if all conditions are satisfied
\end{enumerate}

\subsection*{Common Problem Types}

\subsubsection*{Parallel Lines Problems}
\keybox{
\textbf{Strategy:} When parallel lines cut transversals, corresponding angles are equal.

\textbf{Example:} If $DE \parallel BC$ and $AD, AE$ are transversals, then:
\[\triangle ADE \sim \triangle ABC \implies \dfrac{AD}{AB} = \dfrac{AE}{AC} = \dfrac{DE}{BC}\]
}

\subsubsection*{Angle Bisector Problems}
\keybox{
\textbf{Strategy:} Use Angle Bisector Theorem directly.

\textbf{Example:} If $BD$ bisects $\angle ABC$, then $\dfrac{AB}{AC} = \dfrac{BD}{DC}$
}

\subsubsection*{Median Problems}
\keybox{
\textbf{Strategy:} Midpoints create parallel segments.

\textbf{Example:} If $D, E$ are midpoints of $AB, AC$, then $DE \parallel BC$ and $DE = \frac{1}{2}BC$
}

\subsubsection*{Right Triangle Problems}
\keybox{
\textbf{Strategy:} Use Pythagorean theorem and geometric mean.

\textbf{Example:} In right $\triangle ABC$ ($\angle C = 90^\circ$), $CD \perp AB$:
\[CD^2 = AD \cdot DB \implies CD = \sqrt{AD \cdot DB}\]
}

\section*{📊 QUICK REFERENCE CHART}

\begin{tabular}{|l|l|l|}
\hline
\textbf{Situation} & \textbf{Theorem/Criterion} & \textbf{Result} \\
\hline
Three angles equal & AAA & Similar triangles \\
\hline
Two sides proportional + included angle equal & SAS & Similar triangles \\
\hline
Three sides proportional & SSS & Similar triangles \\
\hline
Parallel lines & Corresponding angles & Similar triangles \\
\hline
Angle bisector & Angle Bisector Theorem & Side division ratio \\
\hline
Midpoints connected & Midpoint Theorem & Parallel segments \\
\hline
Right triangle + altitude & Geometric Mean & $CD^2 = AD \cdot DB$ \\
\hline
Trapezium diagonals & Diagonal proportionality & $AO \cdot OD = BO \cdot OC$ \\
\hline
\end{tabular}

\section*{⚡ QUICK CHECKLIST}

\subsection*{For Similarity Problems}
\begin{itemize}
\item[$\square$] Are there parallel lines? (→ AAA)
\item[$\square$] Are there equal angles with proportional sides? (→ SAS)
\item[$\square$] Are all sides proportional? (→ SSS)
\item[$\square$] Are there transversals to parallels? (→ Proportional segments)
\end{itemize}

\subsection*{Common Mistakes to Avoid}
\begin{itemize}
\item Don't assume triangles are congruent when they're only similar
\item Check that corresponding vertices match when writing similarity
\item Remember that similarity preserves angles but not sizes
\item Area ratios are squares of side ratios
\item Perimeter ratios equal side ratios
\end{itemize}

\section*{🎯 PRACTICE TIPS}

\begin{enumerate}
\item \textbf{Label diagrams clearly} with all given information
\item \textbf{Mark equal angles} with the same symbols (e.g., $\angle$ marks)
\item \textbf{Use color coding} for corresponding parts in similar triangles
\item \textbf{Write proportions} systematically for all corresponding sides
\item \textbf{Cross-multiply carefully} and check your algebra
\item \textbf{Verify answers} by substituting back into original equations
\end{enumerate}

\section*{📝 SYMBOLS \& NOTATION}

\begin{tabular}{|l|l|}
\hline
\textbf{Symbol} & \textbf{Meaning} \\
\hline
$\sim$ & Similar to \\
\hline
$\parallel$ & Parallel to \\
\hline
$\perp$ & Perpendicular to \\
\hline
$\angle$ & Angle \\
\hline
$\triangle$ & Triangle \\
\hline
$\dfrac{a}{b}$ & Ratio of a to b \\
\hline
$\therefore$ & Therefore \\
\hline
$\because$ & Because \\
\hline
\end{tabular}

\begin{center}
\textbf{Remember: Practice makes perfect! Keep solving problems and reviewing these concepts regularly.}
\end{center}

\end{document}