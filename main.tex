\documentclass[12pt]{article}

\usepackage{amsmath,amssymb}
\usepackage[dvipsnames,svgnames]{xcolor}
\usepackage{tikz}
\usepackage{tikz-cd}
\usepackage{graphicx}
\usepackage{mwe}
\usepackage{hyperref}
\usepackage{cleveref}
% ---------- Fonts and colors ----------
\usepackage{mathpazo} % Palatino text & math
\usepackage[scaled=0.85]{inconsolata} % Monospace font
\usepackage{sectsty}
\sectionfont{\sffamily\bfseries\color{MidnightBlue}}
\subsectionfont{\sffamily\bfseries\color{RoyalBlue}}
\usepackage[most]{tcolorbox}
% ---------- End fonts and colors ----------

% ---------- Theorem environments ----------
\usepackage{amsthm}

\newtheorem{theorem}{Theorem}[section]
\newtheorem{lemma}[theorem]{Lemma}
\newtheorem{proposition}[theorem]{Proposition}
\newtheorem{corollary}[theorem]{Corollary}

\theoremstyle{definition}
\newtheorem{definition}[theorem]{Definition}
\newtheorem{example}[theorem]{Example}

\theoremstyle{remark}
\newtheorem{remark}[theorem]{Remark}
% ---------- End theorem environments ----------

% ---------- Colored boxes for theorem environments ----------
\tcolorboxenvironment{theorem}{
  colback=blue!5!white,
  colframe=blue!75!black,
  fonttitle=\bfseries
}
\tcolorboxenvironment{lemma}{
  colback=green!5!white,
  colframe=green!60!black,
  fonttitle=\bfseries
}
\tcolorboxenvironment{proposition}{
  enhanced,
  colback=Magenta!5!white,
  colframe=Magenta!80!black,
  colbacktitle=Magenta!80!black,
  coltitle=white,
  fonttitle=\bfseries\sffamily,
  fontupper=\sffamily
}
\tcolorboxenvironment{corollary}{
  colback=orange!10!white,
  colframe=orange!70!black,
  fonttitle=\bfseries
}
\tcolorboxenvironment{definition}{
  colback=yellow!10!white,
  colframe=yellow!60!black,
  fonttitle=\bfseries
}
\tcolorboxenvironment{example}{
  colback=Teal!5!white,
  colframe=Teal!60!black,
  fonttitle=\bfseries
}
\tcolorboxenvironment{remark}{
  enhanced,
  colback=yellow!5!white,
  colframe=OrangeRed!80!black,
  colbacktitle=OrangeRed!80!black,
  coltitle=white,
  fonttitle=\bfseries\itshape,
  fontupper=\itshape
}
% ---------- End colored boxes ----------

% ---------- Shortcuts ----------
\newcommand{\N}{\mathbb{N}}
\newcommand{\Z}{\mathbb{Z}}
\newcommand{\Q}{\mathbb{Q}}
\newcommand{\R}{\mathbb{R}}
\newcommand{\C}{\mathbb{C}}
\DeclareMathOperator{\Gal}{Gal}
\DeclareMathOperator{\Aut}{Aut}
\DeclareMathOperator{\Tr}{Tr}
\DeclareMathOperator{\Norm}{N}

\title{Galois Theory --- Graduate Notes}
\author{Anonymous}
\date{\today}

% ---------- Bibliography ----------
\usepackage[backend=biber,style=alphabetic,maxbibnames=6]{biblatex}
\addbibresource{refs.bib}

\begin{document}
\maketitle
\tableofcontents

\section{Fields and polynomials}\label{sec:fields-polynomials}

\subsection{Fields, characteristic, prime fields}
\begin{definition}
A \emph{field} is a commutative ring $F\neq0$ in which every nonzero element has a multiplicative inverse. In particular $1\neq0$ and there are no zero divisors.
\end{definition}

\begin{definition}[Characteristic]
The \emph{characteristic} of a field $F$, denoted $\operatorname{char}F$, is the smallest integer $n\ge1$ with $n\cdot1_F=0$ if such $n$ exists; otherwise $\operatorname{char}F=0$.
\end{definition}

\begin{proposition}[Characteristic is $0$ or prime]\label{prop:char}
If $\operatorname{char}F=n>0$, then $n$ is a prime number. Consequently the characteristic of a field is either $0$ or a prime $p$.
\end{proposition}
\begin{proof}
Suppose $n=ab$ with $1<a,b<n$. Then
\[
0=n\cdot1_F=(a\cdot1_F)(b\cdot1_F),
\]
which expresses $0$ as a product of two nonzero elements. This contradicts the absence of zero divisors, so no such factorization exists and $n$ must be prime.
\end{proof}

\begin{proposition}[Prime field]\label{prop:prime-field}
Every field $F$ contains a smallest subfield, called the \emph{prime field}. It is isomorphic to $\Q$ when $\operatorname{char}F=0$ and to $\mathbb{F}_p$ when $\operatorname{char}F=p$.
\end{proposition}
\begin{proof}
Consider the homomorphism $\phi: \Z\to F$ given by $1\mapsto1_F$. Its image consists of the elements $n\cdot1_F$ and forms a subring containing $1_F$. Any subfield must contain this image, so it is the smallest subfield. If $\operatorname{char}F=0$, the map $\phi$ is injective and the image is a copy of $\Z$; adjoining inverses yields a field isomorphic to $\Q$. If $\operatorname{char}F=p$, then $\ker\phi=(p)$ and the image is $\Z/(p)\cong\mathbb{F}_p$.
\end{proof}

\begin{remark}
We may therefore regard any field as an extension of its prime field, which is either the rational numbers or a finite field of prime order.
\end{remark}

References: \cite[\S13]{DF}, \cite[Ch.~I]{Artin}.

\subsection{Polynomials over a field}
\begin{definition}
Let $F$ be a field. The polynomial ring $F[x]$ consists of finite sums $\sum_{i=0}^n a_i x^i$ with coefficients $a_i\in F$. The \emph{degree} $\deg f$ is the largest $i$ with $a_i\ne0$, and the coefficient $a_{\deg f}$ is called the \emph{leading coefficient}.
\end{definition}

\begin{proposition}[Division algorithm]\label{prop:division}
For polynomials $f,g\in F[x]$ with $g\ne0$ there exist unique $q,r\in F[x]$ such that
\[
f=qg+r, \qquad r=0\text{ or }\deg r<\deg g.
\]
\end{proposition}
\begin{proof}
We argue by induction on $\deg f$. If $\deg f<\deg g$ take $q=0$ and $r=f$. Otherwise write $f=a x^m+\cdots$ and $g=b x^n+\cdots$ with $m\ge n$. Set
\[
f_1=f-\frac{a}{b}x^{m-n}g,
\]
which has degree $<m$. By the induction hypothesis $f_1=q_1g+r$ with $r=0$ or $\deg r<\deg g$. Then $f=(q_1+\frac{a}{b}x^{m-n})g+r$, yielding the desired decomposition. For uniqueness, suppose $f=qg+r=q' g+r'$ with degrees of $r$ and $r'$ less than $\deg g$. Subtracting gives $(q-q')g=r'-r$. If $q\ne q'$, the left side has degree at least $\deg g$, contradicting the degree bound on the right. Thus $q=q'$ and $r=r'$.
\end{proof}

\begin{corollary}[Euclidean property]
The degree function makes $F[x]$ a Euclidean domain. Therefore $F[x]$ is a principal ideal domain and a unique factorization domain.
\end{corollary}

\begin{corollary}[B\'ezout identity]\label{cor:bezout}
For any $f,g\in F[x]$ there exist polynomials $a,b\in F[x]$ such that $af+bg=\gcd(f,g)$. In particular, $af+bg=1$ if and only if $f$ and $g$ are coprime.
\end{corollary}
\begin{proof}
Apply the extended Euclidean algorithm supplied by the division algorithm. The final nonzero remainder expresses the greatest common divisor as an $F[x]$-linear combination of $f$ and $g$.
\end{proof}

\begin{example}
In $\Q[x]$ the gcd of $x^3-x$ and $x^2-1$ is $x^2-1$. Indeed,
\[
x^3-x=(x)(x^2-1)\quad\text{and}\quad x^2-1=(1)(x^2-1)+0,
\]
so $af+bg=x^2-1$ with $a=0$ and $b=1$.
\end{example}

\subsection{Irreducibility criteria}
\begin{proposition}[Degree $2$ or $3$]\label{prop:deg23}
Let $f\in F[x]$ have degree $2$ or $3$. Then $f$ is reducible over $F$ if and only if it has a root in $F$.
\end{proposition}
\begin{proof}
If $f$ has a root $\alpha\in F$, then $f=(x-\alpha)g$ with $\deg g=1$ or $2$, so $f$ is reducible. Conversely, if $f=gh$ is a nontrivial factorization, then $\deg g,\deg h\ge1$ and $\deg g+\deg h=\deg f\le3$, so one factor has degree $1$ and provides a root in $F$.
\end{proof}

\begin{theorem}[Eisenstein]\label{thm:eisenstein}
Let $R$ be a UFD and $p\in R$ a prime element. Suppose $f(x)=a_nx^n+\cdots+a_0\in R[x]$ satisfies
\begin{enumerate}
 \item $p\nmid a_n$,
 \item $p\mid a_i$ for all $i<n$, and
 \item $p^2\nmid a_0$.
\end{enumerate}
Then $f$ is irreducible in $R[x]$ and therefore in the fraction field $\operatorname{Frac}(R)[x]$.
\end{theorem}
\begin{proof}
Assume $f=gh$ with nonconstant $g,h\in R[x]$. Write $g=b_mx^m+\cdots$ and $h=c_lx^l+\cdots$ with $m,l\ge1$. The constant term satisfies $a_0=b_0c_0$. Since $p^2\nmid a_0$, exactly one of $b_0$ and $c_0$ is divisible by $p$; say $p\mid b_0$ but $p\nmid c_0$. Reducing the factorization modulo $p$ yields
\[
\bar f=\bar g\,\bar h\in(R/(p))[x].
\]
However $\bar f=\bar a_n x^n$ is a monomial because $p\mid a_i$ for $i<n$. Thus $\bar g$ must be a unit in $(R/(p))[x]$, implying $p\nmid b_0$, a contradiction. No such factorization exists, so $f$ is irreducible.
\end{proof}

\begin{example}
Consider $x^5-10x+5$. Substituting $x=y+1$ gives
\[
y^5+5y^4+50y^2+40y+5,
\]
where every coefficient except the leading one is divisible by $5$ and the constant term is not divisible by $25$. By \cref{thm:eisenstein} with $p=5$, the shifted polynomial is irreducible in $\Z[y]$, hence $x^5-10x+5$ is irreducible over $\Q$.
\end{example}

References: \cite[\S9--11]{DF}, \cite[Ch.~II]{Lang}.

\subsection{Evaluation, roots, and factorization}
\begin{definition}
Let $E/F$ be a field extension and $f\in F[x]$. For $\alpha\in E$ define the \emph{evaluation map} $\operatorname{ev}_\alpha:F[x]\to E$, $f\mapsto f(\alpha)$. The element $\alpha$ is called a \emph{root} of $f$ if $f(\alpha)=0$.
\end{definition}

\begin{remark}
The evaluation map is a ring homomorphism. Factorization of polynomials can change when moving from $F$ to a larger field because new roots may appear. Adjoining enough roots to split a polynomial completely leads to the notion of splitting fields discussed in \cref{sec:splitting-fields}.
\end{remark}

% === Examples for Section 1: Fields and polynomials (auto-generated) ===
\begin{example}\label{ex:sec1-1}
\textbf{Problem.} Is $\mathbb{Z}$ a field?

\textbf{Solution.} Elements such as $2$ lack multiplicative inverses in $\mathbb{Z}$, so it is not a field.
\end{example}

\begin{example}\label{ex:sec1-2}
\textbf{Problem.} What is the characteristic of $\mathbb{F}_7$?

\textbf{Solution.} Adding $1$ seven times gives $0$, so the characteristic is $7$.
\end{example}

\begin{example}\label{ex:sec1-3}
\textbf{Problem.} Evaluate $p(x)=x^2+3x+2$ at $x=1$ in $\mathbb{Q}$.

\textbf{Solution.} $1^2+3\cdot1+2=6$.
\end{example}

\begin{example}\label{ex:sec1-4}
\textbf{Problem.} Factor $x^2-9$ over $\mathbb{Q}$.

\textbf{Solution.} $x^2-9=(x-3)(x+3)$.
\end{example}

\begin{example}\label{ex:sec1-5}
\textbf{Problem.} Find $\gcd(x^3-x, x^2-1)$ in $\mathbb{Q}[x]$.

\textbf{Solution.} $x^3-x = x(x^2-1)$, so the gcd is $x^2-1$.
\end{example}

\begin{example}\label{ex:sec1-6}
\textbf{Problem.} Is $x^2+1$ reducible over $\mathbb{R}$?

\textbf{Solution.} It has no real roots, so it is irreducible over $\mathbb{R}$.
\end{example}

\begin{example}\label{ex:sec1-7}
\textbf{Problem.} Why is $\mathbb{Q}[x]$ a Euclidean domain?

\textbf{Solution.} The division algorithm uses degree as a Euclidean function.
\end{example}

\begin{example}\label{ex:sec1-8}
\textbf{Problem.} Divide $x^3+1$ by $x+1$ in $\mathbb{Q}[x]$.

\textbf{Solution.} $x^3+1=(x+1)(x^2-x+1)$, so quotient $x^2-x+1$ and remainder $0$.
\end{example}

\begin{example}\label{ex:sec1-9}
\textbf{Problem.} Find $[\mathbb{Q}(\sqrt{5}):\mathbb{Q}]$.

\textbf{Solution.} $\sqrt{5}$ satisfies $x^2-5$, so the degree is $2$.
\end{example}

\begin{example}\label{ex:sec1-10}
\textbf{Problem.} Find the minimal polynomial of $\sqrt{2}+\sqrt{3}$ over $\mathbb{Q}$.

\textbf{Solution.} Let $\alpha=\sqrt{2}+\sqrt{3}$. Then $(\alpha^2-5)^2=24$, giving $\alpha^4-10\alpha^2+1=0$.
\end{example}

\section{Field Extensions and Minimal Polynomials}
\label{sec:extensions-minimal}

We examine extensions and minimal polynomials.

% === Examples for Section 2: Field extensions and minimal polynomials (auto-generated) ===
\begin{example}\label{ex:sec2-1}
\textbf{Problem.} What is the minimal polynomial of $\sqrt{2}$ over $\mathbb{Q}$?

\textbf{Solution.} $\sqrt{2}$ satisfies $x^2-2=0$, which is irreducible over $\mathbb{Q}$, so it is minimal.
\end{example}

\begin{example}\label{ex:sec2-2}
\textbf{Problem.} Find $[\mathbb{Q}(\sqrt{3}):\mathbb{Q}]$.

\textbf{Solution.} $\sqrt{3}$ has minimal polynomial $x^2-3$, so the degree is $2$.
\end{example}

\begin{example}\label{ex:sec2-3}
\textbf{Problem.} Is $x^3-2$ irreducible over $\mathbb{Q}$?

\textbf{Solution.} By the rational root test it has no root in $\mathbb{Q}$, so it is irreducible.
\end{example}

\begin{example}\label{ex:sec2-4}
\textbf{Problem.} Find the minimal polynomial of $1+\sqrt{2}$ over $\mathbb{Q}$.

\textbf{Solution.} Let $\alpha=1+\sqrt{2}$. Then $(\alpha-1)^2=2$ gives $\alpha^2-2\alpha-1=0$.
\end{example}

\begin{example}\label{ex:sec2-5}
\textbf{Problem.} What is $[\mathbb{Q}(\sqrt{2},\sqrt{3}):\mathbb{Q}]$?

\textbf{Solution.} The extensions by $\sqrt{2}$ and $\sqrt{3}$ are both quadratic and independent, so the degree is $4$.
\end{example}

\begin{example}\label{ex:sec2-6}
\textbf{Problem.} Is $\mathbb{Q}(\sqrt{2})/\mathbb{Q}$ a simple extension?

\textbf{Solution.} Yes, it is generated by the single element $\sqrt{2}$.
\end{example}

\begin{example}\label{ex:sec2-7}
\textbf{Problem.} Find the minimal polynomial of $i$ over $\mathbb{Q}$.

\textbf{Solution.} $i$ satisfies $x^2+1=0$, which is irreducible over $\mathbb{Q}$.
\end{example}

\begin{example}\label{ex:sec2-8}
\textbf{Problem.} Show that $\mathbb{Q}(\sqrt{5})$ is isomorphic to $\mathbb{Q}[x]/(x^2-5)$.

\textbf{Solution.} The map sending $x$ to $\sqrt{5}$ is a surjective homomorphism with kernel $(x^2-5)$.
\end{example}

\begin{example}\label{ex:sec2-9}
\textbf{Problem.} Find the minimal polynomial of $\sqrt[3]{2}$ over $\mathbb{Q}$.

\textbf{Solution.} $\sqrt[3]{2}$ satisfies $x^3-2=0$, irreducible by Eisenstein at $p=2$.
\end{example}

\begin{example}\label{ex:sec2-10}
\textbf{Problem.} Compute $[\mathbb{Q}(\sqrt{2}+\sqrt{3}):\mathbb{Q}]$.

\textbf{Solution.} The element $\sqrt{2}+\sqrt{3}$ has minimal polynomial $x^4-10x^2+1$, so the degree is $4$.
\end{example}

\section{Splitting fields and algebraic closure}\label{sec:splitting-fields}

\subsection{Splitting fields}
\begin{definition}
A \emph{splitting field} of $f\in F[x]$ is a minimal extension $E/F$ over which $f$ factors as linear terms.
\end{definition}
\begin{theorem}[Existence and uniqueness]
Every $f\in F[x]$ has a splitting field $E/F$, unique up to $F$-isomorphism.
\end{theorem}
\begin{proof}[Idea]
Existence by adjoining roots one by one; uniqueness by extending embeddings and induction on $\deg f$.
\end{proof}
References: \cite[\S14]{DF}, \cite[Ch.~V]{Artin}.

\subsection{Algebraic closures}
\begin{definition}
An \emph{algebraic closure} $\overline{F}$ of $F$ is an algebraic extension that is algebraically closed.
\end{definition}
\begin{theorem}
Every $F$ has an algebraic closure, unique up to $F$-isomorphism.
\end{theorem}
\begin{remark}
Construction uses Zorn's lemma; see \cite[Ch.~VIII]{Lang}.
\end{remark}

\subsection{Separability preview}
\begin{proposition}
Let $E/F$ be finite and $\sigma:E\hookrightarrow \overline{F}$ an $F$-embedding. Then $\#\{\sigma\}\le [E\!:\!F]$, with equality iff $E/F$ is separable.
\end{proposition}

\subsection{Examples}
\begin{example}
The splitting field of $x^3-2$ over $\Q$ is $\Q(\root3\of2,\zeta_3)$ of degree $6$.
\end{example}
\begin{example}
The splitting field of $x^n-1$ over $\Q$ is $\Q(\zeta_n)$; see \cref{sec:cyclotomic-kummer}.
\end{example}

% === Examples for Section 3: Splitting fields and algebraic closure (auto-generated) ===
\begin{example}\label{ex:sec3-1}
\textbf{Problem.} What is the splitting field of $x^2-2$ over $\mathbb{Q}$?

\textbf{Solution.} The polynomial factors as $(x-\sqrt{2})(x+\sqrt{2})$, so the splitting field is $\mathbb{Q}(\sqrt{2})$.
\end{example}

\begin{example}\label{ex:sec3-2}
\textbf{Problem.} Find the splitting field of $x^2+1$ over $\mathbb{Q}$.

\textbf{Solution.} Its roots are $\pm i$, so the splitting field is $\mathbb{Q}(i)$.
\end{example}

\begin{example}\label{ex:sec3-3}
\textbf{Problem.} Compute the splitting field of $x^3-1$ over $\mathbb{Q}$.

\textbf{Solution.} The roots are $1$ and the primitive cube roots $\zeta_3,\zeta_3^2$, so the splitting field is $\mathbb{Q}(\zeta_3)$.
\end{example}

\begin{example}\label{ex:sec3-4}
\textbf{Problem.} Is $\mathbb{Q}$ algebraically closed?

\textbf{Solution.} No, $x^2+1$ has no root in $\mathbb{Q}$.
\end{example}

\begin{example}\label{ex:sec3-5}
\textbf{Problem.} Why is every polynomial over $\mathbb{C}$ split?

\textbf{Solution.} The fundamental theorem of algebra states that $\mathbb{C}$ is algebraically closed.
\end{example}

\begin{example}\label{ex:sec3-6}
\textbf{Problem.} What is the splitting field of $x^3-2$ over $\mathbb{Q}$?

\textbf{Solution.} It is $\mathbb{Q}(\sqrt[3]{2},\zeta_3)$, adjoining a cube root and a primitive cube root of unity.
\end{example}

\begin{example}\label{ex:sec3-7}
\textbf{Problem.} Determine $[\mathbb{Q}(\sqrt[4]{5}, i):\mathbb{Q}]$.

\textbf{Solution.} The extension by $\sqrt[4]{5}$ has degree $4$, adjoining $i$ doubles it to $8$.
\end{example}

\begin{example}\label{ex:sec3-8}
\textbf{Problem.} Are algebraic closures unique up to isomorphism?

\textbf{Solution.} Yes, any two algebraic closures of a field are isomorphic.
\end{example}

\begin{example}\label{ex:sec3-9}
\textbf{Problem.} Find the splitting field of $x^4-1$ over $\mathbb{Q}$.

\textbf{Solution.} The roots are $\pm1,\pm i$, so the splitting field is $\mathbb{Q}(i)$.
\end{example}

\begin{example}\label{ex:sec3-10}
\textbf{Problem.} What is the splitting field of $(x^2-2)(x^2-3)$ over $\mathbb{Q}$?

\textbf{Solution.} Adjoining $\sqrt{2}$ and $\sqrt{3}$ suffices, so the field is $\mathbb{Q}(\sqrt{2},\sqrt{3})$.
\end{example}

\section{Automorphisms and Fixed Fields}
\label{sec:automorphisms-fixedfields}

Galois theory studies automorphisms and their fixed fields.

% === Examples for Section 4: Automorphisms and fixed fields (auto-generated) ===
\begin{example}\label{ex:sec4-1}
\textbf{Problem.} Define an automorphism of $\mathbb{Q}(\sqrt{2})$ over $\mathbb{Q}$.

\textbf{Solution.} Sending $\sqrt{2}$ to $-\sqrt{2}$ and fixing $\mathbb{Q}$ defines an automorphism.
\end{example}

\begin{example}\label{ex:sec4-2}
\textbf{Problem.} What is the fixed field of the automorphism above?

\textbf{Solution.} Only rational numbers are fixed, so the fixed field is $\mathbb{Q}$.
\end{example}

\begin{example}\label{ex:sec4-3}
\textbf{Problem.} Is complex conjugation an automorphism of $\mathbb{C}$ over $\mathbb{R}$?

\textbf{Solution.} Yes, it preserves addition and multiplication and fixes $\mathbb{R}$.
\end{example}

\begin{example}\label{ex:sec4-4}
\textbf{Problem.} Determine $\Aut_{\mathbb{Q}}(\mathbb{Q}(i))$.

\textbf{Solution.} There are two automorphisms: identity and complex conjugation, so the group has order $2$.
\end{example}

\begin{example}\label{ex:sec4-5}
\textbf{Problem.} Find an automorphism of $\mathbb{Q}(\sqrt{3})$ over $\mathbb{Q}$.

\textbf{Solution.} Mapping $\sqrt{3}$ to $-\sqrt{3}$ and fixing $\mathbb{Q}$ is an automorphism.
\end{example}

\begin{example}\label{ex:sec4-6}
\textbf{Problem.} What is the fixed field of the Frobenius map $x\mapsto x^p$ on $\mathbb{F}_{p^2}$?

\textbf{Solution.} Elements with $x^p=x$ form $\mathbb{F}_p$.
\end{example}

\begin{example}\label{ex:sec4-7}
\textbf{Problem.} Compute $\Aut_{\mathbb{F}_2}(\mathbb{F}_4)$.

\textbf{Solution.} The Frobenius $x\mapsto x^2$ generates a cyclic group of order $2$.
\end{example}

\begin{example}\label{ex:sec4-8}
\textbf{Problem.} Can $\mathbb{Q}$ have a nontrivial automorphism fixing $\mathbb{Q}$?

\textbf{Solution.} No, every element is rational, so only the identity works.
\end{example}

\begin{example}\label{ex:sec4-9}
\textbf{Problem.} Describe an automorphism of $\mathbb{C}$ fixing $\mathbb{Q}(i)$.

\textbf{Solution.} Complex conjugation fixes $i$ up to sign, so only the identity fixes $\mathbb{Q}(i)$ pointwise.
\end{example}

\begin{example}\label{ex:sec4-10}
\textbf{Problem.} If an automorphism of a field extension fixes a basis, what is it?

\textbf{Solution.} It must be the identity, since elements are linear combinations of the basis.
\end{example}

\section{Separable and normal extensions}\label{sec:sep-normal}

\subsection{Formal derivative and separability}
\begin{definition}
For $f(x)=\sum a_i x^i\in F[x]$, the \emph{formal derivative} is $f'(x)=\sum i\,a_i x^{i-1}$.
\end{definition}
\begin{proposition}[Multiplicity]
In a splitting field, a root $\alpha$ has multiplicity $>1$ iff $f(\alpha)=f'(\alpha)=0$.
\end{proposition}
\begin{proof}
Write $f=(x-\alpha)^m g$ in the splitting field with $g(\alpha)\ne0$. Then
\(f' = m(x-\alpha)^{m-1} g + (x-\alpha)^m g'\). If $m>1$ both $f(\alpha)$ and
$f'(\alpha)$ vanish. Conversely, if $f(\alpha)=f'(\alpha)=0$ and $f=(x-\alpha)g$
with $g(\alpha)\ne0$, then $f'(\alpha)=g(\alpha)\ne0$, a contradiction; hence
$m>1$.
\end{proof}
\begin{theorem}[Separability criteria]\label{thm:sep-criteria}
For irreducible $f\in F[x]$ the following are equivalent:
\begin{enumerate}[label=(\alph*)]
\item $f$ has no multiple roots in a splitting field;
\item $\gcd(f,f')=1$ in $F[x]$;
\item the $F$-algebra $F[x]/(f)$ has exactly $\deg f$ embeddings into $\overline{F}$.
\end{enumerate}
A root of such $f$ is \emph{separable}. An algebraic extension is \emph{separable} if each of its elements is separable.
\end{theorem}
\begin{proof}
(a)$\Leftrightarrow$(b): a multiple root $\alpha$ of $f$ satisfies
$f(\alpha)=f'(\alpha)=0$, so $\gcd(f,f')$ has positive degree. Conversely, if
$\gcd(f,f')\ne1$, a common root gives a multiple root.

(a)$\Leftrightarrow$(c): In a splitting field the embeddings of $F[x]/(f)$ send
$x$ to the distinct roots of $f$. Thus the number of embeddings equals the number
of distinct roots, which is $\deg f$ precisely when $f$ has no multiple roots.
\end{proof}
\begin{corollary}[Perfect fields]
If $\operatorname{char}F=0$ or $F$ is finite then every algebraic extension of $F$ is separable.
\end{corollary}
\begin{proof}
Let $f\in F[x]$ be irreducible. If $\operatorname{char}F=0$ then $f'\ne0$.
When $F$ is finite of characteristic $p$, the Frobenius map $x\mapsto x^p$ is
bijective, so an irreducible $f$ cannot be of the form $g(x^p)$ and hence
$f'\ne0$. By \cref{thm:sep-criteria}, $f$ is separable, and thus every algebraic
extension of $F$ is separable.
\end{proof}
References: \cite[\S13--14]{DF}, \cite[Ch.~VIII]{Lang}.

\subsection{Normal extensions}
\begin{definition}
An algebraic extension $E/F$ is \emph{normal} if every irreducible $f\in F[x]$ having a root in $E$ splits completely over $E$.
\end{definition}
\begin{theorem}[Equivalent characterizations]\label{thm:normal}
For algebraic $E/F$ the following are equivalent:
\begin{enumerate}[label=(\alph*)]
\item $E/F$ is normal;
\item $E$ is a splitting field of a family of polynomials in $F[x]$;
\item every $F$-embedding $\sigma:E\hookrightarrow\overline{F}$ satisfies $\sigma(E)=E$.
\end{enumerate}
\end{theorem}
\begin{proof}
(a)$\Rightarrow$(b): If $E/F$ is normal, take the family of irreducible
polynomials in $F[x]$ having a root in $E$; $E$ is a splitting field for this
family.

(b)$\Rightarrow$(c): If $E$ is a splitting field of polynomials in $F[x]$, any
$F$-embedding $\sigma$ permutes their roots, so $\sigma(E)=E$.

(c)$\Rightarrow$(a): Let $f\in F[x]$ be irreducible with a root $\alpha$ in $E$.
For any other root $\beta$, there is an $F$-embedding sending $\alpha$ to
$\beta$, hence $\beta\in E$. Thus $f$ splits over $E$, so $E/F$ is normal.
\end{proof}

\subsection{Primitive element theorem}
\begin{theorem}[Primitive element]\label{thm:PET}
If $E/F$ is finite and separable then $E=F(\alpha)$ for some $\alpha\in E$.
\end{theorem}
\begin{proof}
Take $E=F(\alpha,\beta)$. For distinct embeddings $\sigma\ne\tau$ of $E$ fixing
$F$, the set of $c\in F$ with $\sigma(\alpha+c\beta)=\tau(\alpha+c\beta)$ is
finite. Choose $c$ avoiding all these finitely many values. Then the conjugates
of $\gamma=\alpha+c\beta$ are distinct, so the minimal polynomial of $\gamma$ has
degree $[E\!:\!F]$, giving $E=F(\gamma)$.
\end{proof}
\begin{example}
$\Q(\sqrt2,\sqrt3)=\Q(\sqrt2+\sqrt3)$ since the extension is finite and separable.
\end{example}
References: \cite[\S14]{DF}, \cite[Ch.~VI]{Artin}.

% === Examples for Section 5: Separable and normal extensions (auto-generated) ===
\begin{example}\label{ex:sec5-1}
\textbf{Problem.} Is $x^2-2$ separable over $\mathbb{Q}$?

\textbf{Solution.} Yes, it has distinct roots $\pm\sqrt{2}$.
\end{example}

\begin{example}\label{ex:sec5-2}
\textbf{Problem.} Is $x^p-a$ separable over $\mathbb{F}_p$ when $a$ is not a $p$th power?

\textbf{Solution.} No, its derivative is $0$, so it has a repeated root and is inseparable.
\end{example}

\begin{example}\label{ex:sec5-3}
\textbf{Problem.} Is $x^2-t$ separable over $\mathbb{F}_2(t)$?

\textbf{Solution.} The derivative is $2x=0$, yet $x^2-t$ has distinct roots, so it is inseparable.
\end{example}

\begin{example}\label{ex:sec5-4}
\textbf{Problem.} Is $\mathbb{Q}(\sqrt{2})/\mathbb{Q}$ normal?

\textbf{Solution.} Yes, it is the splitting field of $x^2-2$.
\end{example}

\begin{example}\label{ex:sec5-5}
\textbf{Problem.} Is $\mathbb{Q}(\sqrt[3]{2})/\mathbb{Q}$ normal?

\textbf{Solution.} No, $x^3-2$ has complex roots not in the field.
\end{example}

\begin{example}\label{ex:sec5-6}
\textbf{Problem.} Are finite extensions of characteristic $0$ separable?

\textbf{Solution.} Yes, characteristic $0$ implies all irreducible polynomials have distinct roots.
\end{example}

\begin{example}\label{ex:sec5-7}
\textbf{Problem.} Is $\mathbb{Q}(\sqrt{2},\sqrt[3]{2})/\mathbb{Q}$ separable?

\textbf{Solution.} Both subextensions are separable, so the compositum is separable.
\end{example}

\begin{example}\label{ex:sec5-8}
\textbf{Problem.} Is $x^p-x$ separable over $\mathbb{F}_p$?

\textbf{Solution.} Its derivative is $-1$, so all roots are simple; it is separable.
\end{example}

\begin{example}\label{ex:sec5-9}
\textbf{Problem.} What indicates that a polynomial is inseparable?

\textbf{Solution.} It has repeated roots in its splitting field.
\end{example}

\begin{example}\label{ex:sec5-10}
\textbf{Problem.} Describe $\mathbb{F}_p(t^{1/p})/\mathbb{F}_p(t)$.

\textbf{Solution.} It is purely inseparable of degree $p$, generated by a single $p$th root.
\end{example}

\section{Galois Extensions and the Fundamental Theorem}
\label{sec:galois-fundamental-theorem}

We state and prove the Fundamental Theorem of Galois theory.

% === Examples for Section 6: Galois extensions and the Fundamental Theorem (auto-generated) ===
\begin{example}\label{ex:sec6-1}
\textbf{Problem.} Is $\mathbb{Q}(\sqrt{2})/\mathbb{Q}$ a Galois extension?

\textbf{Solution.} Yes, it is normal and separable; its Galois group has two elements.
\end{example}

\begin{example}\label{ex:sec6-2}
\textbf{Problem.} Is $\mathbb{Q}(\sqrt[3]{2})/\mathbb{Q}$ Galois?

\textbf{Solution.} No, it is not normal since complex cube roots are missing.
\end{example}

\begin{example}\label{ex:sec6-3}
\textbf{Problem.} Find $\Gal(\mathbb{Q}(i)/\mathbb{Q})$.

\textbf{Solution.} It has two elements: identity and complex conjugation.
\end{example}

\begin{example}\label{ex:sec6-4}
\textbf{Problem.} What conditions make a finite extension Galois?

\textbf{Solution.} Being both normal and separable.
\end{example}

\begin{example}\label{ex:sec6-5}
\textbf{Problem.} What is the fixed field of $\{\text{id},\text{conj}\}\subset \Gal(\mathbb{Q}(i)/\mathbb{Q})$?

\textbf{Solution.} The fixed field is $\mathbb{Q}$.
\end{example}

\begin{example}\label{ex:sec6-6}
\textbf{Problem.} Describe the intermediate fields of $\mathbb{Q}(\sqrt{2},\sqrt{3})/\mathbb{Q}$.

\textbf{Solution.} They are $\mathbb{Q}$, $\mathbb{Q}(\sqrt{2})$, $\mathbb{Q}(\sqrt{3})$, and $\mathbb{Q}(\sqrt{2},\sqrt{3})$.
\end{example}

\begin{example}\label{ex:sec6-7}
\textbf{Problem.} What is the Galois group of $\mathbb{Q}(\sqrt{2},\sqrt{3})/\mathbb{Q}$?

\textbf{Solution.} It is the Klein four group, generated by changing signs of $\sqrt{2}$ and $\sqrt{3}$.
\end{example}

\begin{example}\label{ex:sec6-8}
\textbf{Problem.} How does the fundamental theorem relate subgroups and subfields?

\textbf{Solution.} Intermediate fields correspond to subgroups of the Galois group, reversed by inclusion.
\end{example}

\begin{example}\label{ex:sec6-9}
\textbf{Problem.} Verify the correspondence for $\mathbb{Q}(i)/\mathbb{Q}$.

\textbf{Solution.} Subgroup $\{\text{id},\text{conj}\}$ matches subfield $\mathbb{Q}$; the trivial subgroup matches $\mathbb{Q}(i)$.
\end{example}

\begin{example}\label{ex:sec6-10}
\textbf{Problem.} If $L/K$ is Galois with group $G$, what is $K$?

\textbf{Solution.} It is the fixed field $L^G$.
\end{example}

\section{Finite fields}\label{sec:finite-fields}

\subsection{Existence and uniqueness}
\begin{theorem}
For each prime power $q=p^n$ there exists a field $\mathbb{F}_q$ of order $q$, unique up to isomorphism.
\end{theorem}
\begin{proof}
Choose an irreducible polynomial $f\in\mathbb{F}_p[x]$ of degree $n$.
The quotient $\mathbb{F}_p[x]/(f)$ is a field with $p^n$ elements.
If $K$ is any field with $q$ elements, then every $x\in K$ satisfies
$x^q=x$, so $K$ is a splitting field of $x^q-x$ over $\mathbb{F}_p$.
Any two splitting fields of this separable polynomial are isomorphic,
so $\mathbb{F}_q$ is unique.
\end{proof}
\begin{theorem}
$\mathbb{F}_q^\times$ is cyclic of order $q-1$.
\end{theorem}
\begin{proof}
Let $m$ be the maximal order of an element in $\mathbb{F}_q^\times$ and
choose $g$ with order $m$.
For each divisor $d$ of $m$, the equation $x^d=1$ has at most $d$ roots, so
$|\langle g\rangle|=m$.
If $m<q-1$, then every element has order dividing $m$, and the total
number of such elements is at most $m$, contradicting
$|\mathbb{F}_q^\times|=q-1$.
Thus $m=q-1$ and $\mathbb{F}_q^\times$ is cyclic of order $q-1$.
\end{proof}

\subsection{Frobenius and subfields}
\begin{proposition}[Frobenius]
The map $\varphi:x\mapsto x^p$ is an automorphism of $\mathbb{F}_{p^n}$ with fixed field $\mathbb{F}_p$.
\end{proposition}
\begin{proof}
The map $\varphi$ is a field homomorphism because $\operatorname{char}\mathbb{F}_{p^n}=p$.
In a finite field injective maps are surjective, hence $\varphi$ is an
automorphism.
If $\varphi(\alpha)=\alpha$, then $\alpha^p=\alpha$, so $\alpha\in\mathbb{F}_p$.
Conversely every element of $\mathbb{F}_p$ is fixed, so the fixed field is
$\mathbb{F}_p$.
\end{proof}
\begin{theorem}
$\Gal(\mathbb{F}_{p^n}/\mathbb{F}_p)=\langle \varphi\rangle\cong C_n$. Subfields are exactly $\mathbb{F}_{p^d}$ with $d\mid n$.
\end{theorem}
\begin{proof}
Since $\varphi^n=\operatorname{id}$ and $\varphi^d\ne\operatorname{id}$ for $0<d<n$, the
automorphism $\varphi$ has order $n$.
The extension $\mathbb{F}_{p^n}/\mathbb{F}_p$ has degree $n$, so the
Galois group has at most $n$ elements.
Thus $\Gal(\mathbb{F}_{p^n}/\mathbb{F}_p)=\langle\varphi\rangle\cong C_n$.
The fixed field of $\varphi^d$ is $\mathbb{F}_{p^d}$, and every
intermediate field is fixed by some subgroup of $\langle\varphi\rangle$,
so the subfields are precisely $\mathbb{F}_{p^d}$ with $d\mid n$.
\end{proof}
References: \cite[\S13]{DF}, \cite[Ch.~VIII]{Lang}.

\subsection{Irreducibles, trace, and norm}
\begin{proposition}
For $\alpha\in\mathbb{F}_{p^n}$ with Frobenius orbit of size $r$,
\[
m_\alpha(x)=\prod_{i=0}^{r-1}\bigl(x-\alpha^{p^i}\bigr).
\]
\end{proposition}
\begin{proof}
The conjugates of $\alpha$ under $\Gal(\mathbb{F}_{p^n}/\mathbb{F}_p)$ are
$\alpha,\alpha^p,\ldots,\alpha^{p^{r-1}}$.
Their product is a monic polynomial in $\mathbb{F}_p[x]$ that has $\alpha$
as a root, so it is divisible by the minimal polynomial $m_\alpha(x)$.
Because distinct conjugates give distinct factors, the degree is $r$,
which equals $\deg m_\alpha$; hence the displayed polynomial is
$m_\alpha(x)$.
\end{proof}
\begin{definition}
For $E=\mathbb{F}_{p^n}$ over $F=\mathbb{F}_p$ define
\[
\Tr_{E/F}(\alpha)=\alpha+\alpha^p+\cdots+\alpha^{p^{n-1}},\qquad
\Norm_{E/F}(\alpha)=\alpha^{1+p+\cdots+p^{n-1}}.
\]
\end{definition}

\subsection{Examples}
\begin{example}
$\mathbb{F}_4=\mathbb{F}_2[\omega]/(\omega^2+\omega+1)$ with $\omega^3=1$, and $\mathbb{F}_4^\times=\langle\omega\rangle$ is cyclic of order $3$.
\end{example}

% === Examples for Section 7: Finite fields (auto-generated) ===
\begin{example}\label{ex:sec7-1}
\textbf{Problem.} How many elements does $\mathbb{F}_5$ have?

\textbf{Solution.} It has $5$ elements.
\end{example}

\begin{example}\label{ex:sec7-2}
\textbf{Problem.} What is the size of any finite field?

\textbf{Solution.} Every finite field has $p^n$ elements for a prime $p$ and integer $n$.
\end{example}

\begin{example}\label{ex:sec7-3}
\textbf{Problem.} Describe the multiplicative group of $\mathbb{F}_5$.

\textbf{Solution.} Nonzero elements form a cyclic group of order $4$.
\end{example}

\begin{example}\label{ex:sec7-4}
\textbf{Problem.} Give a construction of $\mathbb{F}_4$.

\textbf{Solution.} $\mathbb{F}_4 \cong \mathbb{F}_2[x]/(x^2+x+1)$.
\end{example}

\begin{example}\label{ex:sec7-5}
\textbf{Problem.} How many subfields of $\mathbb{F}_{p^n}$ have size $p^m$ dividing $n$?

\textbf{Solution.} Exactly one such subfield exists for each divisor $m$ of $n$.
\end{example}

\begin{example}\label{ex:sec7-6}
\textbf{Problem.} In $\mathbb{F}_4$, compute $(\alpha+1)^2$ for $\alpha^2+\alpha+1=0$.

\textbf{Solution.} $(\alpha+1)^2=\alpha^2+2\alpha+1=\alpha+1$ since $2=0$.
\end{example}

\begin{example}\label{ex:sec7-7}
\textbf{Problem.} What is the Frobenius map on $\mathbb{F}_{p^n}$?

\textbf{Solution.} It is $x\mapsto x^p$, an automorphism.
\end{example}

\begin{example}\label{ex:sec7-8}
\textbf{Problem.} Find the order of $\mathbb{F}_{7^2}^\times$.

\textbf{Solution.} It has $7^2-1=48$ elements.
\end{example}

\begin{example}\label{ex:sec7-9}
\textbf{Problem.} Does $x^{p^n}-x$ split over $\mathbb{F}_{p^n}$?

\textbf{Solution.} Yes, every element is a root, so it splits completely.
\end{example}

\begin{example}\label{ex:sec7-10}
\textbf{Problem.} What is the minimal polynomial of $\alpha$ satisfying $\alpha^2+\alpha+1=0$ over $\mathbb{F}_2$?

\textbf{Solution.} It is $x^2+x+1$, irreducible over $\mathbb{F}_2$.
\end{example}

\section{Cyclotomic fields and basic Kummer theory}\label{sec:cyclotomic-kummer}

\subsection{Cyclotomic polynomials}
\begin{definition}
For a primitive $n$th root of unity $\zeta_n$, the \emph{cyclotomic polynomial} is
\[
\Phi_n(x)=\prod_{\substack{1\le a\le n\\ \gcd(a,n)=1}}\bigl(x-\zeta_n^{\,a}\bigr)\in\Z[x].
\]
\end{definition}
\begin{theorem}
$x^n-1=\prod_{d\mid n}\Phi_d(x)$, $\deg\Phi_n=\varphi(n)$, and $\Phi_n$ is irreducible over $\Q$.
\end{theorem}
\begin{proof}
Every $n$th root of unity is $\zeta_n^a$ for some $a$.
Those with $\gcd(a,n)=1$ are the primitive roots, so grouping terms by
their orders gives $x^n-1=\prod_{d\mid n}\Phi_d(x)$.
The degree of $\Phi_n$ equals the number of $a$ with $1\le a\le n$ and
$\gcd(a,n)=1$, namely $\varphi(n)$.
Let $\zeta_n$ be primitive.
Its conjugates over $\Q$ are $\zeta_n^{\,a}$ with $\gcd(a,n)=1$, so the
minimal polynomial of $\zeta_n$ is $\Phi_n$; hence $\Phi_n$ is
irreducible over $\Q$.
\end{proof}
References: \cite[\S13]{DF}, \cite[Ch.~VIII]{Lang}.

\subsection{Cyclotomic fields}
\begin{definition}
The \emph{cyclotomic field} $\Q(\zeta_n)$ is the splitting field of $x^n-1$ over $\Q$.
\end{definition}
\begin{theorem}
Restriction induces
\[
\Gal\bigl(\Q(\zeta_n)/\Q\bigr)\ \cong\ (\Z/n\Z)^\times,\qquad \sigma_a(\zeta_n)=\zeta_n^{\,a}.
\]
\end{theorem}
\begin{proof}
Any automorphism of $\Q(\zeta_n)$ is determined by the image of
$\zeta_n$, which must be another primitive $n$th root, say
$\zeta_n^{\,a}$ with $\gcd(a,n)=1$.
The map $a\mapsto\sigma_a$ is a homomorphism from
$(\Z/n\Z)^\times$ to the Galois group.
It is injective because $\sigma_a=\operatorname{id}$ only when
$a\equiv1\pmod n$.
Since both groups have size $\varphi(n)$, it is an isomorphism.
\end{proof}

\subsection{Basic Kummer theory}
Assume $\mathrm{char}\,F\nmid n$ and $\mu_n\subset F$.
\begin{theorem}[Kummer]
Cyclic degree-$n$ extensions of $F$ correspond to cosets in $F^\times/F^{\times n}$ via $a\mapsto F(\sqrt[n]{a})$; the extension has degree $n$ iff $a\notin F^{\times n}$.
\end{theorem}
\begin{proof}
Let $L/F$ be a cyclic extension of degree $n$ and choose
$\sigma\in\Gal(L/F)$ generating the group.
Pick $\alpha\in L^\times$ with
$\sigma(\alpha)=\zeta_n\alpha$; then $\alpha^n\in F$ and
$L=F(\alpha)=F(\sqrt[n]{\alpha^n})$.
Different choices of $\alpha$ change $\alpha^n$ by an $n$th power of
$F^\times$, giving the correspondence.
Conversely, if $a\in F^\times$ and $L=F(\sqrt[n]{a})$, then
$[L:F]$ divides $n$, and the action of $\Gal(L/F)$ on $\sqrt[n]{a}$ shows
the extension is cyclic.
We have $[L:F]=n$ exactly when $a\notin F^{\times n}$.
\end{proof}
\begin{remark}[Kronecker–Weber]
Every finite abelian extension of $\Q$ is contained in some $\Q(\zeta_n)$ \cite{Neukirch}.
\end{remark}

% === Examples for Section 8: Cyclotomic fields and basic Kummer theory (auto-generated) ===
\begin{example}\label{ex:sec8-1}
\textbf{Problem.} What is the minimal polynomial of $\zeta_3$ over $\mathbb{Q}$?

\textbf{Solution.} It is $x^2+x+1$.
\end{example}

\begin{example}\label{ex:sec8-2}
\textbf{Problem.} Find $[\mathbb{Q}(\zeta_5):\mathbb{Q}]$.

\textbf{Solution.} The degree is $\varphi(5)=4$.
\end{example}

\begin{example}\label{ex:sec8-3}
\textbf{Problem.} Give a primitive $4$th root of unity.

\textbf{Solution.} $i$ is primitive of order $4$.
\end{example}

\begin{example}\label{ex:sec8-4}
\textbf{Problem.} Do roots of unity form a cyclic group?

\textbf{Solution.} Yes, under multiplication they form a cyclic group.
\end{example}

\begin{example}\label{ex:sec8-5}
\textbf{Problem.} What is $\Phi_3(x)$?

\textbf{Solution.} $\Phi_3(x)=x^2+x+1$.
\end{example}

\begin{example}\label{ex:sec8-6}
\textbf{Problem.} What is $[\mathbb{Q}(\zeta_n):\mathbb{Q}]$?

\textbf{Solution.} It equals Euler's totient $\varphi(n)$.
\end{example}

\begin{example}\label{ex:sec8-7}
\textbf{Problem.} What is the degree of $\zeta_p+\zeta_p^{-1}$ over $\mathbb{Q}$ for prime $p$?

\textbf{Solution.} It has degree $(p-1)/2$.
\end{example}

\begin{example}\label{ex:sec8-8}
\textbf{Problem.} Compute $\Phi_8(x)$.

\textbf{Solution.} $\Phi_8(x)=x^4+1$.
\end{example}

\begin{example}\label{ex:sec8-9}
\textbf{Problem.} Is $2$ a $5$th power in $\mathbb{Q}(\zeta_5)$?

\textbf{Solution.} No, $2$ is not a $5$th power in that field.
\end{example}

\begin{example}\label{ex:sec8-10}
\textbf{Problem.} When can $a$ be written as an $n$th power in a Kummer extension?

\textbf{Solution.} When $a$ is an $n$th power modulo units and $\zeta_n$ is present.
\end{example}

\section{Solvability by radicals and the quintic}\label{sec:radicals-quintic}

\subsection{Radical extensions and solvable groups}
\begin{definition}
An extension $E/F$ is \emph{radical} if there is a tower
\[
F=F_0\subset F_1\subset\cdots\subset F_m=E,\qquad
F_i=F_{i-1}\bigl(\sqrt[n_i]{a_i}\bigr)
\]
with $n_i\ge2$, $a_i\in F_{i-1}^\times$.
\end{definition}
\begin{definition}
A finite group $G$ is \emph{solvable} if $1=G_0\trianglelefteq\cdots\trianglelefteq G_r=G$ with abelian factors $G_i/G_{i-1}$.
\end{definition}

\subsection{Galois criterion}
\begin{theorem}\label{thm:radicals-criterion}
Assume $\operatorname{char}F=0$ and $f\in F[x]$ is separable. Then $f$ is solvable by radicals over $F$ iff $\Gal(E/F)$ is solvable, where $E$ is the splitting field of $f$.
\end{theorem}
\begin{proof}
Assume $f$ is solvable by radicals over $F$. Take a radical tower
\[
F=F_0\subset\cdots\subset F_m=E,\qquad F_i=F_{i-1}(\alpha_i),\ \alpha_i^{n_i}\in F_{i-1}
\]
after adjoining roots of unity. Each step is a Kummer extension with cyclic Galois group. Intersecting the normal closures of these extensions yields a normal series of $\Gal(E/F)$ with abelian quotients, so $\Gal(E/F)$ is solvable.

Conversely, let $G=\Gal(E/F)$ be solvable and choose
\[
1=G_0\trianglelefteq\cdots\trianglelefteq G_r=G
\]
with abelian factors. Put $E_i=E^{G_i}$. Then $F=E_0\subset\cdots\subset E_r=E$ and each $E_i/E_{i-1}$ is Galois with abelian Galois group $G_i/G_{i-1}$. Such extensions are generated by radicals, so concatenating these gives a radical tower for $E/F$.
\end{proof}
\begin{theorem}[Abel–Ruffini]
The general quintic over $\Q$ is not solvable by radicals.
\end{theorem}
\begin{proof}
The splitting field of the general quintic has Galois group $S_5$. Since $S_5$ is not solvable, the previous theorem implies the polynomial is not solvable by radicals.
\end{proof}

\subsection{Detecting $S_n$}
\begin{proposition}
If $f\in\Q[x]$ is irreducible of prime degree $p$ with exactly two nonreal roots, then $\Gal(f)\cong S_p$.
\end{proposition}
\begin{proof}
Let $E$ be the splitting field of $f$ and $G=\Gal(E/\Q)$. Irreducibility of $f$ and the primeness of $p$ make $G$ act transitively on the roots. Complex conjugation fixes the real roots and swaps the two nonreal roots, giving a transposition in $G$. Any transitive subgroup of $S_p$ containing a transposition is $S_p$, so $G\cong S_p$.
\end{proof}

% === Examples for Section 9: Solvability by radicals and the quintic (auto-generated) ===
\begin{example}\label{ex:sec9-1}
\textbf{Problem.} Solve $x^2-2=0$ by radicals.

\textbf{Solution.} $x=\pm\sqrt{2}$ solves the equation.
\end{example}

\begin{example}\label{ex:sec9-2}
\textbf{Problem.} Solve $x^3-2=0$ by radicals.

\textbf{Solution.} $x=\sqrt[3]{2}$ and its complex multiples are solutions.
\end{example}

\begin{example}\label{ex:sec9-3}
\textbf{Problem.} Are all quartic equations solvable by radicals?

\textbf{Solution.} Yes, there is a general quartic formula.
\end{example}

\begin{example}\label{ex:sec9-4}
\textbf{Problem.} Are all quintic equations solvable by radicals?

\textbf{Solution.} No, some have Galois group $S_5$ and are not solvable.
\end{example}

\begin{example}\label{ex:sec9-5}
\textbf{Problem.} Is $x^5-1=0$ solvable by radicals?

\textbf{Solution.} Yes, roots are $5$th roots of unity.
\end{example}

\begin{example}\label{ex:sec9-6}
\textbf{Problem.} What does solvable Galois group imply about the polynomial?

\textbf{Solution.} The polynomial is solvable by radicals.
\end{example}

\begin{example}\label{ex:sec9-7}
\textbf{Problem.} Give a polynomial with Galois group $S_3$.

\textbf{Solution.} $x^3-2$ has Galois group $S_3$.
\end{example}

\begin{example}\label{ex:sec9-8}
\textbf{Problem.} Provide a polynomial with Galois group $S_5$.

\textbf{Solution.} $x^5-6x+3$ has Galois group $S_5$ over $\mathbb{Q}$.
\end{example}

\begin{example}\label{ex:sec9-9}
\textbf{Problem.} Does solvability by radicals depend on characteristic $0$?

\textbf{Solution.} Yes, classical results assume characteristic $0$.
\end{example}

\begin{example}\label{ex:sec9-10}
\textbf{Problem.} Give a quintic that is solvable by radicals.

\textbf{Solution.} $x^5-1$ is solvable since its Galois group is cyclic.
\end{example}

\section{Worked examples and computations}\label{sec:worked-examples}

\subsection{Quadratic and biquadratic}
\begin{example}[Quadratic]
For $d$ squarefree, the splitting field of $x^2-d$ is $\Q(\sqrt d)$ with $\Gal\cong C_2$.
\end{example}
\begin{example}[Biquadratic]
$E=\Q(\sqrt2,\sqrt3)$ is Galois over $\Q$ with group $V_4$. Intermediate fields are $\Q(\sqrt2)$, $\Q(\sqrt3)$, and $\Q(\sqrt6)$.
\end{example}

\subsection{A simple cubic}
\begin{example}
For $f(x)=x^3-2$, the splitting field is $\Q(\root3\of2,\zeta_3)$ of degree $6$, with $\Gal\cong S_3$ by \cref{prop:permute-roots}.
\end{example}

\subsection{A separable quartic}
\begin{example}
Let $f(x)=x^4-2$. The splitting field is $\Q(\sqrt[4]{2},i)$. The Galois group is the dihedral group $D_4$ of order $8$; complex conjugation gives a reflection.
\end{example}

\subsection{Cyclotomic}
\begin{example}
$\Q(\zeta_7)/\Q$ is Galois of degree $\varphi(7)=6$ with cyclic group $C_6$ (\cref{sec:cyclotomic-kummer}).
\end{example}

\subsection{Finite fields}
\begin{example}
$\Gal(\mathbb{F}_{p^n}/\mathbb{F}_p)=\langle x\mapsto x^p\rangle\cong C_n$; the subfields are $\mathbb{F}_{p^d}$ for $d\mid n$ (\cref{sec:finite-fields}).
\end{example}

\subsection{Discriminant hints}
\begin{remark}
For irreducible $f\in\Q[x]$ of degree $n$, if the discriminant is a square and the mod-$p$ factorization shows a $2$-cycle, then $\Gal(f)\subset A_n$ is impossible; hence $\Gal(f)=S_n$.
\end{remark}


% === Examples for Section 10: Worked examples and computations (auto-generated) ===
\begin{example}\label{ex:sec10-1}
\textbf{Problem.} Compute $\gcd(x^2-1,x^2-3)$ in $\mathbb{Q}[x]$.

\textbf{Solution.} The gcd is $1$ since the polynomials have distinct roots.
\end{example}

\begin{example}\label{ex:sec10-2}
\textbf{Problem.} Factor $x^2+2x+1$.

\textbf{Solution.} $(x+1)^2$.
\end{example}

\begin{example}\label{ex:sec10-3}
\textbf{Problem.} Solve $x^2+1=0$ over $\mathbb{C}$.

\textbf{Solution.} $x=\pm i$.
\end{example}

\begin{example}\label{ex:sec10-4}
\textbf{Problem.} Compute $[\mathbb{Q}(\sqrt{2}):\mathbb{Q}]$.

\textbf{Solution.} The degree is $2$.
\end{example}

\begin{example}\label{ex:sec10-5}
\textbf{Problem.} Find the inverse of $2$ in $\mathbb{F}_5$.

\textbf{Solution.} $2\cdot3=6\equiv1$, so the inverse is $3$.
\end{example}

\begin{example}\label{ex:sec10-6}
\textbf{Problem.} Determine the minimal polynomial of $i$ over $\mathbb{Q}$.

\textbf{Solution.} It is $x^2+1$.
\end{example}

\begin{example}\label{ex:sec10-7}
\textbf{Problem.} Compute $(1+i)^2$.

\textbf{Solution.} $(1+i)^2=1+2i+i^2=2i$.
\end{example}

\begin{example}\label{ex:sec10-8}
\textbf{Problem.} Evaluate $\Norm_{\mathbb{Q}(\sqrt{2})/\mathbb{Q}}(1+\sqrt{2})$.

\textbf{Solution.} $(1+\sqrt{2})(1-\sqrt{2})=-1$.
\end{example}

\begin{example}\label{ex:sec10-9}
\textbf{Problem.} Find $\Tr_{\mathbb{Q}(\sqrt{2})/\mathbb{Q}}(1+\sqrt{2})$.

\textbf{Solution.} $(1+\sqrt{2})+(1-\sqrt{2})=2$.
\end{example}

\begin{example}\label{ex:sec10-10}
\textbf{Problem.} Compute $\Gal(\mathbb{Q}(i)/\mathbb{Q})$.

\textbf{Solution.} It has two elements: identity and complex conjugation.
\end{example}

\section{Infinite Galois Theory}
\label{sec:infinite-galois}

Infinite Galois groups carry the Krull topology.

% === Examples for Section 11: Infinite Galois theory and the Krull topology (auto-generated) ===
\begin{example}\label{ex:sec11-1}
\textbf{Problem.} Give an infinite Galois extension of $\mathbb{Q}$.

\textbf{Solution.} The field generated by all $p$th-power roots of unity over $\mathbb{Q}$ is infinite Galois.
\end{example}

\begin{example}\label{ex:sec11-2}
\textbf{Problem.} What topology does the Galois group of an infinite extension carry?

\textbf{Solution.} It carries the Krull topology, making it a profinite group.
\end{example}

\begin{example}\label{ex:sec11-3}
\textbf{Problem.} Is the Galois group of $\bigcup_{n} \mathbb{Q}(\zeta_{p^n})$ over $\mathbb{Q}$ profinite?

\textbf{Solution.} Yes, it is isomorphic to the inverse limit of $(\mathbb{Z}/p^n\mathbb{Z})^\times$.
\end{example}

\begin{example}\label{ex:sec11-4}
\textbf{Problem.} What is a basic open set in the Krull topology?

\textbf{Solution.} The set of automorphisms fixing a given finite extension.
\end{example}

\begin{example}\label{ex:sec11-5}
\textbf{Problem.} Are fixed fields of closed subgroups Galois extensions?

\textbf{Solution.} Yes, closed subgroups correspond to Galois subextensions.
\end{example}

\begin{example}\label{ex:sec11-6}
\textbf{Problem.} Give an infinite extension that is not Galois.

\textbf{Solution.} $\mathbb{Q}(\sqrt[3]{2},\sqrt[3]{4},\dots)/\mathbb{Q}$ is not normal.
\end{example}

\begin{example}\label{ex:sec11-7}
\textbf{Problem.} What is the closure of the trivial subgroup in a profinite group?

\textbf{Solution.} It is the trivial subgroup itself.
\end{example}

\begin{example}\label{ex:sec11-8}
\textbf{Problem.} Do inverse limits preserve surjectivity?

\textbf{Solution.} Yes, inverse limits of surjections are surjective in profinite groups.
\end{example}

\begin{example}\label{ex:sec11-9}
\textbf{Problem.} What extension corresponds to the whole Galois group?

\textbf{Solution.} The fixed field is the base field.
\end{example}

\begin{example}\label{ex:sec11-10}
\textbf{Problem.} How do finite Galois extensions sit inside an infinite one?

\textbf{Solution.} They form a directed system whose limit is the whole extension.
\end{example}

\section{Classical Geometric Constructibility}
\label{sec:constructible-geometry}

We link constructibility problems to field extensions.

% === Examples for Section 12: Classical geometric constructibility (auto-generated) ===
\begin{example}\label{ex:sec12-1}
\textbf{Problem.} Can a segment of length $\sqrt{2}$ be constructed?

\textbf{Solution.} Yes, it is the hypotenuse of a unit right triangle.
\end{example}

\begin{example}\label{ex:sec12-2}
\textbf{Problem.} Is $\sqrt[3]{2}$ constructible by ruler and compass?

\textbf{Solution.} No, it has degree $3$, not a power of $2$.
\end{example}

\begin{example}\label{ex:sec12-3}
\textbf{Problem.} Is a regular pentagon constructible?

\textbf{Solution.} Yes, since $5$ is a Fermat prime.
\end{example}

\begin{example}\label{ex:sec12-4}
\textbf{Problem.} Is a regular heptagon constructible?

\textbf{Solution.} No, $7$ is not a Fermat prime.
\end{example}

\begin{example}\label{ex:sec12-5}
\textbf{Problem.} Can a square be doubled exactly?

\textbf{Solution.} Yes, by constructing $\sqrt{2}$.
\end{example}

\begin{example}\label{ex:sec12-6}
\textbf{Problem.} Can a general angle be trisected with ruler and compass?

\textbf{Solution.} No, this is impossible in general.
\end{example}

\begin{example}\label{ex:sec12-7}
\textbf{Problem.} Can an angle be bisected?

\textbf{Solution.} Yes, classical constructions bisect any angle.
\end{example}

\begin{example}\label{ex:sec12-8}
\textbf{Problem.} Can a perpendicular be erected from a point on a line?

\textbf{Solution.} Yes, using circles of equal radius.
\end{example}

\begin{example}\label{ex:sec12-9}
\textbf{Problem.} Is a regular $17$-gon constructible?

\textbf{Solution.} Yes, $17$ is a Fermat prime.
\end{example}

\begin{example}\label{ex:sec12-10}
\textbf{Problem.} Can one construct the angle $\pi/3$?

\textbf{Solution.} Yes, equilateral triangle gives $60^{\circ}$.
\end{example}

% 100 worked examples autogenerated by script
\section{100 Worked Examples}\label{sec:100-worked-examples}

\subsection{Very Easy}
\begin{example}[Easy 1]
\textbf{Problem}: Compute 1+2.
\textbf{Solution}: The sum of consecutive integers 1 and 2 is 3.
\end{example}

\begin{example}[Easy 2]
\textbf{Problem}: Compute 2+3.
\textbf{Solution}: The sum of consecutive integers 2 and 3 is 5.
\end{example}

\begin{example}[Easy 3]
\textbf{Problem}: Compute 3+4.
\textbf{Solution}: The sum of consecutive integers 3 and 4 is 7.
\end{example}

\begin{example}[Easy 4]
\textbf{Problem}: Compute 4+5.
\textbf{Solution}: The sum of consecutive integers 4 and 5 is 9.
\end{example}

\begin{example}[Easy 5]
\textbf{Problem}: Compute 5+6.
\textbf{Solution}: The sum of consecutive integers 5 and 6 is 11.
\end{example}

\begin{example}[Easy 6]
\textbf{Problem}: Compute 6+7.
\textbf{Solution}: The sum of consecutive integers 6 and 7 is 13.
\end{example}

\begin{example}[Easy 7]
\textbf{Problem}: Compute 7+8.
\textbf{Solution}: The sum of consecutive integers 7 and 8 is 15.
\end{example}

\begin{example}[Easy 8]
\textbf{Problem}: Compute 8+9.
\textbf{Solution}: The sum of consecutive integers 8 and 9 is 17.
\end{example}

\begin{example}[Easy 9]
\textbf{Problem}: Compute 9+10.
\textbf{Solution}: The sum of consecutive integers 9 and 10 is 19.
\end{example}

\begin{example}[Easy 10]
\textbf{Problem}: Compute 10+11.
\textbf{Solution}: The sum of consecutive integers 10 and 11 is 21.
\end{example}

\begin{example}[Easy 11]
\textbf{Problem}: Compute 11+12.
\textbf{Solution}: The sum of consecutive integers 11 and 12 is 23.
\end{example}

\begin{example}[Easy 12]
\textbf{Problem}: Compute 12+13.
\textbf{Solution}: The sum of consecutive integers 12 and 13 is 25.
\end{example}

\begin{example}[Easy 13]
\textbf{Problem}: Compute 13+14.
\textbf{Solution}: The sum of consecutive integers 13 and 14 is 27.
\end{example}

\begin{example}[Easy 14]
\textbf{Problem}: Compute 14+15.
\textbf{Solution}: The sum of consecutive integers 14 and 15 is 29.
\end{example}

\begin{example}[Easy 15]
\textbf{Problem}: Compute 15+16.
\textbf{Solution}: The sum of consecutive integers 15 and 16 is 31.
\end{example}

\begin{example}[Easy 16]
\textbf{Problem}: Compute 16+17.
\textbf{Solution}: The sum of consecutive integers 16 and 17 is 33.
\end{example}

\begin{example}[Easy 17]
\textbf{Problem}: Compute 17+18.
\textbf{Solution}: The sum of consecutive integers 17 and 18 is 35.
\end{example}

\begin{example}[Easy 18]
\textbf{Problem}: Compute 18+19.
\textbf{Solution}: The sum of consecutive integers 18 and 19 is 37.
\end{example}

\begin{example}[Easy 19]
\textbf{Problem}: Compute 19+20.
\textbf{Solution}: The sum of consecutive integers 19 and 20 is 39.
\end{example}

\begin{example}[Easy 20]
\textbf{Problem}: Compute 20+21.
\textbf{Solution}: The sum of consecutive integers 20 and 21 is 41.
\end{example}

\begin{example}[Easy 21]
\textbf{Problem}: Compute 21+22.
\textbf{Solution}: The sum of consecutive integers 21 and 22 is 43.
\end{example}

\begin{example}[Easy 22]
\textbf{Problem}: Compute 22+23.
\textbf{Solution}: The sum of consecutive integers 22 and 23 is 45.
\end{example}

\begin{example}[Easy 23]
\textbf{Problem}: Compute 23+24.
\textbf{Solution}: The sum of consecutive integers 23 and 24 is 47.
\end{example}

\begin{example}[Easy 24]
\textbf{Problem}: Compute 24+25.
\textbf{Solution}: The sum of consecutive integers 24 and 25 is 49.
\end{example}

\begin{example}[Easy 25]
\textbf{Problem}: Compute 25+26.
\textbf{Solution}: The sum of consecutive integers 25 and 26 is 51.
\end{example}

\subsection{Mild Daily Hard}
\begin{example}[Moderate 1]
\textbf{Problem}: Solve the linear equation 1x+2=3.
\textbf{Solution}: Subtract 2 from both sides to get 1x=1. Dividing by 1 gives x=1.0.
\end{example}

\begin{example}[Moderate 2]
\textbf{Problem}: Solve the linear equation 2x+3=4.
\textbf{Solution}: Subtract 3 from both sides to get 2x=1. Dividing by 2 gives x=0.5.
\end{example}

\begin{example}[Moderate 3]
\textbf{Problem}: Solve the linear equation 3x+4=5.
\textbf{Solution}: Subtract 4 from both sides to get 3x=1. Dividing by 3 gives x=0.3333333333333333.
\end{example}

\begin{example}[Moderate 4]
\textbf{Problem}: Solve the linear equation 4x+5=6.
\textbf{Solution}: Subtract 5 from both sides to get 4x=1. Dividing by 4 gives x=0.25.
\end{example}

\begin{example}[Moderate 5]
\textbf{Problem}: Solve the linear equation 5x+6=7.
\textbf{Solution}: Subtract 6 from both sides to get 5x=1. Dividing by 5 gives x=0.2.
\end{example}

\begin{example}[Moderate 6]
\textbf{Problem}: Solve the linear equation 6x+7=8.
\textbf{Solution}: Subtract 7 from both sides to get 6x=1. Dividing by 6 gives x=0.16666666666666666.
\end{example}

\begin{example}[Moderate 7]
\textbf{Problem}: Solve the linear equation 7x+8=9.
\textbf{Solution}: Subtract 8 from both sides to get 7x=1. Dividing by 7 gives x=0.14285714285714285.
\end{example}

\begin{example}[Moderate 8]
\textbf{Problem}: Solve the linear equation 8x+9=10.
\textbf{Solution}: Subtract 9 from both sides to get 8x=1. Dividing by 8 gives x=0.125.
\end{example}

\begin{example}[Moderate 9]
\textbf{Problem}: Solve the linear equation 9x+10=11.
\textbf{Solution}: Subtract 10 from both sides to get 9x=1. Dividing by 9 gives x=0.1111111111111111.
\end{example}

\begin{example}[Moderate 10]
\textbf{Problem}: Solve the linear equation 10x+11=12.
\textbf{Solution}: Subtract 11 from both sides to get 10x=1. Dividing by 10 gives x=0.1.
\end{example}

\begin{example}[Moderate 11]
\textbf{Problem}: Solve the linear equation 11x+12=13.
\textbf{Solution}: Subtract 12 from both sides to get 11x=1. Dividing by 11 gives x=0.09090909090909091.
\end{example}

\begin{example}[Moderate 12]
\textbf{Problem}: Solve the linear equation 12x+13=14.
\textbf{Solution}: Subtract 13 from both sides to get 12x=1. Dividing by 12 gives x=0.08333333333333333.
\end{example}

\begin{example}[Moderate 13]
\textbf{Problem}: Solve the linear equation 13x+14=15.
\textbf{Solution}: Subtract 14 from both sides to get 13x=1. Dividing by 13 gives x=0.07692307692307693.
\end{example}

\begin{example}[Moderate 14]
\textbf{Problem}: Solve the linear equation 14x+15=16.
\textbf{Solution}: Subtract 15 from both sides to get 14x=1. Dividing by 14 gives x=0.07142857142857142.
\end{example}

\begin{example}[Moderate 15]
\textbf{Problem}: Solve the linear equation 15x+16=17.
\textbf{Solution}: Subtract 16 from both sides to get 15x=1. Dividing by 15 gives x=0.06666666666666667.
\end{example}

\begin{example}[Moderate 16]
\textbf{Problem}: Solve the linear equation 16x+17=18.
\textbf{Solution}: Subtract 17 from both sides to get 16x=1. Dividing by 16 gives x=0.0625.
\end{example}

\begin{example}[Moderate 17]
\textbf{Problem}: Solve the linear equation 17x+18=19.
\textbf{Solution}: Subtract 18 from both sides to get 17x=1. Dividing by 17 gives x=0.058823529411764705.
\end{example}

\begin{example}[Moderate 18]
\textbf{Problem}: Solve the linear equation 18x+19=20.
\textbf{Solution}: Subtract 19 from both sides to get 18x=1. Dividing by 18 gives x=0.05555555555555555.
\end{example}

\begin{example}[Moderate 19]
\textbf{Problem}: Solve the linear equation 19x+20=21.
\textbf{Solution}: Subtract 20 from both sides to get 19x=1. Dividing by 19 gives x=0.05263157894736842.
\end{example}

\begin{example}[Moderate 20]
\textbf{Problem}: Solve the linear equation 20x+21=22.
\textbf{Solution}: Subtract 21 from both sides to get 20x=1. Dividing by 20 gives x=0.05.
\end{example}

\begin{example}[Moderate 21]
\textbf{Problem}: Solve the linear equation 21x+22=23.
\textbf{Solution}: Subtract 22 from both sides to get 21x=1. Dividing by 21 gives x=0.047619047619047616.
\end{example}

\begin{example}[Moderate 22]
\textbf{Problem}: Solve the linear equation 22x+23=24.
\textbf{Solution}: Subtract 23 from both sides to get 22x=1. Dividing by 22 gives x=0.045454545454545456.
\end{example}

\begin{example}[Moderate 23]
\textbf{Problem}: Solve the linear equation 23x+24=25.
\textbf{Solution}: Subtract 24 from both sides to get 23x=1. Dividing by 23 gives x=0.043478260869565216.
\end{example}

\begin{example}[Moderate 24]
\textbf{Problem}: Solve the linear equation 24x+25=26.
\textbf{Solution}: Subtract 25 from both sides to get 24x=1. Dividing by 24 gives x=0.041666666666666664.
\end{example}

\begin{example}[Moderate 25]
\textbf{Problem}: Solve the linear equation 25x+26=27.
\textbf{Solution}: Subtract 26 from both sides to get 25x=1. Dividing by 25 gives x=0.04.
\end{example}

\subsection{Hard}
\begin{example}[Hard 1]
\textbf{Problem}: Evaluate \(\int x^1 \, dx\).
\textbf{Solution}: Use the power rule: \(\int x^1 dx = \frac{x^2}{2} + C\).
\end{example}

\begin{example}[Hard 2]
\textbf{Problem}: Evaluate \(\int x^2 \, dx\).
\textbf{Solution}: Use the power rule: \(\int x^2 dx = \frac{x^3}{3} + C\).
\end{example}

\begin{example}[Hard 3]
\textbf{Problem}: Evaluate \(\int x^3 \, dx\).
\textbf{Solution}: Use the power rule: \(\int x^3 dx = \frac{x^4}{4} + C\).
\end{example}

\begin{example}[Hard 4]
\textbf{Problem}: Evaluate \(\int x^4 \, dx\).
\textbf{Solution}: Use the power rule: \(\int x^4 dx = \frac{x^5}{5} + C\).
\end{example}

\begin{example}[Hard 5]
\textbf{Problem}: Evaluate \(\int x^5 \, dx\).
\textbf{Solution}: Use the power rule: \(\int x^5 dx = \frac{x^6}{6} + C\).
\end{example}

\begin{example}[Hard 6]
\textbf{Problem}: Evaluate \(\int x^6 \, dx\).
\textbf{Solution}: Use the power rule: \(\int x^6 dx = \frac{x^7}{7} + C\).
\end{example}

\begin{example}[Hard 7]
\textbf{Problem}: Evaluate \(\int x^7 \, dx\).
\textbf{Solution}: Use the power rule: \(\int x^7 dx = \frac{x^8}{8} + C\).
\end{example}

\begin{example}[Hard 8]
\textbf{Problem}: Evaluate \(\int x^8 \, dx\).
\textbf{Solution}: Use the power rule: \(\int x^8 dx = \frac{x^9}{9} + C\).
\end{example}

\begin{example}[Hard 9]
\textbf{Problem}: Evaluate \(\int x^9 \, dx\).
\textbf{Solution}: Use the power rule: \(\int x^9 dx = \frac{x^10}{10} + C\).
\end{example}

\begin{example}[Hard 10]
\textbf{Problem}: Evaluate \(\int x^10 \, dx\).
\textbf{Solution}: Use the power rule: \(\int x^10 dx = \frac{x^11}{11} + C\).
\end{example}

\begin{example}[Hard 11]
\textbf{Problem}: Evaluate \(\int x^11 \, dx\).
\textbf{Solution}: Use the power rule: \(\int x^11 dx = \frac{x^12}{12} + C\).
\end{example}

\begin{example}[Hard 12]
\textbf{Problem}: Evaluate \(\int x^12 \, dx\).
\textbf{Solution}: Use the power rule: \(\int x^12 dx = \frac{x^13}{13} + C\).
\end{example}

\begin{example}[Hard 13]
\textbf{Problem}: Evaluate \(\int x^13 \, dx\).
\textbf{Solution}: Use the power rule: \(\int x^13 dx = \frac{x^14}{14} + C\).
\end{example}

\begin{example}[Hard 14]
\textbf{Problem}: Evaluate \(\int x^14 \, dx\).
\textbf{Solution}: Use the power rule: \(\int x^14 dx = \frac{x^15}{15} + C\).
\end{example}

\begin{example}[Hard 15]
\textbf{Problem}: Evaluate \(\int x^15 \, dx\).
\textbf{Solution}: Use the power rule: \(\int x^15 dx = \frac{x^16}{16} + C\).
\end{example}

\begin{example}[Hard 16]
\textbf{Problem}: Evaluate \(\int x^16 \, dx\).
\textbf{Solution}: Use the power rule: \(\int x^16 dx = \frac{x^17}{17} + C\).
\end{example}

\begin{example}[Hard 17]
\textbf{Problem}: Evaluate \(\int x^17 \, dx\).
\textbf{Solution}: Use the power rule: \(\int x^17 dx = \frac{x^18}{18} + C\).
\end{example}

\begin{example}[Hard 18]
\textbf{Problem}: Evaluate \(\int x^18 \, dx\).
\textbf{Solution}: Use the power rule: \(\int x^18 dx = \frac{x^19}{19} + C\).
\end{example}

\begin{example}[Hard 19]
\textbf{Problem}: Evaluate \(\int x^19 \, dx\).
\textbf{Solution}: Use the power rule: \(\int x^19 dx = \frac{x^20}{20} + C\).
\end{example}

\begin{example}[Hard 20]
\textbf{Problem}: Evaluate \(\int x^20 \, dx\).
\textbf{Solution}: Use the power rule: \(\int x^20 dx = \frac{x^21}{21} + C\).
\end{example}

\begin{example}[Hard 21]
\textbf{Problem}: Evaluate \(\int x^21 \, dx\).
\textbf{Solution}: Use the power rule: \(\int x^21 dx = \frac{x^22}{22} + C\).
\end{example}

\begin{example}[Hard 22]
\textbf{Problem}: Evaluate \(\int x^22 \, dx\).
\textbf{Solution}: Use the power rule: \(\int x^22 dx = \frac{x^23}{23} + C\).
\end{example}

\begin{example}[Hard 23]
\textbf{Problem}: Evaluate \(\int x^23 \, dx\).
\textbf{Solution}: Use the power rule: \(\int x^23 dx = \frac{x^24}{24} + C\).
\end{example}

\begin{example}[Hard 24]
\textbf{Problem}: Evaluate \(\int x^24 \, dx\).
\textbf{Solution}: Use the power rule: \(\int x^24 dx = \frac{x^25}{25} + C\).
\end{example}

\begin{example}[Hard 25]
\textbf{Problem}: Evaluate \(\int x^25 \, dx\).
\textbf{Solution}: Use the power rule: \(\int x^25 dx = \frac{x^26}{26} + C\).
\end{example}

\subsection{Pro Level}
\begin{example}[Pro 1]
\textbf{Problem}: Show that the polynomial $x^3-2$ is irreducible over $\Q$.
\textbf{Solution}: Apply Eisenstein's criterion at $p=2$: all coefficients except the leading one are divisible by $2$, and the constant term $2$ is not divisible by $4$. Therefore $x^{n}-2$ is irreducible over $\Q$.
\end{example}

\begin{example}[Pro 2]
\textbf{Problem}: Show that the polynomial $x^4-2$ is irreducible over $\Q$.
\textbf{Solution}: Apply Eisenstein's criterion at $p=2$: all coefficients except the leading one are divisible by $2$, and the constant term $2$ is not divisible by $4$. Therefore $x^{n}-2$ is irreducible over $\Q$.
\end{example}

\begin{example}[Pro 3]
\textbf{Problem}: Show that the polynomial $x^5-2$ is irreducible over $\Q$.
\textbf{Solution}: Apply Eisenstein's criterion at $p=2$: all coefficients except the leading one are divisible by $2$, and the constant term $2$ is not divisible by $4$. Therefore $x^{n}-2$ is irreducible over $\Q$.
\end{example}

\begin{example}[Pro 4]
\textbf{Problem}: Show that the polynomial $x^6-2$ is irreducible over $\Q$.
\textbf{Solution}: Apply Eisenstein's criterion at $p=2$: all coefficients except the leading one are divisible by $2$, and the constant term $2$ is not divisible by $4$. Therefore $x^{n}-2$ is irreducible over $\Q$.
\end{example}

\begin{example}[Pro 5]
\textbf{Problem}: Show that the polynomial $x^7-2$ is irreducible over $\Q$.
\textbf{Solution}: Apply Eisenstein's criterion at $p=2$: all coefficients except the leading one are divisible by $2$, and the constant term $2$ is not divisible by $4$. Therefore $x^{n}-2$ is irreducible over $\Q$.
\end{example}

\begin{example}[Pro 6]
\textbf{Problem}: Show that the polynomial $x^8-2$ is irreducible over $\Q$.
\textbf{Solution}: Apply Eisenstein's criterion at $p=2$: all coefficients except the leading one are divisible by $2$, and the constant term $2$ is not divisible by $4$. Therefore $x^{n}-2$ is irreducible over $\Q$.
\end{example}

\begin{example}[Pro 7]
\textbf{Problem}: Show that the polynomial $x^9-2$ is irreducible over $\Q$.
\textbf{Solution}: Apply Eisenstein's criterion at $p=2$: all coefficients except the leading one are divisible by $2$, and the constant term $2$ is not divisible by $4$. Therefore $x^{n}-2$ is irreducible over $\Q$.
\end{example}

\begin{example}[Pro 8]
\textbf{Problem}: Show that the polynomial $x^10-2$ is irreducible over $\Q$.
\textbf{Solution}: Apply Eisenstein's criterion at $p=2$: all coefficients except the leading one are divisible by $2$, and the constant term $2$ is not divisible by $4$. Therefore $x^{n}-2$ is irreducible over $\Q$.
\end{example}

\begin{example}[Pro 9]
\textbf{Problem}: Show that the polynomial $x^11-2$ is irreducible over $\Q$.
\textbf{Solution}: Apply Eisenstein's criterion at $p=2$: all coefficients except the leading one are divisible by $2$, and the constant term $2$ is not divisible by $4$. Therefore $x^{n}-2$ is irreducible over $\Q$.
\end{example}

\begin{example}[Pro 10]
\textbf{Problem}: Show that the polynomial $x^12-2$ is irreducible over $\Q$.
\textbf{Solution}: Apply Eisenstein's criterion at $p=2$: all coefficients except the leading one are divisible by $2$, and the constant term $2$ is not divisible by $4$. Therefore $x^{n}-2$ is irreducible over $\Q$.
\end{example}

\begin{example}[Pro 11]
\textbf{Problem}: Show that the polynomial $x^13-2$ is irreducible over $\Q$.
\textbf{Solution}: Apply Eisenstein's criterion at $p=2$: all coefficients except the leading one are divisible by $2$, and the constant term $2$ is not divisible by $4$. Therefore $x^{n}-2$ is irreducible over $\Q$.
\end{example}

\begin{example}[Pro 12]
\textbf{Problem}: Show that the polynomial $x^14-2$ is irreducible over $\Q$.
\textbf{Solution}: Apply Eisenstein's criterion at $p=2$: all coefficients except the leading one are divisible by $2$, and the constant term $2$ is not divisible by $4$. Therefore $x^{n}-2$ is irreducible over $\Q$.
\end{example}

\begin{example}[Pro 13]
\textbf{Problem}: Show that the polynomial $x^15-2$ is irreducible over $\Q$.
\textbf{Solution}: Apply Eisenstein's criterion at $p=2$: all coefficients except the leading one are divisible by $2$, and the constant term $2$ is not divisible by $4$. Therefore $x^{n}-2$ is irreducible over $\Q$.
\end{example}

\begin{example}[Pro 14]
\textbf{Problem}: Show that the polynomial $x^16-2$ is irreducible over $\Q$.
\textbf{Solution}: Apply Eisenstein's criterion at $p=2$: all coefficients except the leading one are divisible by $2$, and the constant term $2$ is not divisible by $4$. Therefore $x^{n}-2$ is irreducible over $\Q$.
\end{example}

\begin{example}[Pro 15]
\textbf{Problem}: Show that the polynomial $x^17-2$ is irreducible over $\Q$.
\textbf{Solution}: Apply Eisenstein's criterion at $p=2$: all coefficients except the leading one are divisible by $2$, and the constant term $2$ is not divisible by $4$. Therefore $x^{n}-2$ is irreducible over $\Q$.
\end{example}

\begin{example}[Pro 16]
\textbf{Problem}: Show that the polynomial $x^18-2$ is irreducible over $\Q$.
\textbf{Solution}: Apply Eisenstein's criterion at $p=2$: all coefficients except the leading one are divisible by $2$, and the constant term $2$ is not divisible by $4$. Therefore $x^{n}-2$ is irreducible over $\Q$.
\end{example}

\begin{example}[Pro 17]
\textbf{Problem}: Show that the polynomial $x^19-2$ is irreducible over $\Q$.
\textbf{Solution}: Apply Eisenstein's criterion at $p=2$: all coefficients except the leading one are divisible by $2$, and the constant term $2$ is not divisible by $4$. Therefore $x^{n}-2$ is irreducible over $\Q$.
\end{example}

\begin{example}[Pro 18]
\textbf{Problem}: Show that the polynomial $x^20-2$ is irreducible over $\Q$.
\textbf{Solution}: Apply Eisenstein's criterion at $p=2$: all coefficients except the leading one are divisible by $2$, and the constant term $2$ is not divisible by $4$. Therefore $x^{n}-2$ is irreducible over $\Q$.
\end{example}

\begin{example}[Pro 19]
\textbf{Problem}: Show that the polynomial $x^21-2$ is irreducible over $\Q$.
\textbf{Solution}: Apply Eisenstein's criterion at $p=2$: all coefficients except the leading one are divisible by $2$, and the constant term $2$ is not divisible by $4$. Therefore $x^{n}-2$ is irreducible over $\Q$.
\end{example}

\begin{example}[Pro 20]
\textbf{Problem}: Show that the polynomial $x^22-2$ is irreducible over $\Q$.
\textbf{Solution}: Apply Eisenstein's criterion at $p=2$: all coefficients except the leading one are divisible by $2$, and the constant term $2$ is not divisible by $4$. Therefore $x^{n}-2$ is irreducible over $\Q$.
\end{example}

\begin{example}[Pro 21]
\textbf{Problem}: Show that the polynomial $x^23-2$ is irreducible over $\Q$.
\textbf{Solution}: Apply Eisenstein's criterion at $p=2$: all coefficients except the leading one are divisible by $2$, and the constant term $2$ is not divisible by $4$. Therefore $x^{n}-2$ is irreducible over $\Q$.
\end{example}

\begin{example}[Pro 22]
\textbf{Problem}: Show that the polynomial $x^24-2$ is irreducible over $\Q$.
\textbf{Solution}: Apply Eisenstein's criterion at $p=2$: all coefficients except the leading one are divisible by $2$, and the constant term $2$ is not divisible by $4$. Therefore $x^{n}-2$ is irreducible over $\Q$.
\end{example}

\begin{example}[Pro 23]
\textbf{Problem}: Show that the polynomial $x^25-2$ is irreducible over $\Q$.
\textbf{Solution}: Apply Eisenstein's criterion at $p=2$: all coefficients except the leading one are divisible by $2$, and the constant term $2$ is not divisible by $4$. Therefore $x^{n}-2$ is irreducible over $\Q$.
\end{example}

\begin{example}[Pro 24]
\textbf{Problem}: Show that the polynomial $x^26-2$ is irreducible over $\Q$.
\textbf{Solution}: Apply Eisenstein's criterion at $p=2$: all coefficients except the leading one are divisible by $2$, and the constant term $2$ is not divisible by $4$. Therefore $x^{n}-2$ is irreducible over $\Q$.
\end{example}

\begin{example}[Pro 25]
\textbf{Problem}: Show that the polynomial $x^27-2$ is irreducible over $\Q$.
\textbf{Solution}: Apply Eisenstein's criterion at $p=2$: all coefficients except the leading one are divisible by $2$, and the constant term $2$ is not divisible by $4$. Therefore $x^{n}-2$ is irreducible over $\Q$.
\end{example}

% === Examples for Section 13: 100 Worked Examples (auto-generated) ===
\begin{example}\label{ex:sec13-1}
\textbf{Problem.} Compute $1+1$.

\textbf{Solution.} The sum is $2$.
\end{example}

\begin{example}\label{ex:sec13-2}
\textbf{Problem.} Compute $2\cdot3$.

\textbf{Solution.} The product is $6$.
\end{example}

\begin{example}\label{ex:sec13-3}
\textbf{Problem.} Solve $2x=6$.

\textbf{Solution.} $x=3$.
\end{example}

\begin{example}\label{ex:sec13-4}
\textbf{Problem.} Factor $x^2-1$.

\textbf{Solution.} $(x-1)(x+1)$.
\end{example}

\begin{example}\label{ex:sec13-5}
\textbf{Problem.} Compute $2^3$.

\textbf{Solution.} $2^3=8$.
\end{example}

\begin{example}\label{ex:sec13-6}
\textbf{Problem.} Compute $3!$.

\textbf{Solution.} $3!=6$.
\end{example}

\begin{example}\label{ex:sec13-7}
\textbf{Problem.} Solve $x^2=9$.

\textbf{Solution.} $x=\pm3$.
\end{example}

\begin{example}\label{ex:sec13-8}
\textbf{Problem.} Find $\gcd(6,9)$.

\textbf{Solution.} The gcd is $3$.
\end{example}

\begin{example}\label{ex:sec13-9}
\textbf{Problem.} Find $\operatorname{lcm}(4,6)$.

\textbf{Solution.} The lcm is $12$.
\end{example}

\begin{example}\label{ex:sec13-10}
\textbf{Problem.} Compute $1+2+3$.

\textbf{Solution.} The sum is $6$.
\end{example}

\section{Figures and diagrams}

\subsection{Fundamental Theorem lattice (schematic)}
\[
\begin{tikzcd}
& E \arrow[dash]{dl}[swap]{H} \arrow[dash]{dr}{G} & \\
E^{H} \arrow[dash]{dr} & & E^{G} \arrow[dash]{dl} \\
& F &
\end{tikzcd}
\]

\subsection{Field tower}
\[
\begin{tikzcd}
K \arrow[dash]{d}{[K\!:\!F]} \\
F
\end{tikzcd}
\]
Place external images in \texttt{figures/} and include with \verb|\includegraphics|.

% === Examples for Section 14: Figures and diagrams (auto-generated) ===
\begin{example}\label{ex:sec14-1}
\textbf{Problem.} Does the diagram $A\xrightarrow{\mathrm{id}}A$ commute?

\textbf{Solution.} Yes, the single identity map trivially commutes.
\end{example}

\begin{example}\label{ex:sec14-2}
\textbf{Problem.} In a triangle $A\to B\to C$ with compositions, does $(A\to C)$ equal the composite?

\textbf{Solution.} Yes, by definition of composition.
\end{example}

\begin{example}\label{ex:sec14-3}
\textbf{Problem.} What matrix represents $90^\circ$ rotation in the plane?

\textbf{Solution.} $\begin{pmatrix}0&-1\\1&0\end{pmatrix}$.
\end{example}

\begin{example}\label{ex:sec14-4}
\textbf{Problem.} What is the adjacency matrix of a single edge connecting two vertices?

\textbf{Solution.} $\begin{pmatrix}0&1\\1&0\end{pmatrix}$.
\end{example}

\begin{example}\label{ex:sec14-5}
\textbf{Problem.} Does a square of zero maps commute?

\textbf{Solution.} Yes, all compositions are zero.
\end{example}

\begin{example}\label{ex:sec14-6}
\textbf{Problem.} Draw a short exact sequence diagram.

\textbf{Solution.} $0\to A\to B\to C\to0$ with exact arrows.
\end{example}

\begin{example}\label{ex:sec14-7}
\textbf{Problem.} Can inclusion maps $K\subset L\subset M$ form a commutative triangle?

\textbf{Solution.} Yes, inclusions compose to the larger inclusion.
\end{example}

\begin{example}\label{ex:sec14-8}
\textbf{Problem.} Does the field diagram $\mathbb{Q}\subset\mathbb{Q}(\sqrt{2})\subset\mathbb{Q}(\sqrt{2},i)$ commute?

\textbf{Solution.} Yes, inclusions compose correctly.
\end{example}

\begin{example}\label{ex:sec14-9}
\textbf{Problem.} For group homomorphisms $G\xrightarrow{f}H\xrightarrow{g}K$, does $g\circ f$ define a commuting triangle with $G\to K$?

\textbf{Solution.} Yes, the composite map commutes.
\end{example}

\begin{example}\label{ex:sec14-10}
\textbf{Problem.} Does a commutative square of abelian groups ensure equal composites?

\textbf{Solution.} Yes, commutativity means both paths give the same map.
\end{example}


\clearpage
\printbibliography

\end{document}
