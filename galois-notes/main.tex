\documentclass[12pt,a4paper]{article}
\usepackage[utf8]{inputenc}
\usepackage[T1]{fontenc}
\usepackage{lmodern}
\usepackage{geometry}
\geometry{margin=1in}
\usepackage{microtype}
\usepackage{amsmath,amsthm,amssymb,mathtools}
\usepackage{tikz-cd}
\usepackage{graphicx}
\usepackage{xcolor}
\usepackage{hyperref}
\usepackage[nameinlink,capitalize]{cleveref}
\graphicspath{{figures/}}
\theoremstyle{plain}
\newtheorem{theorem}{Theorem}[section]
\newtheorem{lemma}[theorem]{Lemma}
\newtheorem{proposition}[theorem]{Proposition}
\newtheorem{corollary}[theorem]{Corollary}
\theoremstyle{definition}
\newtheorem{definition}[theorem]{Definition}
\newtheorem{example}[theorem]{Example}
\theoremstyle{remark}
\newtheorem{remark}[theorem]{Remark}
\newcommand{\Gal}{\mathrm{Gal}}
\newcommand{\Aut}{\mathrm{Aut}}
\newcommand{\F}{\mathbb{F}}
\newcommand{\Q}{\mathbb{Q}}
\newcommand{\Z}{\mathbb{Z}}
\newcommand{\C}{\mathbb{C}}
\newcommand{\N}{\mathbb{N}}
\newcommand{\K}{\mathbb{K}}
\title{Graduate Notes on Galois Theory}
\author{Jay}
\date{\today}
\begin{document}
\maketitle
\tableofcontents
\section{Study Plan and Roadmap}

\subsection*{Prerequisites}
\begin{itemize}
  \item Linear algebra: eigenvalues, minimal polynomials, Jordan form.
  \item Basic ring and field theory: ideals, PID/UFD, polynomial rings.
  \item Group theory: actions, Sylow, simple groups $A_n,S_n$.
\end{itemize}

\subsection*{Learning Outcomes}
\begin{itemize}
  \item Compute splitting fields and Galois groups for key polynomials.
  \item Apply the Fundamental Theorem of Galois Theory (FTGT).
  \item Decide solvability by radicals and explain Abel--Ruffini.
  \item Work with finite fields and infinite Galois groups (Krull topology).
\end{itemize}

\subsection*{Graduate-Level Outline}
\begin{enumerate}
  \item Fields and morphisms; irreducibility tests.
  \item Extensions; algebraic vs.\ transcendental; minimal polynomials.
  \item Splitting fields and algebraic closures.
  \item Separable/inseparable; perfect fields.
  \item Normal extensions; Primitive Element Theorem.
  \item Field automorphisms; Galois extensions and fixed fields.
  \item FTGT: lattice correspondence; normality and separability conditions.
  \item Classical examples: quadratics, cyclotomics, Kummer, finite fields.
  \item Solvability by radicals; solvable groups; Abel--Ruffini.
  \item Discriminants and resolvents.
  \item Finite fields $\F_{p^n}$; Frobenius and Galois group structure.
  \item Infinite Galois theory; profinite groups; Krull topology.
  \item Cyclotomic fields; basic results and computations.
  \item Directions for research and advanced topics (brief notes).
\end{enumerate}

\subsection*{Study Rhythm}
\begin{itemize}
  \item Problem-first: compute Galois groups for degrees $2$--$5$.
  \item Diagram-first: use \texttt{tikz-cd} to track subfield lattices.
  \item Proof drills: restate and prove FTGT variants.
\end{itemize}

\section{Fields and Morphisms}

\subsection{Definitions}
\begin{definition}[Field]
A \emph{field} $K$ is a commutative ring with $1\neq 0$ in which every nonzero element has a multiplicative inverse.
\end{definition}

\begin{definition}[Characteristic and prime subfield]
The \emph{characteristic} of $K$ is the unique $n\in\{0,2,3,4,\dots\}$ such that $n\cdot 1_K=0$ and no smaller positive integer has this property.  
The \emph{prime subfield} of $K$ is the smallest subfield containing $1$, hence is isomorphic to $\Q$ if $\operatorname{char}K=0$ and to $\F_p$ if $\operatorname{char}K=p>0$.
\end{definition}

\begin{proposition}[Field maps are injective]
A unital ring homomorphism $\varphi:K\to L$ between fields is injective. 
\end{proposition}
\begin{proof}
$\ker\varphi$ is a proper ideal of $K$. Fields have only the zero ideal, hence $\ker\varphi=0$.
\end{proof}

\begin{proposition}[Finite domains are fields]
Every finite integral domain is a field.
\end{proposition}
\begin{proof}
For nonzero $a$, the map $x\mapsto ax$ is injective on the finite set $K$, hence surjective, so $1=ax$ for some $x$ and $a$ is invertible.
\end{proof}

\subsection{Polynomials over a field}
Let $K[x]$ be the polynomial ring. Degree $\deg f$ is the highest power with nonzero coefficient. A root $\alpha\in L$ (in any extension $L/K$) satisfies $f(\alpha)=0$.

\begin{proposition}[Degree $\le 3$ test]
If $f\in K[x]$ has degree $2$ or $3$, then $f$ is irreducible over $K$ iff it has no root in $K$.
\end{proposition}

\begin{theorem}[Rational Root Test]
For $f(x)=a_nx^n+\cdots+a_0\in\Z[x]$ with $a_0\neq 0$, any rational root $\frac{p}{q}$ in lowest terms satisfies $p\mid a_0$ and $q\mid a_n$.
\end{theorem}

\begin{theorem}[Eisenstein]
Let $f(x)=\sum_{i=0}^na_ix^i\in\Z[x]$. If there exists a prime $p$ with $p\mid a_i$ for $i<n$, $p\nmid a_n$, and $p^2\nmid a_0$, then $f$ is irreducible over $\Q$.
\end{theorem}
\begin{proof}[Proof sketch]
Reduce mod $p$ to get $x^n$ up to a nonzero scalar; a nontrivial factorization over $\Q$ would force $p^2\mid a_0$ by Gauss's lemma.\footnote{Gauss's lemma: a primitive factorization in $\Q[x]$ lifts to one in $\Z[x]$.}
\end{proof}

\subsection{Examples}
\begin{example}
$\Q,\R,\C,\F_p$ are fields; $\Z$ is not a field. The rational function field $K(t)$ is a field containing $K$.
\end{example}
\begin{example}[Irreducibility by Eisenstein]
$x^5-2\in\Q[x]$ is irreducible using $p=2$. Hence $[\Q(\sqrt[5]{2}):\Q]=5$.
\end{example}
\begin{example}[Degree $\le 3$ test]
$f(x)=x^3-3x-1$ has no rational root by the Rational Root Test, hence is irreducible over $\Q$.
\end{example}

\subsection{Remarks}
\begin{remark}
In $K[x]$ every irreducible polynomial generates a maximal ideal; $K[x]$ is a PID iff $K$ is a field and one restricts to univariate polynomials.
\end{remark}
\begin{remark}
We will view all fields inside a fixed algebraic closure when convenient.
\end{remark}

\subsection{A tiny diagram}
\[
\begin{tikzcd}
\F_p \arrow[r, hook] \arrow[d, hook] & \F_p(t) \\
\F_{p^n} &
\end{tikzcd}
\]

\section{Extensions and Minimal Polynomials}

\subsection{Field extensions and degree}
\begin{definition}
A \emph{field extension} is an inclusion $K\subseteq L$ of fields. The \emph{degree} is $[L:K]=\dim_K L$ when finite.
\end{definition}

\begin{proposition}[Tower law]
If $K\subseteq L\subseteq M$, then $[M:K]=[M:L]\,[L:K]$ whenever the numbers are finite.
\end{proposition}
\begin{proof}
Choose a $K$-basis of $L$ and an $L$-basis of $M$; the products form a $K$-basis of $M$.
\end{proof}

\subsection{Algebraic and transcendental elements}
\begin{definition}
$\alpha\in L$ is \emph{algebraic over $K$} if there exists nonzero $f\in K[x]$ with $f(\alpha)=0$. Otherwise $\alpha$ is \emph{transcendental}.  
An extension $L/K$ is \emph{algebraic} if every element of $L$ is algebraic over $K$.
\end{definition}

\begin{definition}[Minimal polynomial]
If $\alpha$ is algebraic over $K$, the \emph{minimal polynomial} $m_\alpha(x)\in K[x]$ is the unique monic irreducible polynomial with $m_\alpha(\alpha)=0$.
\end{definition}

\begin{theorem}[Simple extensions]\label{thm:simple}
If $\alpha$ is algebraic over $K$, then $K(\alpha)\cong K[x]/(m_\alpha)$ via $x\mapsto \alpha$, and $[K(\alpha):K]=\deg m_\alpha$.
\end{theorem}
\begin{proof}
Define $\varphi:K[x]\to K(\alpha)$ by evaluation at $\alpha$. Then $\ker\varphi=(m_\alpha)$ and the image is $K(\alpha)$. Apply the First Isomorphism Theorem; dimension equals $\deg m_\alpha$.
\end{proof}

\begin{proposition}
If $f\in K[x]$ is irreducible, then $K[x]/(f)$ is a field, and adjoining any root $\alpha$ of $f$ gives $K(\alpha)\cong K[x]/(f)$.
\end{proposition}

\subsection{Worked examples}
\begin{example}
$\alpha=\sqrt{2}$ over $\Q$: $m_\alpha=x^2-2$, so $[\Q(\alpha):\Q]=2$ and $\{1,\alpha\}$ is a basis.
\end{example}
\begin{example}
$\beta=\sqrt[3]{2}$ over $\Q$: $m_\beta=x^3-2$, so $[\Q(\beta):\Q]=3$. The tower $\Q\subset \Q(\beta)\subset \Q(\beta,\omega)$, where $\omega$ is a primitive cube root of unity, will reappear as a splitting field.
\end{example}
\begin{example}[Adjoining multiple elements]
If $\alpha,\beta$ are algebraic over $K$, then $K(\alpha,\beta)$ is finite over $K$ with
\[
[K(\alpha,\beta):K]\le [K(\alpha):K]\,[K(\beta):K(\alpha)]\le (\deg m_\alpha)(\deg m_\beta).
\]
\end{example}

\subsection{Transcendence and rational functions}
\begin{example}
If $t$ is transcendental over $K$, then $K(t)$ has a $K$-basis $\{1,t,t^2,\dots\}$ as a vector space is false; indeed $K(t)$ is infinite-dimensional over $K$ but generated as a field by $t$ and $K$.\footnote{Beware: $K(t)$ is not a finite extension of $K$.}
\end{example}

\subsection{Remarks}
\begin{remark}
Every algebraic extension is a directed union of finite simple extensions by repeatedly adjoining elements.
\end{remark}

\section{Splitting Fields and Algebraic Closures}

\subsection{Splitting fields}
\begin{definition}
Let $f\in K[x]$ be nonconstant. A field $E\supseteq K$ is a \emph{splitting field} of $f$ over $K$ if (i) $f$ factors in $E[x]$ as a product of linear terms and (ii) $E$ is generated over $K$ by the roots of $f$.
\end{definition}

\begin{theorem}[Existence]\label{thm:splitting-existence}
Every $f\in K[x]$ has a splitting field over $K$, and any two splitting fields are $K$-isomorphic.
\end{theorem}
\begin{proof}[Proof sketch]
Adjoin one root $\alpha_1$ in some algebraic closure to get $K(\alpha_1)$. Factor $f=(x-\alpha_1)g$ and adjoin a root of $g$; repeat finitely many times.  
For uniqueness, use induction on $\deg f$ and extend $K$-embeddings stepwise.
\end{proof}

\begin{proposition}[Embedding bound]
Let $E/K$ be generated by $n$ distinct roots of $f\in K[x]$ with $\deg f=n$. Then the number of $K$-embeddings $E\hookrightarrow \overline{K}$ is at most $n!$. If $f$ has distinct roots, then $[E:K]\le n!$.
\end{proposition}
\begin{proof}[Idea]
An embedding permutes the roots. The image of $E$ is determined by the image of a generating tuple of roots, giving a subgroup of $S_n$.
\end{proof}

\subsection{Algebraic closures}
\begin{definition}
An \emph{algebraic closure} $\overline{K}$ of $K$ is an algebraic extension in which every nonconstant polynomial in $K[x]$ splits completely.
\end{definition}
\begin{theorem}[Existence and uniqueness]
Every field has an algebraic closure, unique up to $K$-isomorphism.
\end{theorem}

\subsection{Examples}
\begin{example}[Cubic $x^3-2$]
Over $\Q$, $f=x^3-2$ has one real root $\beta=\sqrt[3]{2}$ and two complex roots $\omega\beta,\omega^2\beta$ where $\omega=e^{2\pi i/3}$.  
A splitting field is $E=\Q(\beta,\omega)$. One checks $[\Q(\omega):\Q]=2$, $[\Q(\beta):\Q]=3$, and $[\Q(\beta,\omega):\Q]=6$ by the tower law, so $E/\Q$ is degree $6$.
\end{example}

\begin{example}[Cyclotomic quartic]
$f=x^4+1$ over $\Q$ splits in $E=\Q(\zeta_8)$ where $\zeta_8=e^{2\pi i/8}$. Here $E/\Q$ has degree $\varphi(8)=4$ and the roots are $\zeta_8^{\pm1},\zeta_8^{\pm3}$.
\end{example}

\subsection{A small lattice picture}
\[
\begin{tikzcd}
& \Q(\beta,\omega) \arrow[-]{dl} \arrow[-]{dr} & \\
\Q(\beta) \arrow[-]{dr} & & \Q(\omega) \arrow[-]{dl} \\
& \Q &
\end{tikzcd}
\]
\begin{remark}
The dashed look indicates intermediate fields; this picture is only suggestive.\footnote{Accurate lattice structure will be developed after FTGT.}
\end{remark}

\section{Separable and Perfect Fields}

\subsection{Multiple roots and the derivative}
For $f=\sum a_ix^i\in K[x]$, define $f'=\sum ia_ix^{i-1}$. 

\begin{proposition}[Multiple root criterion]\label{prop:gcd}
Let $L\supseteq K$. A root $\alpha\in L$ of $f$ has multiplicity $>1$ iff $f(\alpha)=f'(\alpha)=0$. Equivalently, $f$ has a repeated root in a splitting field iff $\gcd(f,f')\neq 1$ in $K[x]$.
\end{proposition}
\begin{proof}
If $f=(x-\alpha)^2g$, then $f'=2(x-\alpha)g+(x-\alpha)^2g'$ so $f(\alpha)=f'(\alpha)=0$.  
Conversely, if $f(\alpha)=f'(\alpha)=0$, divide $f$ by $(x-\alpha)$ to see $(x-\alpha)$ also divides $f'$, hence $(x-\alpha)^2\mid f$.
\end{proof}

\subsection{Separable polynomials and extensions}
\begin{definition}
A polynomial $f\in K[x]$ is \emph{separable} over $K$ if it has no repeated root in a splitting field, i.e.\ $\gcd(f,f')=1$.  
An element $\alpha$ algebraic over $K$ is \emph{separable} if $m_\alpha$ is separable.  
An algebraic extension $L/K$ is \emph{separable} if every element of $L$ is separable over $K$.
\end{definition}

\begin{theorem}[Embeddings count]\label{thm:sep-emb}
Let $L/K$ be finite. Then the following are equivalent:
\begin{enumerate}
\item $L/K$ is separable.
\item The number of $K$-embeddings $L\hookrightarrow \overline{K}$ equals $[L:K]$.
\item For each $\alpha\in L$, the roots of $m_\alpha$ in $\overline{K}$ are distinct.
\end{enumerate}
\end{theorem}
\begin{proof}[Idea]
For simple $L=K(\alpha)$, embeddings correspond to roots of $m_\alpha$; count equals $\deg m_\alpha$ precisely when the roots are distinct. Use the tower law to extend to general finite $L/K$.
\end{proof}

\subsection{Perfect fields}
\begin{definition}
A field $K$ of characteristic $p>0$ is \emph{perfect} if every element is a $p$-th power, equivalently the Frobenius $x\mapsto x^p$ is surjective. By convention, all fields of characteristic $0$ are perfect.
\end{definition}
\begin{proposition}
Finite fields and characteristic $0$ fields are perfect.
\end{proposition}
\begin{proof}
If $K$ is finite of size $q=p^n$, the Frobenius map $x\mapsto x^p$ is injective on a finite set, hence bijective; thus every finite extension of $\F_p$ is separable. Characteristic $0$ is perfect because $f'=0$ forces $f$ to be constant.
\end{proof}

\subsection{Examples}
\begin{example}[Inseparable polynomial]
Over $K=\F_p(t)$, the polynomial $x^p-t$ is irreducible but not separable since $(x^p-t)'=0$; it has a unique root $t^{1/p}$ of multiplicity $p$ in a splitting field.
\end{example}
\begin{example}[Separable binomials]
Over a field $K$ of characteristic $p>0$, the polynomial $x^p-a$ is separable iff $a\notin K^p$.
\end{example}

\begin{remark}
When $L/K$ is finite, ``Galois'' will mean \emph{normal and separable}. The separability part is verified by \cref{thm:sep-emb}.
\end{remark}

\section{Normal Extensions and the Primitive Element Theorem}

\subsection{Normal extensions}

\begin{definition}[Normality]
A finite extension $L/K$ is \emph{normal} if every $K$-embedding $\sigma:L\hookrightarrow \overline{K}$ satisfies $\sigma(L)=L$. Equivalently, $L$ is the splitting field over $K$ of a family of polynomials in $K[x]$.
\end{definition}

\begin{proposition}[Equivalent criteria]
For a finite extension $L/K$, the following are equivalent:
\begin{enumerate}
\item $L/K$ is normal.
\item Every irreducible $f\in K[x]$ having a root in $L$ splits completely over $L$.
\item $L$ is the splitting field over $K$ of some $f\in K[x]$.
\end{enumerate}
\end{proposition}
\begin{proof}[Proof sketch]
$(1)\Rightarrow(2)$: If $\alpha\in L$ is a root of irreducible $f$, the other roots are $\sigma(\alpha)$ for $K$-embeddings $\sigma:L\to\overline{K}$. Normality forces $\sigma(\alpha)\in L$.  
$(2)\Rightarrow(3)$: Take a finite set of generators $\alpha_i$ of $L$ and let $f=\prod m_{\alpha_i}$.  
$(3)\Rightarrow(1)$: Any $K$-embedding permutes the roots of $f$ and hence preserves the field they generate.
\end{proof}

\subsection{Primitive Element Theorem}

\begin{theorem}[PET, separable case]
If $L/K$ is finite and separable, then $L=K(\alpha)$ for some $\alpha\in L$.
\end{theorem}
\begin{proof}[Idea]
Write $L=K(\alpha_1,\alpha_2)$. For $c\in K$ outside a finite exceptional set, the fields $K(\alpha_1+c\alpha_2)$ and $K(\alpha_1,\alpha_2)$ coincide. Induct on the number of generators.
\end{proof}

\begin{remark}[What can fail]
Without separability the theorem may fail. For instance, in characteristic $p>0$ one can have a finite purely inseparable extension $K\subset L$ with $L=K(\alpha,\beta)$ where $\alpha^p,\beta^p\in K$ but $L\neq K(\gamma)$ for any $\gamma\in L$. This cannot happen in characteristic $0$ nor for finite separable extensions.
\end{remark}

\begin{corollary}
If $L/K$ is finite, separable, and normal, then $L$ is the splitting field of the minimal polynomial of a single element $\alpha$ and $L=K(\alpha)$.
\end{corollary}

\subsection{Examples}

\begin{example}[$\Q(\sqrt{2},\sqrt{3})$]
Over $\Q$, $L=\Q(\sqrt{2},\sqrt{3})$ is separable. By PET,
\[
L=\Q\big(\sqrt{2}+\sqrt{3}\big)
\]
since $m_{\sqrt{2}+\sqrt{3}}(x)=x^4-10x^2+1$ and its splitting field over $\Q$ is $L$.
\end{example}

\begin{example}[A normal but not Galois over a subfield]
Let $E=\Q(\zeta_8)$ and $K=\Q(i)$. Then $E/K$ is normal and cyclic of degree $2$; however $E/\Q$ is also normal, while $\Q(i)/\Q$ is not normal as the irreducible $x^2+1$ has both roots $\pm i$ but $\Q(i)$ contains only one of them as an element, not all conjugates over $\Q$ of elements of $\Q(i)$.
\end{example}

\subsection{Lattice picture}
\[
\begin{tikzcd}
& K(\alpha_1,\alpha_2)=K(\alpha_1+c\alpha_2) \arrow[-]{dl} \arrow[-]{dr} & \\
K(\alpha_1) \arrow[-]{dr} & & K(\alpha_2) \arrow[-]{dl} \\
& K &
\end{tikzcd}
\]
\begin{remark}
For all but finitely many $c\in K$, the top field equals $K(\alpha_1+c\alpha_2)$.
\end{remark}

\section{Automorphisms and Galois Extensions}

\subsection{Automorphisms and fixed fields}

\begin{definition}
For an extension $L/K$, a \emph{$K$-automorphism} of $L$ is a field automorphism $\sigma:L\to L$ fixing $K$ pointwise. The \emph{Galois group} is
\[
\Gal(L/K)=\{\sigma\in \Aut(L)\mid \sigma|_K=\mathrm{id}_K\}.
\]
For any subgroup $H\le \Aut(L)$, the \emph{fixed field} is $L^H=\{x\in L:\sigma(x)=x\ \forall\sigma\in H\}$.
\end{definition}

\begin{proposition}[Basic properties]
$L^H$ is a subfield of $L$ containing $K$. If $H_1\le H_2$ then $L^{H_2}\subseteq L^{H_1}$.
\end{proposition}

\begin{theorem}[Artin's fixed field theorem]
If $H$ is a finite subgroup of $\Aut(L)$, then $L/L^H$ is finite and separable with $|\Gal(L/L^H)|=|H|$. In particular, $K=L^{\Gal(L/K)}$ when $L/K$ is finite and normal.
\end{theorem}
\begin{proof}[Idea]
Show that distinct automorphisms are linearly independent as endomorphisms of $L$ (Artin's lemma). Then $\dim_{L^H}L\ge|H|$. The opposite inequality follows by counting embeddings.
\end{proof}

\subsection{Finite Galois extensions}

\begin{definition}
A finite extension $L/K$ is \emph{Galois} if it is both normal and separable. Equivalently, $|\Gal(L/K)|=[L:K]$.
\end{definition}

\begin{proposition}[Automorphisms act on roots]
Let $f\in K[x]$ and $E$ a splitting field. Then $\Gal(E/K)$ permutes the roots of $f$. If $E/K$ is Galois, the action is faithful.
\end{proposition}

\begin{corollary}
If $E/K$ is the splitting field of a separable polynomial $f$ of degree $n$, then $\Gal(E/K)$ embeds into $S_n$ and $[E:K]\mid n!$.
\end{corollary}

\subsection{Examples}

\begin{example}[Quadratic]
For $L=\Q(\sqrt{d})/\Q$ with squarefree $d$, $\Gal(L/\Q)=\{\mathrm{id},\ \sqrt{d}\mapsto -\sqrt{d}\}\cong C_2$.
\end{example}

\begin{example}[Finite fields]
For $L=\F_{p^n}$ over $K=\F_p$, the Frobenius $\varphi:x\mapsto x^p$ generates $\Gal(L/K)\cong C_n$. The fixed field of $\langle \varphi^m\rangle$ is $\F_{p^{\gcd(n,m)}}$.
\end{example}

\begin{example}[Cyclotomic]
For $L=\Q(\zeta_n)$, $\Gal(L/\Q)\cong (\Z/n\Z)^\times$ via $\sigma_a(\zeta_n)=\zeta_n^a$.
\end{example}

\begin{example}[Cubic $x^3-2$]
If $E$ is the splitting field of $x^3-2$ over $\Q$, then $\Gal(E/\Q)\cong S_3$ and $[E:\Q]=6$; complex conjugation is a transposition.
\end{example}

\subsection{A small picture}
\[
\begin{tikzcd}
& L \arrow[-]{dl}[swap]{\text{fix }H} \arrow[-]{dr}{\text{fix }G} & \\
L^{H} \arrow[-]{dr} & & L^{G}=K \arrow[-]{dl} \\
& K &
\end{tikzcd}
\]
\begin{remark}
Larger subgroups give smaller fixed fields. This reversal drives the Fundamental Theorem next.
\end{remark}

\section{Fundamental Theorem of Galois Theory}

Let $L/K$ be a finite Galois extension with group $G=\Gal(L/K)$.

\subsection{Statement}

\begin{theorem}[FTGT]
The maps
\[
\Phi:\{\text{subgroups }H\le G\}\to\{\text{intermediate fields }E,\ K\subseteq E\subseteq L\},\quad H\mapsto L^{H},
\]
\[
\Psi:\{E\}\to\{H\},\quad E\mapsto \Gal(L/E),
\]
are inclusion-reversing inverses. Moreover:
\begin{enumerate}
\item $[E:K]=|G:\Gal(L/E)|$ and $[L:E]=|\Gal(L/E)|$.
\item $E/K$ is Galois $\iff$ $\Gal(L/E)\trianglelefteq G$, and then
\[
\Gal(E/K)\cong G/\Gal(L/E).
\]
\end{enumerate}
\end{theorem}

\begin{proof}[Proof sketch]
Artin's theorem gives $K=L^{G}$. If $H\le G$, then $L/L^{H}$ is Galois with group $H$, hence $|\Gal(L/L^H)|=|H|$ and indices match degrees. For any intermediate $E$, clearly $E\subseteq L^{\Gal(L/E)}$ and equality follows by counting degrees. Normality of $E/K$ corresponds to conjugation-stability of $\Gal(L/E)$.
\end{proof}

\subsection{Consequences}

\begin{corollary}[Galois correspondence algebra]
The lattice of subgroups of $G$ is anti-isomorphic to the lattice of intermediate fields of $L/K$. Intersections and joins correspond to composita and intersections of fields.
\end{corollary}

\begin{corollary}[Fixed field of a normal subgroup]
If $N\trianglelefteq G$, then $L^{N}/K$ is Galois with group $G/N$.
\end{corollary}

\subsection{Worked example: $x^3-2$}

Let $E$ be the splitting field of $x^3-2$ over $\Q$. Then $E=\Q(\sqrt[3]{2},\omega)$ and $G\cong S_3$.
\[
\begin{tikzcd}
& E \arrow[-]{dl} \arrow[-]{d} \arrow[-]{dr} & \\
\Q(\sqrt[3]{2}) \arrow[-]{dr} & \Q(\omega) \arrow[-]{d} & \Q(\sqrt[3]{2}\omega) \arrow[-]{dl} \\
& \Q &
\end{tikzcd}
\]
The three index-$2$ subgroups of $S_3$ fix the three quadratic subfields, while the three index-$3$ subgroups fix the three cubic subfields. Complex conjugation corresponds to a transposition.

\subsection{Another example: a quartic with dihedral group}

For $f=x^4-2$ over $\Q$, the splitting field $E=\Q(\sqrt[4]{2},i)$ has $G\cong D_4$ of order $8$. The unique Klein $V_4\trianglelefteq D_4$ fixes $\Q(\sqrt{2})$, hence $\Q(\sqrt{2})/\Q$ is Galois with group $C_2$, and $E/\Q(\sqrt{2})$ is a biquadratic extension.

\subsection{Remarks}

\begin{remark}
The correspondence fails without separability or normality: in general, taking fixed fields of subgroups of \emph{all} automorphisms of $L$ over $K$ still yields intermediate fields, but surjectivity and index-degree equalities can fail.
\end{remark}

\section{Classical Examples}

\subsection{Quadratics and biquadratics}
\begin{example}
For $d$ squarefree, $L=\Q(\sqrt{d})$ is Galois of degree $2$ with group $C_2$.  
For $d_1,d_2$ distinct and squarefree, $E=\Q(\sqrt{d_1},\sqrt{d_2})$ has
\[
\Gal(E/\Q)\cong V_4,\qquad E=\Q\big(\sqrt{d_1},\sqrt{d_2}\big),
\]
and the three quadratic subfields correspond to the three order-$2$ subgroups.
\end{example}

\subsection{Cubic $x^3-2$}
\begin{example}
Let $f=x^3-2$. The discriminant is $\Delta(f)=-4(-2)^3-27(1)^2=(-4)(-8)-27=32-27=5$. Since $\Delta(f)$ is not a square in $\Q$, the Galois group of the splitting field is $S_3$; $[E:\Q]=6$ and $E=\Q(\sqrt[3]{2},\omega)$.
\end{example}

\subsection{Cyclotomic fields}
\begin{example}
For a prime $p$, the $p$-th cyclotomic field $K=\Q(\zeta_p)$ has degree $\varphi(p)=p-1$ and
\[
\Gal\big(\Q(\zeta_p)/\Q\big)\cong(\Z/p\Z)^\times\cong C_{p-1}.
\]
The unique quadratic subfield for $p\equiv1\pmod 4$ is fixed by the unique index-$2$ subgroup.
\end{example}

\subsection{Finite fields}
\begin{example}
Let $q=p^n$. The polynomial $x^{q}-x$ in $\F_q[x]$ splits with distinct roots and its roots are exactly the elements of $\F_q$. For $m\mid n$, the subextensions of $\F_{p^n}/\F_p$ are precisely $\F_{p^m}$.
\end{example}

\subsection{A quartic}
\begin{example}
For $f=x^4-2$, $E=\Q(\sqrt[4]{2},i)$ and $\Gal(E/\Q)\cong D_4$. The quadratic subfields are $\Q(\sqrt{2})$, $\Q(i)$, and $\Q(\sqrt{-2})$.
\[
\begin{tikzcd}
& E \arrow[-]{dl} \arrow[-]{d} \arrow[-]{dr} & \\
\Q(\sqrt[4]{2}) \arrow[-]{dr} & \Q(i,\sqrt{2}) \arrow[-]{d} & \Q(i\sqrt[4]{2}) \arrow[-]{dl} \\
& \Q(\sqrt{2}) \arrow[-]{d} & \\
& \Q &
\end{tikzcd}
\]
\end{example}

\subsection{Worked computations}

\begin{example}[Deciding $A_n$ vs $S_n$ by the discriminant]
Let $E$ be the splitting field of a separable $f\in\Q[x]$ of degree $n$. If the discriminant $\Delta(f)$ is a square in $\Q$, then $\Gal(E/\Q)\subseteq A_n$; otherwise it is not contained in $A_n$.
\end{example}

\begin{example}[A solvable quartic]
For $f=x^4+ax^2+b$ with $a^2\ne4b$, the resolvent quadratic $y^2-2ay+(a^2-4b)$ decides the structure: when it splits in $\Q$, the Galois group is contained in $V_4$; otherwise it is $D_4$.
\end{example}

\section{Solvability by Radicals and Abel--Ruffini}

\subsection{Definitions}

\begin{definition}[Radical extension]
An extension $L/K$ is \emph{radical} if there is a tower
\[
K=K_0\subset K_1\subset\cdots\subset K_m=L
\]
with $K_{i}=K_{i-1}(\alpha_i)$ and $\alpha_i^{n_i}\in K_{i-1}$ for some $n_i\ge2$, after adjoining appropriate roots of unity.
\end{definition}

\begin{definition}[Solvable by radicals]
A polynomial $f\in K[x]$ is \emph{solvable by radicals over $K$} if its roots lie in a field obtained from $K$ by a sequence of radical extensions.
\end{definition}

\begin{definition}[Solvable groups]
A finite group $G$ is \emph{solvable} if it has a subnormal series
\[
\{1\}=G_0\trianglelefteq G_1\trianglelefteq\cdots\trianglelefteq G_r=G
\]
with each quotient $G_{i}/G_{i-1}$ abelian.
\end{definition}

\subsection{Galois theory bridge}

\begin{theorem}[Galois $\Rightarrow$ radicals]
Assume $\operatorname{char}K=0$ (or more generally that the needed roots of unity are present). If $f\in K[x]$ is solvable by radicals and $E$ is its splitting field, then $\Gal(E/K)$ is solvable.
\end{theorem}

\begin{theorem}[Radicals $\Leftarrow$ Galois]
Conversely, if $\Gal(E/K)$ is solvable and $K$ contains all roots of unity of order dividing $|\,\Gal(E/K)\,|$, then every element of $E$ can be expressed by radicals over $K$.
\end{theorem}

\begin{corollary}[Abel--Ruffini]
The general quintic polynomial is not solvable by radicals over $\Q$. Equivalently, there exist irreducible quintics over $\Q$ whose splitting-field Galois group is $S_5$, which is not solvable.
\end{corollary}

\subsection{Examples}

\begin{example}[Cubic]
All irreducible cubics over $\Q$ are solvable by radicals because subgroups of $S_3$ are solvable. The casus irreducibilis explains why real radicals may require complex numbers.
\end{example}

\begin{example}[A solvable quintic]
$f(x)=x^5-1$ factors over $\Q$ as $(x-1)\Phi_5(x)$ with $\Gal(\Q(\zeta_5)/\Q)\cong C_4$ solvable; hence its roots are radical over $\Q$.
\end{example}

\begin{example}[A non-solvable quintic]
$f(x)=x^5-6x+3$ has Galois group $S_5$ over $\Q$ (can be shown using modulo reductions and the discriminant). Hence $f$ is not solvable by radicals.
\end{example}

\subsection{Remarks}
\begin{remark}
When roots of unity are missing, use Kummer theory to adjoin them first; solvability is then tested on the enlarged base field.
\end{remark}

\section{Discriminants and Resolvents}

\subsection{Discriminant of a polynomial}

\begin{definition}
For a monic separable $f(x)=\prod_{i=1}^n (x-\alpha_i)\in K[x]$, the \emph{discriminant} is
\[
\Delta(f)=\prod_{i<j}(\alpha_i-\alpha_j)^2\in K.
\]
It equals $(-1)^{n(n-1)/2}\, \frac{1}{a_n}\operatorname{Res}(f,f')$ for a general leading coefficient $a_n$.
\end{definition}

\begin{proposition}[Parity test]
Let $E$ be the splitting field of a separable $f\in K[x]$ of degree $n$ over a field of characteristic $\neq 2$. Then
\[
\Delta(f)\text{ is a square in }K \iff \Gal(E/K)\subseteq A_n.
\]
\end{proposition}
\begin{proof}[Idea]
A permutation of the roots acts on $\sqrt{\Delta(f)}$ by its sign.
\end{proof}

\subsection{Discriminant of number fields}
\begin{definition}
If $L/K$ is a finite separable extension, the \emph{relative discriminant} $\mathfrak{D}_{L/K}$ is defined via any $K$-basis and the trace form; in characteristic $0$ it is an ideal of the ring of integers $\mathcal{O}_K$. For $K=\Q$ one gets an integer $\mathrm{disc}(L)$.
\end{definition}

\begin{example}
For $L=\Q(\sqrt{d})$ with squarefree $d$,
\[
\mathrm{disc}(L)=
\begin{cases}
d & d\equiv 1\pmod 4,\\
4d & d\equiv 2,3\pmod 4.
\end{cases}
\]
\end{example}

\subsection{Resolvents}

\begin{definition}[Resolvent idea]
A \emph{resolvent} is a polynomial whose roots are certain symmetric expressions in the roots of $f$, designed so that the action of $\Gal(E/K)$ can be detected by the factorization of the resolvent over $K$.
\end{definition}

\paragraph{Cubic resolvent for quartics.}
For $f(x)=x^4+ax^3+bx^2+cx+d$ define the \emph{cubic resolvent}
\[
R(y)=y^3-2by^2+(b^2+ac-4d)y+(4bd-a^2d-c^2).
\]
\begin{itemize}
\item If $R$ is irreducible and $\Delta(f)$ is not a square, then $\Gal(E/K)\cong D_4$.
\item If $R$ splits but $\Delta(f)$ not a square, then $\Gal(E/K)\cong V_4$ or $C_4$ (decide by the square status of certain coefficients).
\item If $\Delta(f)$ is a square and $R$ irreducible, then $\Gal(E/K)\subset A_4$ and often equals $A_4$.
\end{itemize}

\paragraph{Quadratic resolvent for cubics.}
For $f(x)=x^3+ax^2+bx+c$, the \emph{discriminant} already decides $A_3$ vs.\ $S_3$; one may also form the quadratic resolvent $y^2-4by+(a b-3c)^2$ to aid explicit formulas.

\subsection{Examples}

\begin{example}[Quartic $x^4-2$]
Here $a=c=0$, $b=0$, $d=-2$. Then $R(y)=y^3+8$ is irreducible over $\Q$, and $\Delta(f)=-2^{11}$ is not a square. Hence $\Gal(E/\Q)\cong D_4$.
\end{example}

\begin{example}[A biquadratic]
For $f(x)=x^4-5x^2+6=(x^2-2)(x^2-3)$, the discriminant is a square and the resolvent splits completely, giving $\Gal\cong V_4$.
\end{example}

\subsection{Remarks}
\begin{remark}
Resolvents are a practical classification tool up to degree $4$. For higher degrees, group-based tests, reductions modulo primes, and transitivity arguments are standard.
\end{remark}

\section{Finite Fields}
\begin{itemize}
  \item Structure of $\F_{p^n}$; Frobenius automorphism.
  \item Galois group cyclic of order $n$.
\end{itemize}

\section{Infinite Galois Theory}
\begin{itemize}
  \item Profinite groups; Krull topology.
  \item Infinite FTGT statement.
\end{itemize}

\section{Cyclotomic Fields}
\begin{itemize}
  \item $\Q(\zeta_n)$ basics; $\Gal(\Q(\zeta_n)/\Q)\cong(\Z/n\Z)^\times$.
\end{itemize}

\section{Pictures}
% Put figures into the figures/ folder. Example template:
%
% \begin{figure}[h]
%   \centering
%   \includegraphics[width=0.7\linewidth]{example-figure}
%   \caption{A sample field lattice.}
% \end{figure}
%
% Keep image filenames extensionless in \includegraphics; LaTeX picks .pdf/.png.

\section*{References}
\begin{itemize}
  \item M.\ Artin, \emph{Algebra}.
  \item I.\ Stewart \& D.\ Tall, \emph{Algebraic Number Theory and Fermat's Last Theorem}.
  \item Dummit \& Foote, \emph{Abstract Algebra}, Ch.\ 13--14.
  \item S.\ Lang, \emph{Algebra}.
\end{itemize}

\end{document}
