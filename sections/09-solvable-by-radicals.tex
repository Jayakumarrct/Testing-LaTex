\section{Solvability by Radicals and Abel--Ruffini}

\subsection{Definitions}

\begin{definition}[Radical extension]
An extension $L/K$ is \emph{radical} if there is a tower
\[
K=K_0\subset K_1\subset\cdots\subset K_m=L
\]
with $K_{i}=K_{i-1}(\alpha_i)$ and $\alpha_i^{n_i}\in K_{i-1}$ for some $n_i\ge2$, after adjoining appropriate roots of unity.
\end{definition}

\begin{definition}[Solvable by radicals]
A polynomial $f\in K[x]$ is \emph{solvable by radicals over $K$} if its roots lie in a field obtained from $K$ by a sequence of radical extensions.
\end{definition}

\begin{definition}[Solvable groups]
A finite group $G$ is \emph{solvable} if it has a subnormal series
\[
\{1\}=G_0\trianglelefteq G_1\trianglelefteq\cdots\trianglelefteq G_r=G
\]
with each quotient $G_{i}/G_{i-1}$ abelian.
\end{definition}

\subsection{Galois theory bridge}

\begin{theorem}[Galois $\Rightarrow$ radicals]
Assume $\operatorname{char}K=0$ (or more generally that the needed roots of unity are present). If $f\in K[x]$ is solvable by radicals and $E$ is its splitting field, then $\Gal(E/K)$ is solvable.
\end{theorem}

\begin{theorem}[Radicals $\Leftarrow$ Galois]
Conversely, if $\Gal(E/K)$ is solvable and $K$ contains all roots of unity of order dividing $|\,\Gal(E/K)\,|$, then every element of $E$ can be expressed by radicals over $K$.
\end{theorem}

\begin{corollary}[Abel--Ruffini]
The general quintic polynomial is not solvable by radicals over $\Q$. Equivalently, there exist irreducible quintics over $\Q$ whose splitting-field Galois group is $S_5$, which is not solvable.
\end{corollary}

\subsection{Examples}

\begin{example}[Cubic]
All irreducible cubics over $\Q$ are solvable by radicals because subgroups of $S_3$ are solvable. The casus irreducibilis explains why real radicals may require complex numbers.
\end{example}

\begin{example}[A solvable quintic]
$f(x)=x^5-1$ factors over $\Q$ as $(x-1)\Phi_5(x)$ with $\Gal(\Q(\zeta_5)/\Q)\cong C_4$ solvable; hence its roots are radical over $\Q$.
\end{example}

\begin{example}[A non-solvable quintic]
$f(x)=x^5-6x+3$ has Galois group $S_5$ over $\Q$ (can be shown using modulo reductions and the discriminant). Hence $f$ is not solvable by radicals.
\end{example}

\subsection{Remarks}
\begin{remark}
When roots of unity are missing, use Kummer theory to adjoin them first; solvability is then tested on the enlarged base field.
\end{remark}
