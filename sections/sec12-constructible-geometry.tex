\section{Classical geometric constructibility}\label{sec:constructibility}

\subsection{Constructible numbers}
\begin{definition}
A complex number $\alpha$ is \emph{constructible} if it can be obtained from $0,1$ using a finite sequence of straightedge-and-compass constructions.
\end{definition}
\begin{theorem}
$\alpha$ is constructible iff there exists a tower
\[
\Q=K_0\subset K_1\subset\cdots\subset K_m\ni \alpha,\qquad
[K_i\!:\!K_{i-1}]=2.
\]
In particular, every constructible $\alpha$ is algebraic over $\Q$ of degree a power of $2$.
\end{theorem}

\subsection{Regular polygons}
\begin{theorem}[Gauss--Wantzel]
A regular $n$-gon is constructible iff
\[
n=2^{k} \prod_{j=1}^r p_j,
\]
where the $p_j$ are distinct Fermat primes $3,5,17,257,65537$.
\end{theorem}
\begin{example}
The $17$-gon is constructible, while the $7$-gon and $9$-gon are not.
\end{example}

\subsection{Classical impossibilities}
\begin{corollary}[Doubling the cube]
There is no straightedge-and-compass construction of $\sqrt[3]{2}$; the degree is $3\not=2^k$.
\end{corollary}
\begin{corollary}[Angle trisection]
A general angle cannot be trisected; e.g.\ $\cos(20^\circ)$ has minimal polynomial of degree $3$ over $\Q$.
\end{corollary}

\subsection{Galois viewpoint}
\begin{remark}
Constructible numbers lie in the compositum of quadratic extensions of $\Q$; thus their splitting fields have Galois groups that are $2$-groups. Nonconstructibility follows by showing the relevant Galois group has odd prime divisors.
\end{remark}

