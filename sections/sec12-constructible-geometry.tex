\section{Classical geometric constructibility}\label{sec:constructibility}

\subsection{Constructible numbers}
\begin{definition}
A complex number $\alpha$ is \emph{constructible} if it can be obtained from $0,1$ using a finite sequence of straightedge-and-compass constructions.
\end{definition}
\begin{theorem}
$\alpha$ is constructible iff there exists a tower
\[
\Q=K_0\subset K_1\subset\cdots\subset K_m\ni \alpha,\qquad [K_i\!:\!K_{i-1}]=2.
\]
Hence any constructible $\alpha$ is algebraic of degree a power of $2$.
\end{theorem}
\begin{proof}
Straightedge-and-compass constructions use only field operations and extraction of square roots. Starting from $\Q$, each step therefore yields a quadratic extension, so any constructed number lies in a tower of such extensions and has degree a power of $2$. Conversely, given a tower with degrees $2$, adjoining the required square roots one by one constructs any element in the top field.
\end{proof}

\subsection{Regular polygons}
\begin{theorem}[Gauss–Wantzel]
A regular $n$-gon is constructible iff $n=2^k \prod p_j$ where the $p_j$ are distinct Fermat primes $3,5,17,257,65537$.
\end{theorem}
\begin{proof}
Constructing a regular $n$-gon is equivalent to constructing $\zeta_n=e^{2\pi i/n}$. By the previous theorem this is possible exactly when $[\Q(\zeta_n):\Q]=\varphi(n)$ is a power of $2$. The totient $\varphi(n)$ is a power of $2$ precisely for numbers of the given form with distinct Fermat primes.
\end{proof}
\begin{example}
The $17$-gon is constructible; the $7$-gon and $9$-gon are not.
\end{example}

\subsection{Classical impossibilities}
\begin{corollary}[Doubling the cube]
No straightedge-and-compass construction of $\sqrt[3]{2}$ exists since $\deg(\sqrt[3]{2}/\Q)=3\ne 2^k$.
\end{corollary}
\begin{proof}
If $\sqrt[3]{2}$ were constructible, its degree over $\Q$ would be a power of $2$. The polynomial $x^3-2$ is irreducible over $\Q$, so the degree is $3$, a contradiction.
\end{proof}
\begin{corollary}[Angle trisection]
A general angle cannot be trisected; e.g.\ $\cos(20^\circ)$ has minimal polynomial of degree $3$.
\end{corollary}
\begin{proof}
Trisecting an arbitrary angle would allow the construction of $\cos(20^\circ)$. Its minimal polynomial $8x^3-6x-1$ is irreducible over $\Q$ and has degree $3$, so this number is not constructible. Therefore a general angle cannot be trisected.
\end{proof}
Reference: \cite[\S13]{DF}.
