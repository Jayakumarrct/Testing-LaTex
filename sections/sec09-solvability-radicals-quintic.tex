\section{Solvability by radicals and the quintic}\label{sec:radicals-quintic}

\subsection{Radical extensions and solvable groups}
\begin{definition}
An extension $E/F$ is \emph{radical} if there is a tower
\[
F=F_0\subset F_1\subset\cdots\subset F_m=E,\qquad
F_i=F_{i-1}\bigl(\sqrt[n_i]{a_i}\bigr)
\]
with $n_i\ge2$, $a_i\in F_{i-1}^\times$.
\end{definition}
\begin{definition}
A finite group $G$ is \emph{solvable} if $1=G_0\trianglelefteq\cdots\trianglelefteq G_r=G$ with abelian factors $G_i/G_{i-1}$.
\end{definition}

\subsection{Galois criterion}
\begin{theorem}\label{thm:radicals-criterion}
Assume $\operatorname{char}F=0$ and $f\in F[x]$ is separable. Then $f$ is solvable by radicals over $F$ iff $\Gal(E/F)$ is solvable, where $E$ is the splitting field of $f$.
\end{theorem}
\begin{proof}[Idea]
($\Rightarrow$) Normalize a radical tower and apply Kummer theory after adjoining roots of unity, producing abelian quotients. ($\Leftarrow$) Induct along a solvable series realizing each abelian quotient by radicals. See \cite[Ch.~14]{DF}, \cite[Ch.~VI]{Artin}.
\end{proof}
\begin{theorem}[Abel–Ruffini]
The general quintic over $\Q$ is not solvable by radicals.
\end{theorem}

\subsection{Detecting $S_n$}
\begin{proposition}
If $f\in\Q[x]$ is irreducible of prime degree $p$ with exactly two nonreal roots, then $\Gal(f)\cong S_p$.
\end{proposition}
\begin{proof}[Idea]
Complex conjugation acts as a transposition; any transitive subgroup of $S_p$ containing a transposition is $S_p$.
\end{proof}
