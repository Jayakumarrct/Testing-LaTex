\section{Solvability by radicals and the quintic}\label{sec:radicals-quintic}

\subsection{Radical extensions}
\begin{definition}
An extension $E/F$ is \emph{radical} if there is a tower
\[
F=F_0\subset F_1\subset\cdots\subset F_m=E,\qquad
F_{i}=F_{i-1}\bigl(\sqrt[n_i]{a_i}\bigr)
\]
with $n_i\ge2$ and $a_i\in F_{i-1}^\times$. A polynomial $f\in F[x]$ is \emph{solvable by radicals} if its roots lie in some radical extension of $F$.
\end{definition}

\subsection{Solvable groups}
\begin{definition}
A finite group $G$ is \emph{solvable} if it admits a subnormal series
\[
1=G_0\trianglelefteq G_1\trianglelefteq\cdots\trianglelefteq G_r=G
\]
with abelian quotients $G_{i}/G_{i-1}$.
\end{definition}

\subsection{Galois-theoretic criterion}
\begin{theorem}
Assume $\operatorname{char}F=0$ and let $f\in F[x]$ be separable. Then $f$ is solvable by radicals over $F$ iff its Galois group
$G=\Gal(E/F)$ over the splitting field $E$ is solvable.
\end{theorem}
\begin{proof}[Idea]
($\Rightarrow$) Take the normal closure of a radical tower and use Kummer theory (after adjoining the needed roots of unity) to obtain a series with abelian quotients. ($\Leftarrow$) Induct along a solvable series and adjoin radicals realizing the abelian steps.
\end{proof}

\subsection{Consequences}
\begin{corollary}
Quadratics and cubics are solvable by radicals; quartics are solvable by radicals via Ferrari's method; the general quintic is not.
\end{corollary}
\begin{theorem}[Abel--Ruffini]
The generic polynomial of degree $5$ over $\Q$ is not solvable by radicals.
\end{theorem}

\subsection{Detecting $S_n$}
\begin{proposition}
Let $f\in\Q[x]$ be irreducible of prime degree $p$ with exactly two nonreal roots. Then $\Gal(f)\cong S_p$.
\end{proposition}
\begin{proof}[Idea]
Complex conjugation is a transposition in the Galois group. Irreducibility forces a transitive subgroup of $S_p$ containing a transposition, hence $S_p$.
\end{proof}
\begin{example}
Many quintics of the form $x^5+ax+b$ have Galois group $S_5$; such polynomials are not solvable by radicals.
\end{example}

