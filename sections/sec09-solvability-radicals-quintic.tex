\section{Solvability by radicals and the quintic}\label{sec:radicals-quintic}

\subsection{Radical extensions and solvable groups}
\begin{definition}
An extension $E/F$ is \emph{radical} if there is a tower
\[
F=F_0\subset F_1\subset\cdots\subset F_m=E,\qquad
F_i=F_{i-1}\bigl(\sqrt[n_i]{a_i}\bigr)
\]
with $n_i\ge2$, $a_i\in F_{i-1}^\times$.
\end{definition}
\begin{definition}
A finite group $G$ is \emph{solvable} if $1=G_0\trianglelefteq\cdots\trianglelefteq G_r=G$ with abelian factors $G_i/G_{i-1}$.
\end{definition}

\subsection{Galois criterion}
\begin{theorem}\label{thm:radicals-criterion}
Assume $\operatorname{char}F=0$ and $f\in F[x]$ is separable. Then $f$ is solvable by radicals over $F$ iff $\Gal(E/F)$ is solvable, where $E$ is the splitting field of $f$.
\end{theorem}
\begin{proof}
Assume $f$ is solvable by radicals over $F$. Take a radical tower
\[
F=F_0\subset\cdots\subset F_m=E,\qquad F_i=F_{i-1}(\alpha_i),\ \alpha_i^{n_i}\in F_{i-1}
\]
after adjoining roots of unity. Each step is a Kummer extension with cyclic Galois group. Intersecting the normal closures of these extensions yields a normal series of $\Gal(E/F)$ with abelian quotients, so $\Gal(E/F)$ is solvable.

Conversely, let $G=\Gal(E/F)$ be solvable and choose
\[
1=G_0\trianglelefteq\cdots\trianglelefteq G_r=G
\]
with abelian factors. Put $E_i=E^{G_i}$. Then $F=E_0\subset\cdots\subset E_r=E$ and each $E_i/E_{i-1}$ is Galois with abelian Galois group $G_i/G_{i-1}$. Such extensions are generated by radicals, so concatenating these gives a radical tower for $E/F$.
\end{proof}
\begin{theorem}[Abel–Ruffini]
The general quintic over $\Q$ is not solvable by radicals.
\end{theorem}
\begin{proof}
The splitting field of the general quintic has Galois group $S_5$. Since $S_5$ is not solvable, the previous theorem implies the polynomial is not solvable by radicals.
\end{proof}

\subsection{Detecting $S_n$}
\begin{proposition}
If $f\in\Q[x]$ is irreducible of prime degree $p$ with exactly two nonreal roots, then $\Gal(f)\cong S_p$.
\end{proposition}
\begin{proof}
Let $E$ be the splitting field of $f$ and $G=\Gal(E/\Q)$. Irreducibility of $f$ and the primeness of $p$ make $G$ act transitively on the roots. Complex conjugation fixes the real roots and swaps the two nonreal roots, giving a transposition in $G$. Any transitive subgroup of $S_p$ containing a transposition is $S_p$, so $G\cong S_p$.
\end{proof}
