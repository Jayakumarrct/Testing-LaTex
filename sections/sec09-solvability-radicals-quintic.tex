\section{Solvability by radicals and the quintic}\label{sec:radicals-quintic}

\subsection{Radical extensions and solvable groups}
\begin{definition}
An extension $E/F$ is \emph{radical} if there is a tower
\[
F=F_0\subset F_1\subset\cdots\subset F_m=E,\qquad
F_i=F_{i-1}\bigl(\sqrt[n_i]{a_i}\bigr)
\]
with $n_i\ge2$, $a_i\in F_{i-1}^\times$.
\end{definition}
\begin{definition}
A finite group $G$ is \emph{solvable} if $1=G_0\trianglelefteq\cdots\trianglelefteq G_r=G$ with abelian factors $G_i/G_{i-1}$.
\end{definition}

\subsection{Galois criterion}
\begin{theorem}\label{thm:radicals-criterion}
Assume $\operatorname{char}F=0$ and $f\in F[x]$ is separable. Then $f$ is solvable by radicals over $F$ iff $\Gal(E/F)$ is solvable, where $E$ is the splitting field of $f$.
\end{theorem}
\begin{proof}
($\Rightarrow$) Suppose the roots of $f$ lie in a radical tower
$F=F_0\subset\cdots\subset F_m$.
After adjoining the necessary roots of unity, each step
$F_{i-1}\subset F_i$ is a Kummer extension with abelian Galois group.
The Galois group of $E/F$ is then built from these abelian quotients, so it is solvable.

($\Leftarrow$) Conversely assume $G=\Gal(E/F)$ is solvable.
Choose a normal series with abelian factors and let the corresponding
fields be $F=E_0\subset\cdots\subset E_r=E$.
Each $E_i/E_{i-1}$ is an abelian extension, hence by Kummer theory is
obtained by adjoining radicals (after adjoining roots of unity).
Thus the roots of $f$ lie in a radical tower over $F$.
\end{proof}
\begin{theorem}[Abel–Ruffini]
The general quintic over $\Q$ is not solvable by radicals.
\end{theorem}
\begin{proof}
The splitting field of the general quintic has Galois group $S_5$.
Since $S_5$ is not solvable, the criterion above shows that the general
quintic is not solvable by radicals.
\end{proof}

\subsection{Detecting $S_n$}
\begin{proposition}
If $f\in\Q[x]$ is irreducible of prime degree $p$ with exactly two nonreal roots, then $\Gal(f)\cong S_p$.
\end{proposition}
\begin{proof}
Let $G=\Gal(f)$ acting on the $p$ roots.
Irreducibility makes $G$ transitive.
Exactly two roots are nonreal, so complex conjugation swaps them and fixes the others, giving a transposition in $G$.
By Cauchy's theorem $G$ contains an element of order $p$, hence a
$p$-cycle.
A transitive subgroup of $S_p$ containing a transposition and a $p$-cycle is $S_p$, so $G\cong S_p$.
\end{proof}
