\section{Splitting fields and algebraic closure}\label{sec:splitting-fields}

\subsection{Splitting fields}
\begin{definition}
A \emph{splitting field} of $f\in F[x]$ is a minimal extension $E/F$ over which $f$ factors as linear terms.
\end{definition}
\begin{theorem}[Existence and uniqueness]
Every $f\in F[x]$ has a splitting field $E/F$, unique up to $F$-isomorphism.
\end{theorem}
\begin{proof}[Idea]
Existence by adjoining roots one by one; uniqueness by extending embeddings and induction on $\deg f$.
\end{proof}
References: \cite[\S14]{DF}, \cite[Ch.~V]{Artin}.

\subsection{Algebraic closures}
\begin{definition}
An \emph{algebraic closure} $\overline{F}$ of $F$ is an algebraic extension that is algebraically closed.
\end{definition}
\begin{theorem}
Every $F$ has an algebraic closure, unique up to $F$-isomorphism.
\end{theorem}
\begin{remark}
Construction uses Zorn's lemma; see \cite[Ch.~VIII]{Lang}.
\end{remark}

\subsection{Separability preview}
\begin{proposition}
Let $E/F$ be finite and $\sigma:E\hookrightarrow \overline{F}$ an $F$-embedding. Then $\#\{\sigma\}\le [E\!:\!F]$, with equality iff $E/F$ is separable.
\end{proposition}

\subsection{Examples}
\begin{example}
The splitting field of $x^3-2$ over $\Q$ is $\Q(\root3\of2,\zeta_3)$ of degree $6$.
\end{example}
\begin{example}
The splitting field of $x^n-1$ over $\Q$ is $\Q(\zeta_n)$; see \cref{sec:cyclotomic-kummer}.
\end{example}
