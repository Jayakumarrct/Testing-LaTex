\section{Splitting fields and algebraic closure}\label{sec:splitting-fields}

\subsection{Splitting fields}
\begin{definition}
A \emph{splitting field} of $f\in F[x]$ is a minimal extension $E/F$ over which $f$ factors as linear terms.
\end{definition}
\begin{theorem}[Existence and uniqueness]
Every $f\in F[x]$ has a splitting field $E/F$, unique up to $F$-isomorphism.
\end{theorem}
\begin{proof}
Pick a root $\alpha$ of $f$ in some extension of $F$ and let $F_1=F(\alpha)$. Then
\(f=(x-\alpha)g\) with $g\in F_1[x]$ of smaller degree, so by induction on the
degree we obtain a field $E$ over which $g$ splits. This $E$ is generated over
$F$ by adjoining all the roots of $f$, hence it is a splitting field.

For uniqueness, let $E$ and $E'$ be splitting fields of $f$. Choose a root
$\alpha\in E$ and a corresponding root $\beta\in E'$ of its minimal polynomial
over $F$. The map $\alpha\mapsto\beta$ extends to an $F$-isomorphism
$F(\alpha)\cong F(\beta)$. Applying the inductive construction to the remaining
factors of $f$ extends this to an $F$-isomorphism $E\cong E'$.
\end{proof}
References: \cite[\S14]{DF}, \cite[Ch.~V]{Artin}.

\subsection{Algebraic closures}
\begin{definition}
An \emph{algebraic closure} $\overline{F}$ of $F$ is an algebraic extension that is algebraically closed.
\end{definition}
\begin{theorem}
Every $F$ has an algebraic closure, unique up to $F$-isomorphism.
\end{theorem}
\begin{proof}
Consider the set of algebraic extensions of $F$ ordered by inclusion in which
each polynomial of $F[x]$ has at least one root. Any chain has an upper bound by
the union of its fields, so Zorn's lemma yields a maximal such extension
$\overline{F}$. If a polynomial in $\overline{F}[x]$ were irreducible with no
root in $\overline{F}$, adjoining a root would contradict maximality, hence
$\overline{F}$ is algebraically closed and algebraic over $F$.

If $\overline{F}'$ is another algebraic closure, the previous theorem gives an
$F$-isomorphism between their splitting fields for each polynomial in $F[x]$,
which patch together to an $F$-isomorphism $\overline{F}\cong\overline{F}'$.
\end{proof}
\begin{remark}
Construction uses Zorn's lemma; see \cite[Ch.~VIII]{Lang}.
\end{remark}

\subsection{Separability preview}
\begin{proposition}
Let $E/F$ be finite and $\sigma:E\hookrightarrow \overline{F}$ an $F$-embedding. Then $\#\{\sigma\}\le [E\!:\!F]$, with equality iff $E/F$ is separable.
\end{proposition}
\begin{proof}
Write $E=F(\alpha_1,\ldots,\alpha_n)$ and set $E_i=F(\alpha_1,\ldots,\alpha_i)$.
An $F$-embedding of $E_{i-1}$ extends to at most $d_i$ embeddings of $E_i$,
where $d_i$ is the number of distinct roots of the minimal polynomial of
$\alpha_i$ over $E_{i-1}$. Thus the number of $F$-embeddings of $E$ is at most
$d_1\cdots d_n=[E\!:\!F]$. Equality holds precisely when each $d_i$ equals the
degree of the corresponding minimal polynomial, which is the definition of
separability for $E/F$.
\end{proof}

\subsection{Examples}
\begin{example}
The splitting field of $x^3-2$ over $\Q$ is $\Q(\root3\of2,\zeta_3)$ of degree $6$.
\end{example}
\begin{example}
The splitting field of $x^n-1$ over $\Q$ is $\Q(\zeta_n)$; see \cref{sec:cyclotomic-kummer}.
\end{example}
