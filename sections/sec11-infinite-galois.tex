\section{Infinite Galois theory and the Krull topology}\label{sec:infinite-galois}

\subsection{Krull topology}
\begin{definition}
Let $E/F$ be Galois (possibly infinite) with group $G=\Gal(E/F)$. The \emph{Krull topology} on $G$ has neighborhood basis
\[
\mathcal{N}=\bigl\{\Gal(E/K): F\subseteq K\subseteq E,\ K/F\ \text{finite Galois}\bigr\}.
\]
Then $G$ is compact, totally disconnected, Hausdorff; i.e.\ profinite.
\end{definition}

\subsection{Infinite Fundamental Theorem}
\begin{theorem}
For $E/F$ Galois with group $G$, the maps
\[
K\mapsto \Gal(E/K),\qquad H\mapsto E^{H}
\]
give inclusion-reversing bijections between intermediate fields and \emph{closed} subgroups $H\le G$. For any $H\le G$, $\overline{H}=\Gal\bigl(E/E^{H}\bigr)$.
\end{theorem}
\begin{proof}
Let $K$ be intermediate. For every finite Galois $L$ with $F\subseteq L\subseteq K$, the classical fundamental theorem gives $\Gal(E/K)=\bigcap\Gal(E/L)$. Intersections of open subgroups are closed, so $\Gal(E/K)$ is closed and the map $K\mapsto\Gal(E/K)$ reverses inclusions.

Conversely, let $H\le G$ be closed and set $K=E^{H}$. If $\sigma$ fixes $K$, then it fixes each finite subextension of $K$, hence $\sigma\in H$ and $H=\Gal(E/K)$. The assignments are inverse bijections. For any subgroup $H$, the closure is $\overline{H}=\Gal(E/E^{H})$ by definition of the Krull topology.
\end{proof}

\subsection{Inverse limit description}
\begin{theorem}
Let $\mathcal{K}$ be the directed set of finite Galois extensions $K/F$ in $E$. Then
\[
\Gal(E/F)\ \cong\ \varprojlim_{K\in\mathcal{K}} \Gal(K/F)
\]
as topological groups.
\end{theorem}
\begin{proof}
Restriction to $K$ yields compatible surjections $\Gal(E/F)\to\Gal(K/F)$. These define a continuous map
\[
\phi:\Gal(E/F)\longrightarrow\varprojlim_{K\in\mathcal{K}}\Gal(K/F).
\]
If $\sigma\ne\tau$ in $\Gal(E/F)$, some element of $E$ is moved differently and lies in a finite $K$, so the restrictions to $K$ differ; hence $\phi$ is injective. Given a coherent family $(\sigma_K)_K$ in the inverse limit, the maps agree on overlaps and extend to an automorphism of $E$, producing an element of $\Gal(E/F)$. Thus $\phi$ is surjective and a homeomorphism.
\end{proof}
References: \cite[Ch.~VIII]{Lang}, \cite[Ch.~I]{Neukirch}.

\subsection{Examples}
\begin{example}[Finite fields]
$\Gal(\overline{\mathbb{F}}_p/\mathbb{F}_p)\cong\widehat{\Z}$ generated by Frobenius.
\end{example}
\begin{example}[Absolute Galois group]
$G_\Q=\Gal(\overline{\Q}/\Q)$ is profinite; every finite group appears as a quotient of an open subgroup.
\end{example}
