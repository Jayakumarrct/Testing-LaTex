% 100 worked examples autogenerated by script
\section{100 Worked Examples}\label{sec:100-worked-examples}

\subsection{Very Easy}
\begin{example}[Easy 1]
\textbf{Problem}: Compute 1+2.
\textbf{Solution}: The sum of consecutive integers 1 and 2 is 3.
\end{example}

\begin{example}[Easy 2]
\textbf{Problem}: Compute 2+3.
\textbf{Solution}: The sum of consecutive integers 2 and 3 is 5.
\end{example}

\begin{example}[Easy 3]
\textbf{Problem}: Compute 3+4.
\textbf{Solution}: The sum of consecutive integers 3 and 4 is 7.
\end{example}

\begin{example}[Easy 4]
\textbf{Problem}: Compute 4+5.
\textbf{Solution}: The sum of consecutive integers 4 and 5 is 9.
\end{example}

\begin{example}[Easy 5]
\textbf{Problem}: Compute 5+6.
\textbf{Solution}: The sum of consecutive integers 5 and 6 is 11.
\end{example}

\begin{example}[Easy 6]
\textbf{Problem}: Compute 6+7.
\textbf{Solution}: The sum of consecutive integers 6 and 7 is 13.
\end{example}

\begin{example}[Easy 7]
\textbf{Problem}: Compute 7+8.
\textbf{Solution}: The sum of consecutive integers 7 and 8 is 15.
\end{example}

\begin{example}[Easy 8]
\textbf{Problem}: Compute 8+9.
\textbf{Solution}: The sum of consecutive integers 8 and 9 is 17.
\end{example}

\begin{example}[Easy 9]
\textbf{Problem}: Compute 9+10.
\textbf{Solution}: The sum of consecutive integers 9 and 10 is 19.
\end{example}

\begin{example}[Easy 10]
\textbf{Problem}: Compute 10+11.
\textbf{Solution}: The sum of consecutive integers 10 and 11 is 21.
\end{example}

\begin{example}[Easy 11]
\textbf{Problem}: Compute 11+12.
\textbf{Solution}: The sum of consecutive integers 11 and 12 is 23.
\end{example}

\begin{example}[Easy 12]
\textbf{Problem}: Compute 12+13.
\textbf{Solution}: The sum of consecutive integers 12 and 13 is 25.
\end{example}

\begin{example}[Easy 13]
\textbf{Problem}: Compute 13+14.
\textbf{Solution}: The sum of consecutive integers 13 and 14 is 27.
\end{example}

\begin{example}[Easy 14]
\textbf{Problem}: Compute 14+15.
\textbf{Solution}: The sum of consecutive integers 14 and 15 is 29.
\end{example}

\begin{example}[Easy 15]
\textbf{Problem}: Compute 15+16.
\textbf{Solution}: The sum of consecutive integers 15 and 16 is 31.
\end{example}

\begin{example}[Easy 16]
\textbf{Problem}: Compute 16+17.
\textbf{Solution}: The sum of consecutive integers 16 and 17 is 33.
\end{example}

\begin{example}[Easy 17]
\textbf{Problem}: Compute 17+18.
\textbf{Solution}: The sum of consecutive integers 17 and 18 is 35.
\end{example}

\begin{example}[Easy 18]
\textbf{Problem}: Compute 18+19.
\textbf{Solution}: The sum of consecutive integers 18 and 19 is 37.
\end{example}

\begin{example}[Easy 19]
\textbf{Problem}: Compute 19+20.
\textbf{Solution}: The sum of consecutive integers 19 and 20 is 39.
\end{example}

\begin{example}[Easy 20]
\textbf{Problem}: Compute 20+21.
\textbf{Solution}: The sum of consecutive integers 20 and 21 is 41.
\end{example}

\begin{example}[Easy 21]
\textbf{Problem}: Compute 21+22.
\textbf{Solution}: The sum of consecutive integers 21 and 22 is 43.
\end{example}

\begin{example}[Easy 22]
\textbf{Problem}: Compute 22+23.
\textbf{Solution}: The sum of consecutive integers 22 and 23 is 45.
\end{example}

\begin{example}[Easy 23]
\textbf{Problem}: Compute 23+24.
\textbf{Solution}: The sum of consecutive integers 23 and 24 is 47.
\end{example}

\begin{example}[Easy 24]
\textbf{Problem}: Compute 24+25.
\textbf{Solution}: The sum of consecutive integers 24 and 25 is 49.
\end{example}

\begin{example}[Easy 25]
\textbf{Problem}: Compute 25+26.
\textbf{Solution}: The sum of consecutive integers 25 and 26 is 51.
\end{example}

\subsection{Mild Daily Hard}
\begin{example}[Moderate 1]
\textbf{Problem}: Solve the linear equation 1x+2=3.
\textbf{Solution}: Subtract 2 from both sides to get 1x=1. Dividing by 1 gives x=1.0.
\end{example}

\begin{example}[Moderate 2]
\textbf{Problem}: Solve the linear equation 2x+3=4.
\textbf{Solution}: Subtract 3 from both sides to get 2x=1. Dividing by 2 gives x=0.5.
\end{example}

\begin{example}[Moderate 3]
\textbf{Problem}: Solve the linear equation 3x+4=5.
\textbf{Solution}: Subtract 4 from both sides to get 3x=1. Dividing by 3 gives x=0.3333333333333333.
\end{example}

\begin{example}[Moderate 4]
\textbf{Problem}: Solve the linear equation 4x+5=6.
\textbf{Solution}: Subtract 5 from both sides to get 4x=1. Dividing by 4 gives x=0.25.
\end{example}

\begin{example}[Moderate 5]
\textbf{Problem}: Solve the linear equation 5x+6=7.
\textbf{Solution}: Subtract 6 from both sides to get 5x=1. Dividing by 5 gives x=0.2.
\end{example}

\begin{example}[Moderate 6]
\textbf{Problem}: Solve the linear equation 6x+7=8.
\textbf{Solution}: Subtract 7 from both sides to get 6x=1. Dividing by 6 gives x=0.16666666666666666.
\end{example}

\begin{example}[Moderate 7]
\textbf{Problem}: Solve the linear equation 7x+8=9.
\textbf{Solution}: Subtract 8 from both sides to get 7x=1. Dividing by 7 gives x=0.14285714285714285.
\end{example}

\begin{example}[Moderate 8]
\textbf{Problem}: Solve the linear equation 8x+9=10.
\textbf{Solution}: Subtract 9 from both sides to get 8x=1. Dividing by 8 gives x=0.125.
\end{example}

\begin{example}[Moderate 9]
\textbf{Problem}: Solve the linear equation 9x+10=11.
\textbf{Solution}: Subtract 10 from both sides to get 9x=1. Dividing by 9 gives x=0.1111111111111111.
\end{example}

\begin{example}[Moderate 10]
\textbf{Problem}: Solve the linear equation 10x+11=12.
\textbf{Solution}: Subtract 11 from both sides to get 10x=1. Dividing by 10 gives x=0.1.
\end{example}

\begin{example}[Moderate 11]
\textbf{Problem}: Solve the linear equation 11x+12=13.
\textbf{Solution}: Subtract 12 from both sides to get 11x=1. Dividing by 11 gives x=0.09090909090909091.
\end{example}

\begin{example}[Moderate 12]
\textbf{Problem}: Solve the linear equation 12x+13=14.
\textbf{Solution}: Subtract 13 from both sides to get 12x=1. Dividing by 12 gives x=0.08333333333333333.
\end{example}

\begin{example}[Moderate 13]
\textbf{Problem}: Solve the linear equation 13x+14=15.
\textbf{Solution}: Subtract 14 from both sides to get 13x=1. Dividing by 13 gives x=0.07692307692307693.
\end{example}

\begin{example}[Moderate 14]
\textbf{Problem}: Solve the linear equation 14x+15=16.
\textbf{Solution}: Subtract 15 from both sides to get 14x=1. Dividing by 14 gives x=0.07142857142857142.
\end{example}

\begin{example}[Moderate 15]
\textbf{Problem}: Solve the linear equation 15x+16=17.
\textbf{Solution}: Subtract 16 from both sides to get 15x=1. Dividing by 15 gives x=0.06666666666666667.
\end{example}

\begin{example}[Moderate 16]
\textbf{Problem}: Solve the linear equation 16x+17=18.
\textbf{Solution}: Subtract 17 from both sides to get 16x=1. Dividing by 16 gives x=0.0625.
\end{example}

\begin{example}[Moderate 17]
\textbf{Problem}: Solve the linear equation 17x+18=19.
\textbf{Solution}: Subtract 18 from both sides to get 17x=1. Dividing by 17 gives x=0.058823529411764705.
\end{example}

\begin{example}[Moderate 18]
\textbf{Problem}: Solve the linear equation 18x+19=20.
\textbf{Solution}: Subtract 19 from both sides to get 18x=1. Dividing by 18 gives x=0.05555555555555555.
\end{example}

\begin{example}[Moderate 19]
\textbf{Problem}: Solve the linear equation 19x+20=21.
\textbf{Solution}: Subtract 20 from both sides to get 19x=1. Dividing by 19 gives x=0.05263157894736842.
\end{example}

\begin{example}[Moderate 20]
\textbf{Problem}: Solve the linear equation 20x+21=22.
\textbf{Solution}: Subtract 21 from both sides to get 20x=1. Dividing by 20 gives x=0.05.
\end{example}

\begin{example}[Moderate 21]
\textbf{Problem}: Solve the linear equation 21x+22=23.
\textbf{Solution}: Subtract 22 from both sides to get 21x=1. Dividing by 21 gives x=0.047619047619047616.
\end{example}

\begin{example}[Moderate 22]
\textbf{Problem}: Solve the linear equation 22x+23=24.
\textbf{Solution}: Subtract 23 from both sides to get 22x=1. Dividing by 22 gives x=0.045454545454545456.
\end{example}

\begin{example}[Moderate 23]
\textbf{Problem}: Solve the linear equation 23x+24=25.
\textbf{Solution}: Subtract 24 from both sides to get 23x=1. Dividing by 23 gives x=0.043478260869565216.
\end{example}

\begin{example}[Moderate 24]
\textbf{Problem}: Solve the linear equation 24x+25=26.
\textbf{Solution}: Subtract 25 from both sides to get 24x=1. Dividing by 24 gives x=0.041666666666666664.
\end{example}

\begin{example}[Moderate 25]
\textbf{Problem}: Solve the linear equation 25x+26=27.
\textbf{Solution}: Subtract 26 from both sides to get 25x=1. Dividing by 25 gives x=0.04.
\end{example}

\subsection{Hard}
\begin{example}[Hard 1]
\textbf{Problem}: Evaluate \(\int x^1 \, dx\).
\textbf{Solution}: Use the power rule: \(\int x^1 dx = \frac{x^2}{2} + C\).
\end{example}

\begin{example}[Hard 2]
\textbf{Problem}: Evaluate \(\int x^2 \, dx\).
\textbf{Solution}: Use the power rule: \(\int x^2 dx = \frac{x^3}{3} + C\).
\end{example}

\begin{example}[Hard 3]
\textbf{Problem}: Evaluate \(\int x^3 \, dx\).
\textbf{Solution}: Use the power rule: \(\int x^3 dx = \frac{x^4}{4} + C\).
\end{example}

\begin{example}[Hard 4]
\textbf{Problem}: Evaluate \(\int x^4 \, dx\).
\textbf{Solution}: Use the power rule: \(\int x^4 dx = \frac{x^5}{5} + C\).
\end{example}

\begin{example}[Hard 5]
\textbf{Problem}: Evaluate \(\int x^5 \, dx\).
\textbf{Solution}: Use the power rule: \(\int x^5 dx = \frac{x^6}{6} + C\).
\end{example}

\begin{example}[Hard 6]
\textbf{Problem}: Evaluate \(\int x^6 \, dx\).
\textbf{Solution}: Use the power rule: \(\int x^6 dx = \frac{x^7}{7} + C\).
\end{example}

\begin{example}[Hard 7]
\textbf{Problem}: Evaluate \(\int x^7 \, dx\).
\textbf{Solution}: Use the power rule: \(\int x^7 dx = \frac{x^8}{8} + C\).
\end{example}

\begin{example}[Hard 8]
\textbf{Problem}: Evaluate \(\int x^8 \, dx\).
\textbf{Solution}: Use the power rule: \(\int x^8 dx = \frac{x^9}{9} + C\).
\end{example}

\begin{example}[Hard 9]
\textbf{Problem}: Evaluate \(\int x^9 \, dx\).
\textbf{Solution}: Use the power rule: \(\int x^9 dx = \frac{x^10}{10} + C\).
\end{example}

\begin{example}[Hard 10]
\textbf{Problem}: Evaluate \(\int x^10 \, dx\).
\textbf{Solution}: Use the power rule: \(\int x^10 dx = \frac{x^11}{11} + C\).
\end{example}

\begin{example}[Hard 11]
\textbf{Problem}: Evaluate \(\int x^11 \, dx\).
\textbf{Solution}: Use the power rule: \(\int x^11 dx = \frac{x^12}{12} + C\).
\end{example}

\begin{example}[Hard 12]
\textbf{Problem}: Evaluate \(\int x^12 \, dx\).
\textbf{Solution}: Use the power rule: \(\int x^12 dx = \frac{x^13}{13} + C\).
\end{example}

\begin{example}[Hard 13]
\textbf{Problem}: Evaluate \(\int x^13 \, dx\).
\textbf{Solution}: Use the power rule: \(\int x^13 dx = \frac{x^14}{14} + C\).
\end{example}

\begin{example}[Hard 14]
\textbf{Problem}: Evaluate \(\int x^14 \, dx\).
\textbf{Solution}: Use the power rule: \(\int x^14 dx = \frac{x^15}{15} + C\).
\end{example}

\begin{example}[Hard 15]
\textbf{Problem}: Evaluate \(\int x^15 \, dx\).
\textbf{Solution}: Use the power rule: \(\int x^15 dx = \frac{x^16}{16} + C\).
\end{example}

\begin{example}[Hard 16]
\textbf{Problem}: Evaluate \(\int x^16 \, dx\).
\textbf{Solution}: Use the power rule: \(\int x^16 dx = \frac{x^17}{17} + C\).
\end{example}

\begin{example}[Hard 17]
\textbf{Problem}: Evaluate \(\int x^17 \, dx\).
\textbf{Solution}: Use the power rule: \(\int x^17 dx = \frac{x^18}{18} + C\).
\end{example}

\begin{example}[Hard 18]
\textbf{Problem}: Evaluate \(\int x^18 \, dx\).
\textbf{Solution}: Use the power rule: \(\int x^18 dx = \frac{x^19}{19} + C\).
\end{example}

\begin{example}[Hard 19]
\textbf{Problem}: Evaluate \(\int x^19 \, dx\).
\textbf{Solution}: Use the power rule: \(\int x^19 dx = \frac{x^20}{20} + C\).
\end{example}

\begin{example}[Hard 20]
\textbf{Problem}: Evaluate \(\int x^20 \, dx\).
\textbf{Solution}: Use the power rule: \(\int x^20 dx = \frac{x^21}{21} + C\).
\end{example}

\begin{example}[Hard 21]
\textbf{Problem}: Evaluate \(\int x^21 \, dx\).
\textbf{Solution}: Use the power rule: \(\int x^21 dx = \frac{x^22}{22} + C\).
\end{example}

\begin{example}[Hard 22]
\textbf{Problem}: Evaluate \(\int x^22 \, dx\).
\textbf{Solution}: Use the power rule: \(\int x^22 dx = \frac{x^23}{23} + C\).
\end{example}

\begin{example}[Hard 23]
\textbf{Problem}: Evaluate \(\int x^23 \, dx\).
\textbf{Solution}: Use the power rule: \(\int x^23 dx = \frac{x^24}{24} + C\).
\end{example}

\begin{example}[Hard 24]
\textbf{Problem}: Evaluate \(\int x^24 \, dx\).
\textbf{Solution}: Use the power rule: \(\int x^24 dx = \frac{x^25}{25} + C\).
\end{example}

\begin{example}[Hard 25]
\textbf{Problem}: Evaluate \(\int x^25 \, dx\).
\textbf{Solution}: Use the power rule: \(\int x^25 dx = \frac{x^26}{26} + C\).
\end{example}

\subsection{Pro Level}
\begin{example}[Pro 1]
\textbf{Problem}: Show that the polynomial $x^3-2$ is irreducible over $\Q$.
\textbf{Solution}: Apply Eisenstein's criterion at $p=2$: all coefficients except the leading one are divisible by $2$, and the constant term $2$ is not divisible by $4$. Therefore $x^{n}-2$ is irreducible over $\Q$.
\end{example}

\begin{example}[Pro 2]
\textbf{Problem}: Show that the polynomial $x^4-2$ is irreducible over $\Q$.
\textbf{Solution}: Apply Eisenstein's criterion at $p=2$: all coefficients except the leading one are divisible by $2$, and the constant term $2$ is not divisible by $4$. Therefore $x^{n}-2$ is irreducible over $\Q$.
\end{example}

\begin{example}[Pro 3]
\textbf{Problem}: Show that the polynomial $x^5-2$ is irreducible over $\Q$.
\textbf{Solution}: Apply Eisenstein's criterion at $p=2$: all coefficients except the leading one are divisible by $2$, and the constant term $2$ is not divisible by $4$. Therefore $x^{n}-2$ is irreducible over $\Q$.
\end{example}

\begin{example}[Pro 4]
\textbf{Problem}: Show that the polynomial $x^6-2$ is irreducible over $\Q$.
\textbf{Solution}: Apply Eisenstein's criterion at $p=2$: all coefficients except the leading one are divisible by $2$, and the constant term $2$ is not divisible by $4$. Therefore $x^{n}-2$ is irreducible over $\Q$.
\end{example}

\begin{example}[Pro 5]
\textbf{Problem}: Show that the polynomial $x^7-2$ is irreducible over $\Q$.
\textbf{Solution}: Apply Eisenstein's criterion at $p=2$: all coefficients except the leading one are divisible by $2$, and the constant term $2$ is not divisible by $4$. Therefore $x^{n}-2$ is irreducible over $\Q$.
\end{example}

\begin{example}[Pro 6]
\textbf{Problem}: Show that the polynomial $x^8-2$ is irreducible over $\Q$.
\textbf{Solution}: Apply Eisenstein's criterion at $p=2$: all coefficients except the leading one are divisible by $2$, and the constant term $2$ is not divisible by $4$. Therefore $x^{n}-2$ is irreducible over $\Q$.
\end{example}

\begin{example}[Pro 7]
\textbf{Problem}: Show that the polynomial $x^9-2$ is irreducible over $\Q$.
\textbf{Solution}: Apply Eisenstein's criterion at $p=2$: all coefficients except the leading one are divisible by $2$, and the constant term $2$ is not divisible by $4$. Therefore $x^{n}-2$ is irreducible over $\Q$.
\end{example}

\begin{example}[Pro 8]
\textbf{Problem}: Show that the polynomial $x^10-2$ is irreducible over $\Q$.
\textbf{Solution}: Apply Eisenstein's criterion at $p=2$: all coefficients except the leading one are divisible by $2$, and the constant term $2$ is not divisible by $4$. Therefore $x^{n}-2$ is irreducible over $\Q$.
\end{example}

\begin{example}[Pro 9]
\textbf{Problem}: Show that the polynomial $x^11-2$ is irreducible over $\Q$.
\textbf{Solution}: Apply Eisenstein's criterion at $p=2$: all coefficients except the leading one are divisible by $2$, and the constant term $2$ is not divisible by $4$. Therefore $x^{n}-2$ is irreducible over $\Q$.
\end{example}

\begin{example}[Pro 10]
\textbf{Problem}: Show that the polynomial $x^12-2$ is irreducible over $\Q$.
\textbf{Solution}: Apply Eisenstein's criterion at $p=2$: all coefficients except the leading one are divisible by $2$, and the constant term $2$ is not divisible by $4$. Therefore $x^{n}-2$ is irreducible over $\Q$.
\end{example}

\begin{example}[Pro 11]
\textbf{Problem}: Show that the polynomial $x^13-2$ is irreducible over $\Q$.
\textbf{Solution}: Apply Eisenstein's criterion at $p=2$: all coefficients except the leading one are divisible by $2$, and the constant term $2$ is not divisible by $4$. Therefore $x^{n}-2$ is irreducible over $\Q$.
\end{example}

\begin{example}[Pro 12]
\textbf{Problem}: Show that the polynomial $x^14-2$ is irreducible over $\Q$.
\textbf{Solution}: Apply Eisenstein's criterion at $p=2$: all coefficients except the leading one are divisible by $2$, and the constant term $2$ is not divisible by $4$. Therefore $x^{n}-2$ is irreducible over $\Q$.
\end{example}

\begin{example}[Pro 13]
\textbf{Problem}: Show that the polynomial $x^15-2$ is irreducible over $\Q$.
\textbf{Solution}: Apply Eisenstein's criterion at $p=2$: all coefficients except the leading one are divisible by $2$, and the constant term $2$ is not divisible by $4$. Therefore $x^{n}-2$ is irreducible over $\Q$.
\end{example}

\begin{example}[Pro 14]
\textbf{Problem}: Show that the polynomial $x^16-2$ is irreducible over $\Q$.
\textbf{Solution}: Apply Eisenstein's criterion at $p=2$: all coefficients except the leading one are divisible by $2$, and the constant term $2$ is not divisible by $4$. Therefore $x^{n}-2$ is irreducible over $\Q$.
\end{example}

\begin{example}[Pro 15]
\textbf{Problem}: Show that the polynomial $x^17-2$ is irreducible over $\Q$.
\textbf{Solution}: Apply Eisenstein's criterion at $p=2$: all coefficients except the leading one are divisible by $2$, and the constant term $2$ is not divisible by $4$. Therefore $x^{n}-2$ is irreducible over $\Q$.
\end{example}

\begin{example}[Pro 16]
\textbf{Problem}: Show that the polynomial $x^18-2$ is irreducible over $\Q$.
\textbf{Solution}: Apply Eisenstein's criterion at $p=2$: all coefficients except the leading one are divisible by $2$, and the constant term $2$ is not divisible by $4$. Therefore $x^{n}-2$ is irreducible over $\Q$.
\end{example}

\begin{example}[Pro 17]
\textbf{Problem}: Show that the polynomial $x^19-2$ is irreducible over $\Q$.
\textbf{Solution}: Apply Eisenstein's criterion at $p=2$: all coefficients except the leading one are divisible by $2$, and the constant term $2$ is not divisible by $4$. Therefore $x^{n}-2$ is irreducible over $\Q$.
\end{example}

\begin{example}[Pro 18]
\textbf{Problem}: Show that the polynomial $x^20-2$ is irreducible over $\Q$.
\textbf{Solution}: Apply Eisenstein's criterion at $p=2$: all coefficients except the leading one are divisible by $2$, and the constant term $2$ is not divisible by $4$. Therefore $x^{n}-2$ is irreducible over $\Q$.
\end{example}

\begin{example}[Pro 19]
\textbf{Problem}: Show that the polynomial $x^21-2$ is irreducible over $\Q$.
\textbf{Solution}: Apply Eisenstein's criterion at $p=2$: all coefficients except the leading one are divisible by $2$, and the constant term $2$ is not divisible by $4$. Therefore $x^{n}-2$ is irreducible over $\Q$.
\end{example}

\begin{example}[Pro 20]
\textbf{Problem}: Show that the polynomial $x^22-2$ is irreducible over $\Q$.
\textbf{Solution}: Apply Eisenstein's criterion at $p=2$: all coefficients except the leading one are divisible by $2$, and the constant term $2$ is not divisible by $4$. Therefore $x^{n}-2$ is irreducible over $\Q$.
\end{example}

\begin{example}[Pro 21]
\textbf{Problem}: Show that the polynomial $x^23-2$ is irreducible over $\Q$.
\textbf{Solution}: Apply Eisenstein's criterion at $p=2$: all coefficients except the leading one are divisible by $2$, and the constant term $2$ is not divisible by $4$. Therefore $x^{n}-2$ is irreducible over $\Q$.
\end{example}

\begin{example}[Pro 22]
\textbf{Problem}: Show that the polynomial $x^24-2$ is irreducible over $\Q$.
\textbf{Solution}: Apply Eisenstein's criterion at $p=2$: all coefficients except the leading one are divisible by $2$, and the constant term $2$ is not divisible by $4$. Therefore $x^{n}-2$ is irreducible over $\Q$.
\end{example}

\begin{example}[Pro 23]
\textbf{Problem}: Show that the polynomial $x^25-2$ is irreducible over $\Q$.
\textbf{Solution}: Apply Eisenstein's criterion at $p=2$: all coefficients except the leading one are divisible by $2$, and the constant term $2$ is not divisible by $4$. Therefore $x^{n}-2$ is irreducible over $\Q$.
\end{example}

\begin{example}[Pro 24]
\textbf{Problem}: Show that the polynomial $x^26-2$ is irreducible over $\Q$.
\textbf{Solution}: Apply Eisenstein's criterion at $p=2$: all coefficients except the leading one are divisible by $2$, and the constant term $2$ is not divisible by $4$. Therefore $x^{n}-2$ is irreducible over $\Q$.
\end{example}

\begin{example}[Pro 25]
\textbf{Problem}: Show that the polynomial $x^27-2$ is irreducible over $\Q$.
\textbf{Solution}: Apply Eisenstein's criterion at $p=2$: all coefficients except the leading one are divisible by $2$, and the constant term $2$ is not divisible by $4$. Therefore $x^{n}-2$ is irreducible over $\Q$.
\end{example}
