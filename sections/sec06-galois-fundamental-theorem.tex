\section{Galois extensions and the Fundamental Theorem}\label{sec:galois-ft}

\subsection{Definitions}
\begin{definition}
A finite extension $E/F$ is \emph{Galois} if it is normal and separable. Its Galois group is $\Gal(E/F)=\Aut_F(E)$.
\end{definition}

\subsection{Artin's theorem and counting}
\begin{theorem}[Artin]
If $G$ is a finite group of $F$-automorphisms of $E$, then $E/E^G$ is Galois and $\Gal(E/E^G)\cong G$.
\end{theorem}
\begin{proof}
Let $K=E^G$. For $\alpha\in E$ set
\(f_\alpha(x)=\prod_{\sigma\in G}(x-\sigma(\alpha))\in K[x]\). The roots of
$f_\alpha$ lie in $E$ and are distinct, so $E/K$ is normal and separable, hence
Galois. The inclusion $G\subseteq\Gal(E/K)$ is clear. Conversely, if
$\tau\in\Gal(E/K)$, then $\tau(\alpha)$ is a root of $f_\alpha$ for every
$\alpha$, so $\tau$ agrees with some $\sigma\in G$ on generators of $E$, forcing
$\tau=\sigma$. Thus $\Gal(E/K)=G$.
\end{proof}
\begin{theorem}[Counting]
For finite $E/F$, $\#\Aut_F(E)\le [E\!:\!F]$, with equality iff $E/F$ is Galois.
\end{theorem}
\begin{proof}
Let $G=\Aut_F(E)$ and $K=E^G$. By Artin's theorem, $E/K$ is Galois and
$[E\!:\!K]=\#G$. Since $F\subseteq K$, we have $[E\!:\!F]\ge [E\!:\!K]=\#G$.
Equality holds precisely when $K=F$, i.e. when $E/F$ is Galois.
\end{proof}

\subsection{Fundamental Theorem of Galois Theory}
\begin{theorem}\label{thm:FTGT}
Let $E/F$ be finite Galois with group $G=\Gal(E/F)$. The maps
\[
K\mapsto \Gal(E/K),\qquad H\mapsto E^H
\]
yield inverse inclusion-reversing bijections between intermediate fields $F\subseteq K\subseteq E$ and subgroups $H\le G$. Moreover
\[
[E\!:\!K]=\#H,\qquad [K\!:\!F]=\#G/\#H,
\]
and $K/F$ is normal $\iff$ $H\trianglelefteq G$, in which case $\Gal(K/F)\cong G/H$.
\end{theorem}
\begin{proof}
For $K$ with $F\subseteq K\subseteq E$, Artin's theorem gives $E^{\Gal(E/K)}=K$.
For $H\le G$, the same theorem yields $\Gal(E/E^H)=H$. These equalities show the
maps are inverse and inclusion reversing. Furthermore,
$[E\!:\!K]=\#\Gal(E/K)=\#H$ and $[K\!:\!F]=\#G/\#H$.

If $H\trianglelefteq G$ then conjugation by $G$ preserves $H$, so automorphisms
of $E$ descend to $K=E^H$, giving $\Gal(K/F)\cong G/H$. Conversely, if $K/F$ is
normal, any $\sigma\in G$ sends $K$ to itself, hence $\sigma H\sigma^{-1}=H$ and
$H\trianglelefteq G$.
\end{proof}

\subsection{Examples}
\begin{example}[Quadratic]
For squarefree $d$, $\Q(\sqrt d)/\Q$ is Galois with group $C_2$.
\end{example}
\begin{example}[Cyclotomic prime]
$\Q(\zeta_p)/\Q$ has Galois group $(\Z/p\Z)^\times\cong C_{p-1}$ for prime $p$; see \cref{sec:cyclotomic-kummer}.
\end{example}
