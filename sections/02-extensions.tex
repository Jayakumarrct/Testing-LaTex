\section{Extensions and Minimal Polynomials}

\subsection{Field extensions and degree}
\begin{definition}
A \emph{field extension} is an inclusion $K\subseteq L$ of fields. The \emph{degree} is $[L:K]=\dim_K L$ when finite.
\end{definition}

\begin{proposition}[Tower law]
If $K\subseteq L\subseteq M$, then $[M:K]=[M:L]\,[L:K]$ whenever the numbers are finite.
\end{proposition}
\begin{proof}
Choose a $K$-basis of $L$ and an $L$-basis of $M$; the products form a $K$-basis of $M$.
\end{proof}

\subsection{Algebraic and transcendental elements}
\begin{definition}
$\alpha\in L$ is \emph{algebraic over $K$} if there exists nonzero $f\in K[x]$ with $f(\alpha)=0$. Otherwise $\alpha$ is \emph{transcendental}.  
An extension $L/K$ is \emph{algebraic} if every element of $L$ is algebraic over $K$.
\end{definition}

\begin{definition}[Minimal polynomial]
If $\alpha$ is algebraic over $K$, the \emph{minimal polynomial} $m_\alpha(x)\in K[x]$ is the unique monic irreducible polynomial with $m_\alpha(\alpha)=0$.
\end{definition}

\begin{theorem}[Simple extensions]\label{thm:simple}
If $\alpha$ is algebraic over $K$, then $K(\alpha)\cong K[x]/(m_\alpha)$ via $x\mapsto \alpha$, and $[K(\alpha):K]=\deg m_\alpha$.
\end{theorem}
\begin{proof}
Define $\varphi:K[x]\to K(\alpha)$ by evaluation at $\alpha$. Then $\ker\varphi=(m_\alpha)$ and the image is $K(\alpha)$. Apply the First Isomorphism Theorem; dimension equals $\deg m_\alpha$.
\end{proof}

\begin{proposition}
If $f\in K[x]$ is irreducible, then $K[x]/(f)$ is a field, and adjoining any root $\alpha$ of $f$ gives $K(\alpha)\cong K[x]/(f)$.
\end{proposition}

\subsection{Worked examples}
\begin{example}
$\alpha=\sqrt{2}$ over $\Q$: $m_\alpha=x^2-2$, so $[\Q(\alpha):\Q]=2$ and $\{1,\alpha\}$ is a basis.
\end{example}
\begin{example}
$\beta=\sqrt[3]{2}$ over $\Q$: $m_\beta=x^3-2$, so $[\Q(\beta):\Q]=3$. The tower $\Q\subset \Q(\beta)\subset \Q(\beta,\omega)$, where $\omega$ is a primitive cube root of unity, will reappear as a splitting field.
\end{example}
\begin{example}[Adjoining multiple elements]
If $\alpha,\beta$ are algebraic over $K$, then $K(\alpha,\beta)$ is finite over $K$ with
\[
[K(\alpha,\beta):K]\le [K(\alpha):K]\,[K(\beta):K(\alpha)]\le (\deg m_\alpha)(\deg m_\beta).
\]
\end{example}

\subsection{Transcendence and rational functions}
\begin{example}
If $t$ is transcendental over $K$, then $K(t)$ has a $K$-basis $\{1,t,t^2,\dots\}$ as a vector space is false; indeed $K(t)$ is infinite-dimensional over $K$ but generated as a field by $t$ and $K$.\footnote{Beware: $K(t)$ is not a finite extension of $K$.}
\end{example}

\subsection{Remarks}
\begin{remark}
Every algebraic extension is a directed union of finite simple extensions by repeatedly adjoining elements.
\end{remark}
