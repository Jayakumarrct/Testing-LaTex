\section{Field extensions and minimal polynomials}\label{sec:extensions-minimal}

\subsection{Extensions and degree}
\begin{definition}
An \emph{extension} $E/F$ is a field $E$ containing $F$ as a subfield; its degree is $[E\!:\!F]=\dim_F E$.
\end{definition}
\begin{theorem}[Tower law]\label{thm:tower}
If $K/E/F$, then $[K\!:\!F]=[K\!:\!E]\,[E\!:\!F]$.
\end{theorem}
\begin{proof}
Choose an $F$-basis $\{e_i\}$ of $E$ and an $E$-basis $\{k_j\}$ of $K$. Then $\{e_i k_j\}$ is an $F$-basis of $K$.
\end{proof}
References: \cite[\S13]{DF}, \cite[Ch.~V]{Artin}.

\subsection{Algebraic vs.\ transcendental}
\begin{definition}
$\alpha\in E$ is \emph{algebraic over $F$} if it satisfies some nonzero $f\in F[x]$; otherwise \emph{transcendental}.
\end{definition}

\subsection{Minimal polynomial and simple extensions}
\begin{definition}
For algebraic $\alpha$, the \emph{minimal polynomial} $m_{\alpha,F}\in F[x]$ is the unique monic irreducible with $m_{\alpha,F}(\alpha)=0$.
\end{definition}
\begin{proposition}[Structure]\label{prop:simple}
If $n=\deg m_{\alpha,F}$ then
\[
F(\alpha)\cong F[x]/(m_{\alpha,F}),\qquad
\{1,\alpha,\dots,\alpha^{n-1}\}\ \text{is an $F$-basis, so } [F(\alpha)\!:\!F]=n.
\]
\end{proposition}
\begin{proof}
Evaluate polynomials at $\alpha$ to obtain a surjective homomorphism $\varphi:F[x]\to F(\alpha)$. Its kernel consists of polynomials vanishing at $\alpha$; since $m_{\alpha,F}$ is the monic irreducible with this property, $\ker\varphi=(m_{\alpha,F})$.
Thus $F(\alpha)\cong F[x]/(m_{\alpha,F})$.
Every element of $F(\alpha)$ is $r(\alpha)$ for some $r\in F[x]$; dividing $r$ by $m_{\alpha,F}$ shows it is an $F$-linear combination of $1,\alpha,\dots,\alpha^{n-1}$.
If $\sum_{i=0}^{n-1}c_i\alpha^i=0$, the polynomial $\sum_{i=0}^{n-1}c_ix^i$ lies in $\ker\varphi$ and hence is $0$, so the coefficients vanish.
Therefore $\{1,\alpha,\dots,\alpha^{n-1}\}$ is a basis and $[F(\alpha)\!:\!F]=n$.
\end{proof}
\begin{remark}[Conjugates]
Each $F$-embedding $\sigma:F(\alpha)\hookrightarrow \overline{F}$ sends $\alpha$ to a root of $m_{\alpha,F}$; there are at most $n$ such embeddings.
\end{remark}
References: \cite[\S13--14]{DF}, \cite[Ch.~V]{Artin}.

\subsection{Examples}
\begin{example}
For squarefree $d\in\Z$, $m_{\sqrt d,\Q}=x^2-d$ so $[\Q(\sqrt d):\Q]=2$.
\end{example}
\begin{example}
With $\alpha=\sqrt2+\sqrt3$, $m_{\alpha,\Q}(x)=x^4-10x^2+1$ and $[\Q(\alpha):\Q]=4$.
\end{example}
