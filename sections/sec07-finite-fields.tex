\section{Finite fields}\label{sec:finite-fields}

\subsection{Existence and uniqueness}
\begin{theorem}
For each prime power $q=p^n$ there exists a field $\mathbb{F}_q$ of order $q$, unique up to isomorphism.
\end{theorem}
\begin{proof}
Choose an irreducible polynomial $f\in\mathbb{F}_p[x]$ of degree $n$.
The quotient $\mathbb{F}_p[x]/(f)$ is a field with $p^n$ elements.
If $K$ is any field with $q$ elements, then every $x\in K$ satisfies
$x^q=x$, so $K$ is a splitting field of $x^q-x$ over $\mathbb{F}_p$.
Any two splitting fields of this separable polynomial are isomorphic,
so $\mathbb{F}_q$ is unique.
\end{proof}
\begin{theorem}
$\mathbb{F}_q^\times$ is cyclic of order $q-1$.
\end{theorem}
\begin{proof}
Let $m$ be the maximal order of an element in $\mathbb{F}_q^\times$ and
choose $g$ with order $m$.
For each divisor $d$ of $m$, the equation $x^d=1$ has at most $d$ roots, so
$|\langle g\rangle|=m$.
If $m<q-1$, then every element has order dividing $m$, and the total
number of such elements is at most $m$, contradicting
$|\mathbb{F}_q^\times|=q-1$.
Thus $m=q-1$ and $\mathbb{F}_q^\times$ is cyclic of order $q-1$.
\end{proof}

\subsection{Frobenius and subfields}
\begin{proposition}[Frobenius]
The map $\varphi:x\mapsto x^p$ is an automorphism of $\mathbb{F}_{p^n}$ with fixed field $\mathbb{F}_p$.
\end{proposition}
\begin{proof}
The map $\varphi$ is a field homomorphism because $\operatorname{char}\mathbb{F}_{p^n}=p$.
In a finite field injective maps are surjective, hence $\varphi$ is an
automorphism.
If $\varphi(\alpha)=\alpha$, then $\alpha^p=\alpha$, so $\alpha\in\mathbb{F}_p$.
Conversely every element of $\mathbb{F}_p$ is fixed, so the fixed field is
$\mathbb{F}_p$.
\end{proof}
\begin{theorem}
$\Gal(\mathbb{F}_{p^n}/\mathbb{F}_p)=\langle \varphi\rangle\cong C_n$. Subfields are exactly $\mathbb{F}_{p^d}$ with $d\mid n$.
\end{theorem}
\begin{proof}
Since $\varphi^n=\operatorname{id}$ and $\varphi^d\ne\operatorname{id}$ for $0<d<n$, the
automorphism $\varphi$ has order $n$.
The extension $\mathbb{F}_{p^n}/\mathbb{F}_p$ has degree $n$, so the
Galois group has at most $n$ elements.
Thus $\Gal(\mathbb{F}_{p^n}/\mathbb{F}_p)=\langle\varphi\rangle\cong C_n$.
The fixed field of $\varphi^d$ is $\mathbb{F}_{p^d}$, and every
intermediate field is fixed by some subgroup of $\langle\varphi\rangle$,
so the subfields are precisely $\mathbb{F}_{p^d}$ with $d\mid n$.
\end{proof}
References: \cite[\S13]{DF}, \cite[Ch.~VIII]{Lang}.

\subsection{Irreducibles, trace, and norm}
\begin{proposition}
For $\alpha\in\mathbb{F}_{p^n}$ with Frobenius orbit of size $r$,
\[
m_\alpha(x)=\prod_{i=0}^{r-1}\bigl(x-\alpha^{p^i}\bigr).
\]
\end{proposition}
\begin{proof}
The conjugates of $\alpha$ under $\Gal(\mathbb{F}_{p^n}/\mathbb{F}_p)$ are
$\alpha,\alpha^p,\ldots,\alpha^{p^{r-1}}$.
Their product is a monic polynomial in $\mathbb{F}_p[x]$ that has $\alpha$
as a root, so it is divisible by the minimal polynomial $m_\alpha(x)$.
Because distinct conjugates give distinct factors, the degree is $r$,
which equals $\deg m_\alpha$; hence the displayed polynomial is
$m_\alpha(x)$.
\end{proof}
\begin{definition}
For $E=\mathbb{F}_{p^n}$ over $F=\mathbb{F}_p$ define
\[
\Tr_{E/F}(\alpha)=\alpha+\alpha^p+\cdots+\alpha^{p^{n-1}},\qquad
\Norm_{E/F}(\alpha)=\alpha^{1+p+\cdots+p^{n-1}}.
\]
\end{definition}

\subsection{Examples}
\begin{example}
$\mathbb{F}_4=\mathbb{F}_2[\omega]/(\omega^2+\omega+1)$ with $\omega^3=1$, and $\mathbb{F}_4^\times=\langle\omega\rangle$ is cyclic of order $3$.
\end{example}
