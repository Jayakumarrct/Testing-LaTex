\section{Fields and polynomials}\label{sec:fields-polynomials}

\subsection{Fields, characteristic, prime fields}
\begin{definition}
A \emph{field} is a commutative ring $F\neq 0$ in which every nonzero element has a multiplicative inverse.
\end{definition}
\begin{proposition}[Characteristic]\label{prop:char}
The characteristic of a field is either $0$ or a prime $p$; in the latter case $F$ contains a copy of $\mathbb{F}_p$.
\end{proposition}
\begin{proof}
If $\operatorname{char}F=n>0$ and $n=ab$ with $1<a,b<n$, then $(a\cdot1_F)(b\cdot1_F)=0$ gives a zero divisor. Fields have none, so $n$ is prime.
\end{proof}
\begin{proposition}[Prime field]
Every field contains a smallest subfield (the \emph{prime field}), isomorphic to $\Q$ when $\operatorname{char}F=0$ and to $\mathbb{F}_p$ when $\operatorname{char}F=p$.
\end{proposition}
\begin{proof}
Let $S$ be the subring generated by $1_F$. If $\operatorname{char}F=p>0$, then the map $\mathbb{F}_p\to F$ sending $\overline{n}$ to $n\cdot1_F$ is injective with image $S$, so $S\cong\mathbb{F}_p$.
If $\operatorname{char}F=0$, the map $\Q\to F$ given by $a/b\mapsto (a\cdot1_F)(b\cdot1_F)^{-1}$ embeds $\Q$ into $F$ and its image is $S$.
Every subfield of $F$ contains $S$ because it contains $1_F$, so $S$ is the smallest subfield.
\end{proof}
References: \cite[\S13]{DF}, \cite[Ch.~I]{Artin}.

\subsection{Polynomials over a field}
\begin{proposition}[Division algorithm; Euclidean domain]\label{prop:division}
If $f,g\in F[x]$ with $g\neq0$, there exist unique $q,r\in F[x]$ such that $f=qg+r$ and $\deg r<\deg g$. Hence $F[x]$ is a Euclidean domain with measure $\deg$ and thus a PID and a UFD.
\end{proposition}
\begin{proof}
If $\deg f<\deg g$ take $q=0$ and $r=f$. Otherwise let $n=\deg f$, $m=\deg g$, and write $a_n$ and $b_m$ for the leading coefficients.
Then $f - (a_n/b_m)x^{n-m}g$ has degree $<n$.
Repeatedly subtract such multiples until the remainder has degree $<m$; this gives $f=qg+r$.
Uniqueness follows because if $f=qg+r=q'g+r'$ with $\deg r,\deg r'<m$, then $(q-q')g=r'-r$ and the degree constraint forces $q=q'$ and $r=r'$.
Thus $F[x]$ is Euclidean with norm $\deg$, hence a PID and a UFD.
\end{proof}
\begin{corollary}[Bezout identity]
There exist $a,b\in F[x]$ with $a f + b g = \gcd(f,g)$.
\end{corollary}
\begin{proof}
The Euclidean algorithm in $F[x]$ produces $\gcd(f,g)$ as a linear combination of $f$ and $g$, yielding such $a$ and $b$.
\end{proof}

\subsection{Irreducibility criteria}
\begin{proposition}[Degree $2$ or $3$]
If $\deg f\in\{2,3\}$ then $f$ is irreducible over $F$ iff it has no root in $F$.
\end{proposition}
\begin{proof}
If $f$ has a root $\alpha\in F$, then $f=(x-\alpha)h$ with $\deg h=\deg f-1$, so $f$ is reducible.
Conversely, if $f$ factors as $gh$ with $\deg f\in\{2,3\}$, one factor must be linear; its root is a root of $f$.
\end{proof}
\begin{theorem}[Eisenstein]\label{thm:eisenstein}
Let $R$ be a UFD with prime $p$. If $f(x)=a_nx^n+\cdots+a_0\in R[x]$ satisfies $p\nmid a_n$, $p\mid a_i$ for $i<n$, and $p^2\nmid a_0$, then $f$ is irreducible in $R[x]$ and in $\operatorname{Frac}(R)[x]$.
\end{theorem}
\begin{proof}
Suppose $f=gh$ with $g,h\in R[x]$ of positive degree and write $g(0)=b_0$, $h(0)=c_0$.
Since $a_0=b_0c_0$ and $p^2\nmid a_0$, exactly one of $b_0,c_0$ is divisible by $p$; assume $p\mid b_0$ and $p\nmid c_0$.
Reduce mod $p$: in $(R/p)[x]$ we have $\overline{f}(x)=\overline{a_n}x^n$, so $\overline{g}(x)=\overline{b_s}x^s$ and $\overline{h}(x)=\overline{c_t}x^t$ with $s+t=n$.
Thus every coefficient of $g$ except the leading one is divisible by $p$.
Looking at the coefficient of $x^k$ in $f$ for $k<s$ and using $p\nmid c_0$, induction shows that $p$ divides $b_k$ for all $k\le s$.
In particular $p\mid b_s$, contradicting $p\nmid a_n=b_sc_t$.
Hence $f$ is irreducible in $R[x]$ and therefore in $\operatorname{Frac}(R)[x]$.
\end{proof}
\begin{example}
$x^5-10x+5$ is irreducible over $\Q$ after the shift $x\mapsto x+1$ by \cref{thm:eisenstein} with $p=5$.
\end{example}
References: \cite[\S9--11]{DF}, \cite[Ch.~II]{Lang}.

\subsection{Evaluation, roots, and factorization}
\begin{definition}
If $E/F$ is an extension and $\alpha\in E$, then $\alpha$ is a \emph{root} of $f\in F[x]$ if $f(\alpha)=0$ (evaluate in $E$).
\end{definition}
\begin{remark}
Factorization in $F[x]$ vs.\ in an extension differs by adjoining roots; this motivates splitting fields, treated in \cref{sec:splitting-fields}.
\end{remark}
