\section{Fields and polynomials}\label{sec:fields-polynomials}

\subsection{Fields, characteristic, prime fields}
\begin{definition}
A \emph{field} is a commutative ring $F\neq 0$ in which every nonzero element has a multiplicative inverse.
\end{definition}
\begin{proposition}[Characteristic]\label{prop:char}
The characteristic of a field is either $0$ or a prime $p$; in the latter case $F$ contains a copy of $\mathbb{F}_p$.
\end{proposition}
\begin{proof}
If $\operatorname{char}F=n>0$ and $n=ab$ with $1<a,b<n$, then $(a\cdot1_F)(b\cdot1_F)=0$ gives a zero divisor. Fields have none, so $n$ is prime.
\end{proof}
\begin{proposition}[Prime field]
Every field contains a smallest subfield (the \emph{prime field}), isomorphic to $\Q$ when $\operatorname{char}F=0$ and to $\mathbb{F}_p$ when $\operatorname{char}F=p$.
\end{proposition}
\begin{proof}
Let $S$ be the set of all subfields of $F$ and let $K=\bigcap_{E\in S}E$. Then $K$ is a subfield contained in every subfield of $F$, so it is the smallest one. It is generated by $1_F$. If $\operatorname{char}F=p>0$, then $1_F+\cdots+1_F=0$ ($p$ times) gives a copy of $\mathbb{F}_p$. If $\operatorname{char}F=0$, the map $\Z\to F$, $n\mapsto n\cdot1_F$, is injective and its image generates a copy of $\Q$. This subfield $K$ is the prime field.
\end{proof}
References: \cite[\S13]{DF}, \cite[Ch.~I]{Artin}.

\subsection{Polynomials over a field}
\begin{proposition}[Division algorithm; Euclidean domain]\label{prop:division}
If $f,g\in F[x]$ with $g\neq0$, there exist unique $q,r\in F[x]$ such that $f=qg+r$ and $\deg r<\deg g$. Hence $F[x]$ is a Euclidean domain with measure $\deg$ and thus a PID and a UFD.
\end{proposition}
\begin{proof}
If $\deg f<\deg g$ take $q=0$ and $r=f$. Otherwise, write $f=a x^m+\cdots$ and $g=b x^n+\cdots$ with $m\ge n$. Set $f_1=f-\frac{a}{b}x^{m-n}g$; then $\deg f_1<m$. Repeating yields $f=qg+r$ with $\deg r<\deg g$. If $f=qg+r=q'g+r'$ with $\deg r,\deg r'<\deg g$, then $g(q-q')=r'-r$. The right side has degree $<\deg g$, forcing $q=q'$ and $r=r'$. The Euclidean algorithm now shows $F[x]$ is a Euclidean domain, hence a PID and a UFD.
\end{proof}
\begin{corollary}[Bezout identity]
There exist $a,b\in F[x]$ with $a f + b g = \gcd(f,g)$.
\end{corollary}
\begin{proof}
Apply the Euclidean algorithm to $f$ and $g$. Writing each remainder as a combination of the previous two gives $\gcd(f,g)$ as a combination $af+bg$ for some $a,b\in F[x]$.
\end{proof}

\subsection{Irreducibility criteria}
\begin{proposition}[Degree $2$ or $3$]
If $\deg f\in\{2,3\}$ then $f$ is irreducible over $F$ iff it has no root in $F$.
\end{proposition}
\begin{proof}
If $f$ has a root $\alpha\in F$, then $f=(x-\alpha)g$ with $\deg g=\deg f-1\ge1$, so $f$ is reducible. Conversely, if $f=gh$ with positive degrees, then $\deg g+\deg h=\deg f\in\{2,3\}$. Hence one factor has degree $1$, giving a root in $F$.
\end{proof}
\begin{theorem}[Eisenstein]\label{thm:eisenstein}
Let $R$ be a UFD with prime $p$. If $f(x)=a_nx^n+\cdots+a_0\in R[x]$ satisfies $p\nmid a_n$, $p\mid a_i$ for $i<n$, and $p^2\nmid a_0$, then $f$ is irreducible in $R[x]$ and in $\operatorname{Frac}(R)[x]$.
\end{theorem}
\begin{proof}
Assume $f=gh$ with $g,h\in R[x]$ of positive degree. Let $g=b_sx^s+\cdots+b_0$ and $h=c_tx^t+\cdots+c_0$ with $b_s c_t=a_n$. Since $p\nmid a_n$, neither $b_s$ nor $c_t$ is divisible by $p$. The coefficients of lower degree in $f$ are divisible by $p$, so $b_0c_t+b_sc_0\equiv0\pmod p$. Because $p\nmid b_s c_t$, both $b_0$ and $c_0$ must be divisible by $p$. Then $p^2\mid b_0c_0=a_0$, contradicting $p^2\nmid a_0$. Thus $f$ is irreducible in $R[x]$. Gauss's lemma now gives irreducibility in $\operatorname{Frac}(R)[x]$.
\end{proof}
\begin{example}
$x^5-10x+5$ is irreducible over $\Q$ after the shift $x\mapsto x+1$ by \cref{thm:eisenstein} with $p=5$.
\end{example}
References: \cite[\S9--11]{DF}, \cite[Ch.~II]{Lang}.

\subsection{Evaluation, roots, and factorization}
\begin{definition}
If $E/F$ is an extension and $\alpha\in E$, then $\alpha$ is a \emph{root} of $f\in F[x]$ if $f(\alpha)=0$ (evaluate in $E$).
\end{definition}
\begin{remark}
Factorization in $F[x]$ vs.\ in an extension differs by adjoining roots; this motivates splitting fields, treated in \cref{sec:splitting-fields}.
\end{remark}
