\section{Separable and normal extensions}\label{sec:sep-normal}

\subsection{Formal derivative and separability}
\begin{definition}
For $f(x)=\sum a_i x^i\in F[x]$, the \emph{formal derivative} is $f'(x)=\sum i\,a_i x^{i-1}$.
\end{definition}
\begin{proposition}[Multiplicity]
In a splitting field, a root $\alpha$ has multiplicity $>1$ iff $f(\alpha)=f'(\alpha)=0$.
\end{proposition}
\begin{proof}
Write $f=(x-\alpha)^m g$ in the splitting field with $g(\alpha)\ne0$. Then
\(f' = m(x-\alpha)^{m-1} g + (x-\alpha)^m g'\). If $m>1$ both $f(\alpha)$ and
$f'(\alpha)$ vanish. Conversely, if $f(\alpha)=f'(\alpha)=0$ and $f=(x-\alpha)g$
with $g(\alpha)\ne0$, then $f'(\alpha)=g(\alpha)\ne0$, a contradiction; hence
$m>1$.
\end{proof}
\begin{theorem}[Separability criteria]\label{thm:sep-criteria}
For irreducible $f\in F[x]$ the following are equivalent:
\begin{enumerate}[label=(\alph*)]
\item $f$ has no multiple roots in a splitting field;
\item $\gcd(f,f')=1$ in $F[x]$;
\item the $F$-algebra $F[x]/(f)$ has exactly $\deg f$ embeddings into $\overline{F}$.
\end{enumerate}
A root of such $f$ is \emph{separable}. An algebraic extension is \emph{separable} if each of its elements is separable.
\end{theorem}
\begin{proof}
(a)$\Leftrightarrow$(b): a multiple root $\alpha$ of $f$ satisfies
$f(\alpha)=f'(\alpha)=0$, so $\gcd(f,f')$ has positive degree. Conversely, if
$\gcd(f,f')\ne1$, a common root gives a multiple root.

(a)$\Leftrightarrow$(c): In a splitting field the embeddings of $F[x]/(f)$ send
$x$ to the distinct roots of $f$. Thus the number of embeddings equals the number
of distinct roots, which is $\deg f$ precisely when $f$ has no multiple roots.
\end{proof}
\begin{corollary}[Perfect fields]
If $\operatorname{char}F=0$ or $F$ is finite then every algebraic extension of $F$ is separable.
\end{corollary}
\begin{proof}
Let $f\in F[x]$ be irreducible. If $\operatorname{char}F=0$ then $f'\ne0$.
When $F$ is finite of characteristic $p$, the Frobenius map $x\mapsto x^p$ is
bijective, so an irreducible $f$ cannot be of the form $g(x^p)$ and hence
$f'\ne0$. By \cref{thm:sep-criteria}, $f$ is separable, and thus every algebraic
extension of $F$ is separable.
\end{proof}
References: \cite[\S13--14]{DF}, \cite[Ch.~VIII]{Lang}.

\subsection{Normal extensions}
\begin{definition}
An algebraic extension $E/F$ is \emph{normal} if every irreducible $f\in F[x]$ having a root in $E$ splits completely over $E$.
\end{definition}
\begin{theorem}[Equivalent characterizations]\label{thm:normal}
For algebraic $E/F$ the following are equivalent:
\begin{enumerate}[label=(\alph*)]
\item $E/F$ is normal;
\item $E$ is a splitting field of a family of polynomials in $F[x]$;
\item every $F$-embedding $\sigma:E\hookrightarrow\overline{F}$ satisfies $\sigma(E)=E$.
\end{enumerate}
\end{theorem}
\begin{proof}
(a)$\Rightarrow$(b): If $E/F$ is normal, take the family of irreducible
polynomials in $F[x]$ having a root in $E$; $E$ is a splitting field for this
family.

(b)$\Rightarrow$(c): If $E$ is a splitting field of polynomials in $F[x]$, any
$F$-embedding $\sigma$ permutes their roots, so $\sigma(E)=E$.

(c)$\Rightarrow$(a): Let $f\in F[x]$ be irreducible with a root $\alpha$ in $E$.
For any other root $\beta$, there is an $F$-embedding sending $\alpha$ to
$\beta$, hence $\beta\in E$. Thus $f$ splits over $E$, so $E/F$ is normal.
\end{proof}

\subsection{Primitive element theorem}
\begin{theorem}[Primitive element]\label{thm:PET}
If $E/F$ is finite and separable then $E=F(\alpha)$ for some $\alpha\in E$.
\end{theorem}
\begin{proof}
Take $E=F(\alpha,\beta)$. For distinct embeddings $\sigma\ne\tau$ of $E$ fixing
$F$, the set of $c\in F$ with $\sigma(\alpha+c\beta)=\tau(\alpha+c\beta)$ is
finite. Choose $c$ avoiding all these finitely many values. Then the conjugates
of $\gamma=\alpha+c\beta$ are distinct, so the minimal polynomial of $\gamma$ has
degree $[E\!:\!F]$, giving $E=F(\gamma)$.
\end{proof}
\begin{example}
$\Q(\sqrt2,\sqrt3)=\Q(\sqrt2+\sqrt3)$ since the extension is finite and separable.
\end{example}
References: \cite[\S14]{DF}, \cite[Ch.~VI]{Artin}.
