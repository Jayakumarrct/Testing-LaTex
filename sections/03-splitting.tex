\section{Splitting Fields and Algebraic Closures}

\subsection{Splitting fields}
\begin{definition}
Let $f\in K[x]$ be nonconstant. A field $E\supseteq K$ is a \emph{splitting field} of $f$ over $K$ if (i) $f$ factors in $E[x]$ as a product of linear terms and (ii) $E$ is generated over $K$ by the roots of $f$.
\end{definition}

\begin{theorem}[Existence]\label{thm:splitting-existence}
Every $f\in K[x]$ has a splitting field over $K$, and any two splitting fields are $K$-isomorphic.
\end{theorem}
\begin{proof}[Proof sketch]
Adjoin one root $\alpha_1$ in some algebraic closure to get $K(\alpha_1)$. Factor $f=(x-\alpha_1)g$ and adjoin a root of $g$; repeat finitely many times.  
For uniqueness, use induction on $\deg f$ and extend $K$-embeddings stepwise.
\end{proof}

\begin{proposition}[Embedding bound]
Let $E/K$ be generated by $n$ distinct roots of $f\in K[x]$ with $\deg f=n$. Then the number of $K$-embeddings $E\hookrightarrow \overline{K}$ is at most $n!$. If $f$ has distinct roots, then $[E:K]\le n!$.
\end{proposition}
\begin{proof}[Idea]
An embedding permutes the roots. The image of $E$ is determined by the image of a generating tuple of roots, giving a subgroup of $S_n$.
\end{proof}

\subsection{Algebraic closures}
\begin{definition}
An \emph{algebraic closure} $\overline{K}$ of $K$ is an algebraic extension in which every nonconstant polynomial in $K[x]$ splits completely.
\end{definition}
\begin{theorem}[Existence and uniqueness]
Every field has an algebraic closure, unique up to $K$-isomorphism.
\end{theorem}

\subsection{Examples}
\begin{example}[Cubic $x^3-2$]
Over $\Q$, $f=x^3-2$ has one real root $\beta=\sqrt[3]{2}$ and two complex roots $\omega\beta,\omega^2\beta$ where $\omega=e^{2\pi i/3}$.  
A splitting field is $E=\Q(\beta,\omega)$. One checks $[\Q(\omega):\Q]=2$, $[\Q(\beta):\Q]=3$, and $[\Q(\beta,\omega):\Q]=6$ by the tower law, so $E/\Q$ is degree $6$.
\end{example}

\begin{example}[Cyclotomic quartic]
$f=x^4+1$ over $\Q$ splits in $E=\Q(\zeta_8)$ where $\zeta_8=e^{2\pi i/8}$. Here $E/\Q$ has degree $\varphi(8)=4$ and the roots are $\zeta_8^{\pm1},\zeta_8^{\pm3}$.
\end{example}

\subsection{A small lattice picture}
\[
\begin{tikzcd}
& \Q(\beta,\omega) \arrow[-]{dl} \arrow[-]{dr} & \\
\Q(\beta) \arrow[-]{dr} & & \Q(\omega) \arrow[-]{dl} \\
& \Q &
\end{tikzcd}
\]
\begin{remark}
The dashed look indicates intermediate fields; this picture is only suggestive.\footnote{Accurate lattice structure will be developed after FTGT.}
\end{remark}
