\section{Classical Examples}

\subsection{Quadratics and biquadratics}
\begin{example}
For $d$ squarefree, $L=\Q(\sqrt{d})$ is Galois of degree $2$ with group $C_2$.  
For $d_1,d_2$ distinct and squarefree, $E=\Q(\sqrt{d_1},\sqrt{d_2})$ has
\[
\Gal(E/\Q)\cong V_4,\qquad E=\Q\big(\sqrt{d_1},\sqrt{d_2}\big),
\]
and the three quadratic subfields correspond to the three order-$2$ subgroups.
\end{example}

\subsection{Cubic $x^3-2$}
\begin{example}
Let $f=x^3-2$. The discriminant is $\Delta(f)=-4(-2)^3-27(1)^2=(-4)(-8)-27=32-27=5$. Since $\Delta(f)$ is not a square in $\Q$, the Galois group of the splitting field is $S_3$; $[E:\Q]=6$ and $E=\Q(\sqrt[3]{2},\omega)$.
\end{example}

\subsection{Cyclotomic fields}
\begin{example}
For a prime $p$, the $p$-th cyclotomic field $K=\Q(\zeta_p)$ has degree $\varphi(p)=p-1$ and
\[
\Gal\big(\Q(\zeta_p)/\Q\big)\cong(\Z/p\Z)^\times\cong C_{p-1}.
\]
The unique quadratic subfield for $p\equiv1\pmod 4$ is fixed by the unique index-$2$ subgroup.
\end{example}

\subsection{Finite fields}
\begin{example}
Let $q=p^n$. The polynomial $x^{q}-x$ in $\F_q[x]$ splits with distinct roots and its roots are exactly the elements of $\F_q$. For $m\mid n$, the subextensions of $\F_{p^n}/\F_p$ are precisely $\F_{p^m}$.
\end{example}

\subsection{A quartic}
\begin{example}
For $f=x^4-2$, $E=\Q(\sqrt[4]{2},i)$ and $\Gal(E/\Q)\cong D_4$. The quadratic subfields are $\Q(\sqrt{2})$, $\Q(i)$, and $\Q(\sqrt{-2})$.
\[
\begin{tikzcd}
& E \arrow[-]{dl} \arrow[-]{d} \arrow[-]{dr} & \\
\Q(\sqrt[4]{2}) \arrow[-]{dr} & \Q(i,\sqrt{2}) \arrow[-]{d} & \Q(i\sqrt[4]{2}) \arrow[-]{dl} \\
& \Q(\sqrt{2}) \arrow[-]{d} & \\
& \Q &
\end{tikzcd}
\]
\end{example}

\subsection{Worked computations}

\begin{example}[Deciding $A_n$ vs $S_n$ by the discriminant]
Let $E$ be the splitting field of a separable $f\in\Q[x]$ of degree $n$. If the discriminant $\Delta(f)$ is a square in $\Q$, then $\Gal(E/\Q)\subseteq A_n$; otherwise it is not contained in $A_n$.
\end{example}

\begin{example}[A solvable quartic]
For $f=x^4+ax^2+b$ with $a^2\ne4b$, the resolvent quadratic $y^2-2ay+(a^2-4b)$ decides the structure: when it splits in $\Q$, the Galois group is contained in $V_4$; otherwise it is $D_4$.
\end{example}
