\section{Discriminants and Resolvents}

\subsection{Discriminant of a polynomial}

\begin{definition}
For a monic separable $f(x)=\prod_{i=1}^n (x-\alpha_i)\in K[x]$, the \emph{discriminant} is
\[
\Delta(f)=\prod_{i<j}(\alpha_i-\alpha_j)^2\in K.
\]
It equals $(-1)^{n(n-1)/2}\, \frac{1}{a_n}\operatorname{Res}(f,f')$ for a general leading coefficient $a_n$.
\end{definition}

\begin{proposition}[Parity test]
Let $E$ be the splitting field of a separable $f\in K[x]$ of degree $n$ over a field of characteristic $\neq 2$. Then
\[
\Delta(f)\text{ is a square in }K \iff \Gal(E/K)\subseteq A_n.
\]
\end{proposition}
\begin{proof}[Idea]
A permutation of the roots acts on $\sqrt{\Delta(f)}$ by its sign.
\end{proof}

\subsection{Discriminant of number fields}
\begin{definition}
If $L/K$ is a finite separable extension, the \emph{relative discriminant} $\mathfrak{D}_{L/K}$ is defined via any $K$-basis and the trace form; in characteristic $0$ it is an ideal of the ring of integers $\mathcal{O}_K$. For $K=\Q$ one gets an integer $\mathrm{disc}(L)$.
\end{definition}

\begin{example}
For $L=\Q(\sqrt{d})$ with squarefree $d$,
\[
\mathrm{disc}(L)=
\begin{cases}
d & d\equiv 1\pmod 4,\\
4d & d\equiv 2,3\pmod 4.
\end{cases}
\]
\end{example}

\subsection{Resolvents}

\begin{definition}[Resolvent idea]
A \emph{resolvent} is a polynomial whose roots are certain symmetric expressions in the roots of $f$, designed so that the action of $\Gal(E/K)$ can be detected by the factorization of the resolvent over $K$.
\end{definition}

\paragraph{Cubic resolvent for quartics.}
For $f(x)=x^4+ax^3+bx^2+cx+d$ define the \emph{cubic resolvent}
\[
R(y)=y^3-2by^2+(b^2+ac-4d)y+(4bd-a^2d-c^2).
\]
\begin{itemize}
\item If $R$ is irreducible and $\Delta(f)$ is not a square, then $\Gal(E/K)\cong D_4$.
\item If $R$ splits but $\Delta(f)$ not a square, then $\Gal(E/K)\cong V_4$ or $C_4$ (decide by the square status of certain coefficients).
\item If $\Delta(f)$ is a square and $R$ irreducible, then $\Gal(E/K)\subset A_4$ and often equals $A_4$.
\end{itemize}

\paragraph{Quadratic resolvent for cubics.}
For $f(x)=x^3+ax^2+bx+c$, the \emph{discriminant} already decides $A_3$ vs.\ $S_3$; one may also form the quadratic resolvent $y^2-4by+(a b-3c)^2$ to aid explicit formulas.

\subsection{Examples}

\begin{example}[Quartic $x^4-2$]
Here $a=c=0$, $b=0$, $d=-2$. Then $R(y)=y^3+8$ is irreducible over $\Q$, and $\Delta(f)=-2^{11}$ is not a square. Hence $\Gal(E/\Q)\cong D_4$.
\end{example}

\begin{example}[A biquadratic]
For $f(x)=x^4-5x^2+6=(x^2-2)(x^2-3)$, the discriminant is a square and the resolvent splits completely, giving $\Gal\cong V_4$.
\end{example}

\subsection{Remarks}
\begin{remark}
Resolvents are a practical classification tool up to degree $4$. For higher degrees, group-based tests, reductions modulo primes, and transitivity arguments are standard.
\end{remark}
