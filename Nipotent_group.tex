\documentclass{article}
\usepackage{amsmath,amsthm,amssymb}

\newtheorem{theorem}{Theorem}[section]
\newtheorem{lemma}[theorem]{Lemma}
\newtheorem{proposition}[theorem]{Proposition}
\newtheorem{corollary}[theorem]{Corollary}
\theoremstyle{definition}
\newtheorem{definition}[theorem]{Definition}
\newtheorem{example}[theorem]{Example}
\newtheorem{remark}[theorem]{Remark}

\title{Nilpotent Groups}
\author{Study Notes}
\date{}

\begin{document}
\maketitle
\tableofcontents

\section{Introduction}
Nilpotent groups occupy an important middle ground between abelian and general groups.  
They arise naturally in many parts of algebra and topology and often serve as a testing ground for ideas.  
The purpose of these notes is to collect the basic properties of nilpotent groups and to illustrate them with examples.

\begin{definition}
Let $G$ be a group.  The \emph{center} $Z(G)$ of $G$ is the subgroup of elements that commute with every element of $G$:
\[
Z(G)=\{ z\in G : zg=gz \text{ for all } g\in G\}.
\]
\end{definition}

\begin{definition}[Upper central series]
Define $Z_0(G)=1$ and inductively define $Z_{i+1}(G)$ so that $Z_{i+1}(G)/Z_i(G)=Z(G/Z_i(G))$.  
The chain
\[
1=Z_0(G)\le Z_1(G) \le Z_2(G) \le \cdots
\]
is called the \emph{upper central series} of $G$.
\end{definition}

\begin{definition}
A group $G$ is \emph{nilpotent} if its upper central series terminates at $G$ after finitely many steps, that is, $Z_c(G)=G$ for some integer $c\ge 0$.  The least such $c$ is the \emph{nilpotency class} of $G$.
\end{definition}

Every abelian group has class $1$ since $Z(G)=G$.  Many non-abelian nilpotent groups have large centers or are built from simple pieces, making them accessible to computation.

\section{Equivalent characterisations}
The following theorem lists several useful equivalent definitions of nilpotent groups.

\begin{theorem}\label{thm:equiv}
For a finite group $G$ the following are equivalent:
\begin{enumerate}
  \item $G$ is nilpotent.
  \item $G$ is the internal direct product of its Sylow subgroups.
  \item Every proper subgroup of $G$ is properly contained in its normaliser.
  \item For every $g\in G$, the conjugation map $x\mapsto gxg^{-1}$ has finite order dividing some power of a prime.
\end{enumerate}
\end{theorem}

\begin{proof}
The standard references contain detailed proofs; we sketch the key ideas.

$(1)\Rightarrow(2)$:  Let $P$ be a Sylow $p$-subgroup of $G$.  The upper central series of $G$ shows each $P$ is normal, hence $G$ is the direct product of its Sylow subgroups.

$(2)\Rightarrow(3)$:  Suppose $H$ is a proper subgroup.  As $G$ is a direct product of normal Sylow subgroups, $H$ misses some Sylow subgroup $P$.  Then $H$ normalises $H\cdot P$ properly, so $H\lneq N_G(H)$.

$(3)\Rightarrow(4)$:  For $g\in G$, the subgroup generated by $g$ sits properly inside its normaliser unless it is normal.  Hence the conjugation action by $g$ on $\langle g\rangle$ is trivial or has $p$-power order.

$(4)\Rightarrow(1)$:  Engel's theorem implies that condition~(4) forces $G$ to be nilpotent.
\end{proof}

A stronger notion of nilpotency uses the \emph{lower central series}.

\begin{definition}[Lower central series]
Set $\gamma_1(G)=G$ and define $\gamma_{i+1}(G)=[\gamma_i(G),G]$, the subgroup generated by commutators of elements of $\gamma_i(G)$ with $G$.  
We obtain a descending chain
\[
G=\gamma_1(G)\ge \gamma_2(G)\ge \gamma_3(G) \ge \cdots.
\]
The group is nilpotent of class at most $c$ if and only if $\gamma_{c+1}(G)=1$.
\end{definition}

One can show that the upper and lower central series terminate in the same number of steps.

\section{Basic properties}

We record some structural properties of nilpotent groups.

\begin{proposition}
Every subgroup and homomorphic image of a nilpotent group is nilpotent.  Furthermore, a finite extension of a nilpotent normal subgroup is nilpotent.
\end{proposition}
\begin{proof}
If $G$ is nilpotent then $Z_i(H)=Z_i(G)\cap H$ shows any subgroup $H$ is nilpotent.  
For a homomorphic image $G/N$, the sequence $Z_i(G/N)=Z_i(G)N/N$ is central.  
Finally, if $N\trianglelefteq G$ is nilpotent and $G/N$ is nilpotent, then $G$ is nilpotent by analysing a central series for $N$ and lifting it to $G$.
\end{proof}

\begin{proposition}
Let $G$ be nilpotent.  Then $Z(G)\ne 1$ and indeed $|Z(G)|\ge p$ for every prime $p$ dividing $|G|$.
\end{proposition}
\begin{proof}
We prove the claim by induction on the class $c$.  If $c=1$, $G$ is abelian and $Z(G)=G$.  For $c>1$, the quotient $G/Z(G)$ is nilpotent of class $c-1$, so by induction its center is nontrivial.  The preimage of $Z(G/Z(G))$ in $G$ properly contains $Z(G)$, showing $Z(G)$ is nontrivial.  The order bound follows because each Sylow subgroup is normal and intersects the center nontrivially.
\end{proof}

\begin{corollary}
Every finite $p$-group is nilpotent.
\end{corollary}
\begin{proof}
Apply the previous proposition to $G$ and note that the center of a nontrivial $p$-group is nontrivial.  Induct on the order.
\end{proof}

The converse is false: a nilpotent group need not have prime power order, but its Sylow subgroups interact simply.

\begin{theorem}[Sylow decomposition]
Let $G$ be a finite nilpotent group.  Then 
\[
G \cong P_1 \times P_2 \times \cdots \times P_k
\]
where the $P_i$ are the Sylow subgroups of $G$.  In particular $|G|$ is the product of coprime integers $|P_i|$.
\end{theorem}
\begin{proof}
Each Sylow subgroup is normal and any two intersect trivially because their orders are coprime.  Multiply the subgroups together to get the whole group.
\end{proof}

\begin{corollary}
A finite group is nilpotent if and only if it is the direct product of its Sylow subgroups.
\end{corollary}

\begin{proposition}
Let $G$ be finite and nilpotent.  Every maximal subgroup $M$ is normal and has prime index.
\end{proposition}
\begin{proof}
Let $P$ be a Sylow subgroup not contained in $M$.  Then $G=MP$ and $M\triangleleft G$ because $P$ normalises $M$.  The index $|G:M|$ divides $|P|$, hence is a power of $p$.  Maximality forces the index to equal $p$.
\end{proof}

Nilpotent groups sit inside the class of solvable groups.

\begin{proposition}
Every nilpotent group is solvable.  In fact, the derived series terminates no later than the lower central series.
\end{proposition}
\begin{proof}
Since $[\gamma_i(G),\gamma_i(G)] \subseteq \gamma_{2i}(G)$, the lower central series dominates the derived series.  If $\gamma_{c+1}(G)=1$ then after at most $c$ steps the derived series is trivial.
\end{proof}

\section{Examples}

We now collect some examples, from easy to more challenging.

\begin{example}
Any abelian group is nilpotent of class $1$.
\end{example}

\begin{example}
The quaternion group $Q_8=\{\pm1,\pm i,\pm j,\pm k\}$ has center $\{\pm 1\}$.  The quotient $Q_8/Z(Q_8)$ is the Klein four group, so $Q_8$ has class $2$.
\end{example}

\begin{example}[Heisenberg group]
Let $H$ be the group of $3\times3$ upper triangular matrices with ones on the diagonal over a field $F$:
\[
H=\left\{\begin{pmatrix}1&a&c\\0&1&b\\0&0&1\end{pmatrix} : a,b,c\in F\right\}.
\]
Computation shows $Z(H)$ consists of matrices with $a=b=0$.  The group has class $2$ regardless of the field.
\end{example}

\begin{example}[Unitriangular matrices]
The group $UT_n(F)$ of $n\times n$ upper triangular matrices with ones on the diagonal over a field $F$ is nilpotent of class $n-1$.  The lower central series is obtained by successively forcing more super-diagonals to vanish.
\end{example}

\begin{example}[Direct products]
If $G$ and $H$ are nilpotent, then so is $G\times H$, and its class is the maximum of the classes of $G$ and $H$.
\end{example}

\begin{example}[A non-example]
The symmetric group $S_3$ is solvable but not nilpotent because its Sylow $2$-subgroup is not normal.
\end{example}

\section{Engel conditions}

An element $x\in G$ is called \emph{left Engel} if for every $g\in G$ there exists $n$ with $[\cdots[[g,x],x],\ldots,x]=1$ (where $x$ occurs $n$ times).  A group in which every element is left Engel is called an \emph{Engel group}.

\begin{theorem}[Engel's Theorem]
A finite Engel group is nilpotent.
\end{theorem}

This theorem generalises the implication $(4)\Rightarrow(1)$ in Theorem~\ref{thm:equiv}.  The proof uses Lie ring methods and is beyond the scope of these notes, but the statement is a powerful tool.

\section{Fitting subgroup}

The \emph{Fitting subgroup} $F(G)$ of a finite group $G$ is the largest nilpotent normal subgroup.  It is the direct product of the normal Sylow subgroups of $G$.  The Fitting subgroup controls much of the structure of $G$ and plays a central role in solvable group theory.

\section{Exercises}

\begin{enumerate}
  \item Show that every subgroup of index $p$ in a finite nilpotent group is normal.
  \item Prove that a group of order $p^2q$ with $p<q$ primes is nilpotent.
  \item Determine the nilpotency class of $UT_4(\mathbb{Z}_p)$.
\end{enumerate}

\end{document}

