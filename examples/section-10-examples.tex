\begin{example}\label{ex:sec10-1}
\textbf{Problem.} Compute $\gcd(x^2-1,x^2-3)$ in $\mathbb{Q}[x]$.

\textbf{Solution.} The gcd is $1$ since the polynomials have distinct roots.
\end{example}

\begin{example}\label{ex:sec10-2}
\textbf{Problem.} Factor $x^2+2x+1$.

\textbf{Solution.} $(x+1)^2$.
\end{example}

\begin{example}\label{ex:sec10-3}
\textbf{Problem.} Solve $x^2+1=0$ over $\mathbb{C}$.

\textbf{Solution.} $x=\pm i$.
\end{example}

\begin{example}\label{ex:sec10-4}
\textbf{Problem.} Compute $[\mathbb{Q}(\sqrt{2}):\mathbb{Q}]$.

\textbf{Solution.} The degree is $2$.
\end{example}

\begin{example}\label{ex:sec10-5}
\textbf{Problem.} Find the inverse of $2$ in $\mathbb{F}_5$.

\textbf{Solution.} $2\cdot3=6\equiv1$, so the inverse is $3$.
\end{example}

\begin{example}\label{ex:sec10-6}
\textbf{Problem.} Determine the minimal polynomial of $i$ over $\mathbb{Q}$.

\textbf{Solution.} It is $x^2+1$.
\end{example}

\begin{example}\label{ex:sec10-7}
\textbf{Problem.} Compute $(1+i)^2$.

\textbf{Solution.} $(1+i)^2=1+2i+i^2=2i$.
\end{example}

\begin{example}\label{ex:sec10-8}
\textbf{Problem.} Evaluate $\Norm_{\mathbb{Q}(\sqrt{2})/\mathbb{Q}}(1+\sqrt{2})$.

\textbf{Solution.} $(1+\sqrt{2})(1-\sqrt{2})=-1$.
\end{example}

\begin{example}\label{ex:sec10-9}
\textbf{Problem.} Find $\Tr_{\mathbb{Q}(\sqrt{2})/\mathbb{Q}}(1+\sqrt{2})$.

\textbf{Solution.} $(1+\sqrt{2})+(1-\sqrt{2})=2$.
\end{example}

\begin{example}\label{ex:sec10-10}
\textbf{Problem.} Compute $\Gal(\mathbb{Q}(i)/\mathbb{Q})$.

\textbf{Solution.} It has two elements: identity and complex conjugation.
\end{example}
