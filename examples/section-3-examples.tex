\begin{example}\label{ex:sec3-1}
\textbf{Problem.} What is the splitting field of $x^2-2$ over $\mathbb{Q}$?

\textbf{Solution.} The polynomial factors as $(x-\sqrt{2})(x+\sqrt{2})$, so the splitting field is $\mathbb{Q}(\sqrt{2})$.
\end{example}

\begin{example}\label{ex:sec3-2}
\textbf{Problem.} Find the splitting field of $x^2+1$ over $\mathbb{Q}$.

\textbf{Solution.} Its roots are $\pm i$, so the splitting field is $\mathbb{Q}(i)$.
\end{example}

\begin{example}\label{ex:sec3-3}
\textbf{Problem.} Compute the splitting field of $x^3-1$ over $\mathbb{Q}$.

\textbf{Solution.} The roots are $1$ and the primitive cube roots $\zeta_3,\zeta_3^2$, so the splitting field is $\mathbb{Q}(\zeta_3)$.
\end{example}

\begin{example}\label{ex:sec3-4}
\textbf{Problem.} Is $\mathbb{Q}$ algebraically closed?

\textbf{Solution.} No, $x^2+1$ has no root in $\mathbb{Q}$.
\end{example}

\begin{example}\label{ex:sec3-5}
\textbf{Problem.} Why is every polynomial over $\mathbb{C}$ split?

\textbf{Solution.} The fundamental theorem of algebra states that $\mathbb{C}$ is algebraically closed.
\end{example}

\begin{example}\label{ex:sec3-6}
\textbf{Problem.} What is the splitting field of $x^3-2$ over $\mathbb{Q}$?

\textbf{Solution.} It is $\mathbb{Q}(\sqrt[3]{2},\zeta_3)$, adjoining a cube root and a primitive cube root of unity.
\end{example}

\begin{example}\label{ex:sec3-7}
\textbf{Problem.} Determine $[\mathbb{Q}(\sqrt[4]{5}, i):\mathbb{Q}]$.

\textbf{Solution.} The extension by $\sqrt[4]{5}$ has degree $4$, adjoining $i$ doubles it to $8$.
\end{example}

\begin{example}\label{ex:sec3-8}
\textbf{Problem.} Are algebraic closures unique up to isomorphism?

\textbf{Solution.} Yes, any two algebraic closures of a field are isomorphic.
\end{example}

\begin{example}\label{ex:sec3-9}
\textbf{Problem.} Find the splitting field of $x^4-1$ over $\mathbb{Q}$.

\textbf{Solution.} The roots are $\pm1,\pm i$, so the splitting field is $\mathbb{Q}(i)$.
\end{example}

\begin{example}\label{ex:sec3-10}
\textbf{Problem.} What is the splitting field of $(x^2-2)(x^2-3)$ over $\mathbb{Q}$?

\textbf{Solution.} Adjoining $\sqrt{2}$ and $\sqrt{3}$ suffices, so the field is $\mathbb{Q}(\sqrt{2},\sqrt{3})$.
\end{example}
