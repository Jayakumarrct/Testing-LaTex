\begin{example}\label{ex:sec4-1}
\textbf{Problem.} Define an automorphism of $\mathbb{Q}(\sqrt{2})$ over $\mathbb{Q}$.

\textbf{Solution.} Sending $\sqrt{2}$ to $-\sqrt{2}$ and fixing $\mathbb{Q}$ defines an automorphism.
\end{example}

\begin{example}\label{ex:sec4-2}
\textbf{Problem.} What is the fixed field of the automorphism above?

\textbf{Solution.} Only rational numbers are fixed, so the fixed field is $\mathbb{Q}$.
\end{example}

\begin{example}\label{ex:sec4-3}
\textbf{Problem.} Is complex conjugation an automorphism of $\mathbb{C}$ over $\mathbb{R}$?

\textbf{Solution.} Yes, it preserves addition and multiplication and fixes $\mathbb{R}$.
\end{example}

\begin{example}\label{ex:sec4-4}
\textbf{Problem.} Determine $\Aut_{\mathbb{Q}}(\mathbb{Q}(i))$.

\textbf{Solution.} There are two automorphisms: identity and complex conjugation, so the group has order $2$.
\end{example}

\begin{example}\label{ex:sec4-5}
\textbf{Problem.} Find an automorphism of $\mathbb{Q}(\sqrt{3})$ over $\mathbb{Q}$.

\textbf{Solution.} Mapping $\sqrt{3}$ to $-\sqrt{3}$ and fixing $\mathbb{Q}$ is an automorphism.
\end{example}

\begin{example}\label{ex:sec4-6}
\textbf{Problem.} What is the fixed field of the Frobenius map $x\mapsto x^p$ on $\mathbb{F}_{p^2}$?

\textbf{Solution.} Elements with $x^p=x$ form $\mathbb{F}_p$.
\end{example}

\begin{example}\label{ex:sec4-7}
\textbf{Problem.} Compute $\Aut_{\mathbb{F}_2}(\mathbb{F}_4)$.

\textbf{Solution.} The Frobenius $x\mapsto x^2$ generates a cyclic group of order $2$.
\end{example}

\begin{example}\label{ex:sec4-8}
\textbf{Problem.} Can $\mathbb{Q}$ have a nontrivial automorphism fixing $\mathbb{Q}$?

\textbf{Solution.} No, every element is rational, so only the identity works.
\end{example}

\begin{example}\label{ex:sec4-9}
\textbf{Problem.} Describe an automorphism of $\mathbb{C}$ fixing $\mathbb{Q}(i)$.

\textbf{Solution.} Complex conjugation fixes $i$ up to sign, so only the identity fixes $\mathbb{Q}(i)$ pointwise.
\end{example}

\begin{example}\label{ex:sec4-10}
\textbf{Problem.} If an automorphism of a field extension fixes a basis, what is it?

\textbf{Solution.} It must be the identity, since elements are linear combinations of the basis.
\end{example}
