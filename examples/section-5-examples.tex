\begin{example}\label{ex:sec5-1}
\textbf{Problem.} Is $x^2-2$ separable over $\mathbb{Q}$?

\textbf{Solution.} Yes, it has distinct roots $\pm\sqrt{2}$.
\end{example}

\begin{example}\label{ex:sec5-2}
\textbf{Problem.} Is $x^p-a$ separable over $\mathbb{F}_p$ when $a$ is not a $p$th power?

\textbf{Solution.} No, its derivative is $0$, so it has a repeated root and is inseparable.
\end{example}

\begin{example}\label{ex:sec5-3}
\textbf{Problem.} Is $x^2-t$ separable over $\mathbb{F}_2(t)$?

\textbf{Solution.} The derivative is $2x=0$, yet $x^2-t$ has distinct roots, so it is inseparable.
\end{example}

\begin{example}\label{ex:sec5-4}
\textbf{Problem.} Is $\mathbb{Q}(\sqrt{2})/\mathbb{Q}$ normal?

\textbf{Solution.} Yes, it is the splitting field of $x^2-2$.
\end{example}

\begin{example}\label{ex:sec5-5}
\textbf{Problem.} Is $\mathbb{Q}(\sqrt[3]{2})/\mathbb{Q}$ normal?

\textbf{Solution.} No, $x^3-2$ has complex roots not in the field.
\end{example}

\begin{example}\label{ex:sec5-6}
\textbf{Problem.} Are finite extensions of characteristic $0$ separable?

\textbf{Solution.} Yes, characteristic $0$ implies all irreducible polynomials have distinct roots.
\end{example}

\begin{example}\label{ex:sec5-7}
\textbf{Problem.} Is $\mathbb{Q}(\sqrt{2},\sqrt[3]{2})/\mathbb{Q}$ separable?

\textbf{Solution.} Both subextensions are separable, so the compositum is separable.
\end{example}

\begin{example}\label{ex:sec5-8}
\textbf{Problem.} Is $x^p-x$ separable over $\mathbb{F}_p$?

\textbf{Solution.} Its derivative is $-1$, so all roots are simple; it is separable.
\end{example}

\begin{example}\label{ex:sec5-9}
\textbf{Problem.} What indicates that a polynomial is inseparable?

\textbf{Solution.} It has repeated roots in its splitting field.
\end{example}

\begin{example}\label{ex:sec5-10}
\textbf{Problem.} Describe $\mathbb{F}_p(t^{1/p})/\mathbb{F}_p(t)$.

\textbf{Solution.} It is purely inseparable of degree $p$, generated by a single $p$th root.
\end{example}
