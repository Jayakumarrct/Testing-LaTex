\begin{example}\label{ex:sec1-1}
\textbf{Problem.} Is $\mathbb{Z}$ a field?

\textbf{Solution.} Elements such as $2$ lack multiplicative inverses in $\mathbb{Z}$, so it is not a field.
\end{example}

\begin{example}\label{ex:sec1-2}
\textbf{Problem.} What is the characteristic of $\mathbb{F}_7$?

\textbf{Solution.} Adding $1$ seven times gives $0$, so the characteristic is $7$.
\end{example}

\begin{example}\label{ex:sec1-3}
\textbf{Problem.} Evaluate $p(x)=x^2+3x+2$ at $x=1$ in $\mathbb{Q}$.

\textbf{Solution.} $1^2+3\cdot1+2=6$.
\end{example}

\begin{example}\label{ex:sec1-4}
\textbf{Problem.} Factor $x^2-9$ over $\mathbb{Q}$.

\textbf{Solution.} $x^2-9=(x-3)(x+3)$.
\end{example}

\begin{example}\label{ex:sec1-5}
\textbf{Problem.} Find $\gcd(x^3-x, x^2-1)$ in $\mathbb{Q}[x]$.

\textbf{Solution.} $x^3-x = x(x^2-1)$, so the gcd is $x^2-1$.
\end{example}

\begin{example}\label{ex:sec1-6}
\textbf{Problem.} Is $x^2+1$ reducible over $\mathbb{R}$?

\textbf{Solution.} It has no real roots, so it is irreducible over $\mathbb{R}$.
\end{example}

\begin{example}\label{ex:sec1-7}
\textbf{Problem.} Why is $\mathbb{Q}[x]$ a Euclidean domain?

\textbf{Solution.} The division algorithm uses degree as a Euclidean function.
\end{example}

\begin{example}\label{ex:sec1-8}
\textbf{Problem.} Divide $x^3+1$ by $x+1$ in $\mathbb{Q}[x]$.

\textbf{Solution.} $x^3+1=(x+1)(x^2-x+1)$, so quotient $x^2-x+1$ and remainder $0$.
\end{example}

\begin{example}\label{ex:sec1-9}
\textbf{Problem.} Find $[\mathbb{Q}(\sqrt{5}):\mathbb{Q}]$.

\textbf{Solution.} $\sqrt{5}$ satisfies $x^2-5$, so the degree is $2$.
\end{example}

\begin{example}\label{ex:sec1-10}
\textbf{Problem.} Find the minimal polynomial of $\sqrt{2}+\sqrt{3}$ over $\mathbb{Q}$.

\textbf{Solution.} Let $\alpha=\sqrt{2}+\sqrt{3}$. Then $(\alpha^2-5)^2=24$, giving $\alpha^4-10\alpha^2+1=0$.
\end{example}
