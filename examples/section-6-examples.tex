\begin{example}\label{ex:sec6-1}
\textbf{Problem.} Is $\mathbb{Q}(\sqrt{2})/\mathbb{Q}$ a Galois extension?

\textbf{Solution.} Yes, it is normal and separable; its Galois group has two elements.
\end{example}

\begin{example}\label{ex:sec6-2}
\textbf{Problem.} Is $\mathbb{Q}(\sqrt[3]{2})/\mathbb{Q}$ Galois?

\textbf{Solution.} No, it is not normal since complex cube roots are missing.
\end{example}

\begin{example}\label{ex:sec6-3}
\textbf{Problem.} Find $\Gal(\mathbb{Q}(i)/\mathbb{Q})$.

\textbf{Solution.} It has two elements: identity and complex conjugation.
\end{example}

\begin{example}\label{ex:sec6-4}
\textbf{Problem.} What conditions make a finite extension Galois?

\textbf{Solution.} Being both normal and separable.
\end{example}

\begin{example}\label{ex:sec6-5}
\textbf{Problem.} What is the fixed field of $\{\text{id},\text{conj}\}\subset \Gal(\mathbb{Q}(i)/\mathbb{Q})$?

\textbf{Solution.} The fixed field is $\mathbb{Q}$.
\end{example}

\begin{example}\label{ex:sec6-6}
\textbf{Problem.} Describe the intermediate fields of $\mathbb{Q}(\sqrt{2},\sqrt{3})/\mathbb{Q}$.

\textbf{Solution.} They are $\mathbb{Q}$, $\mathbb{Q}(\sqrt{2})$, $\mathbb{Q}(\sqrt{3})$, and $\mathbb{Q}(\sqrt{2},\sqrt{3})$.
\end{example}

\begin{example}\label{ex:sec6-7}
\textbf{Problem.} What is the Galois group of $\mathbb{Q}(\sqrt{2},\sqrt{3})/\mathbb{Q}$?

\textbf{Solution.} It is the Klein four group, generated by changing signs of $\sqrt{2}$ and $\sqrt{3}$.
\end{example}

\begin{example}\label{ex:sec6-8}
\textbf{Problem.} How does the fundamental theorem relate subgroups and subfields?

\textbf{Solution.} Intermediate fields correspond to subgroups of the Galois group, reversed by inclusion.
\end{example}

\begin{example}\label{ex:sec6-9}
\textbf{Problem.} Verify the correspondence for $\mathbb{Q}(i)/\mathbb{Q}$.

\textbf{Solution.} Subgroup $\{\text{id},\text{conj}\}$ matches subfield $\mathbb{Q}$; the trivial subgroup matches $\mathbb{Q}(i)$.
\end{example}

\begin{example}\label{ex:sec6-10}
\textbf{Problem.} If $L/K$ is Galois with group $G$, what is $K$?

\textbf{Solution.} It is the fixed field $L^G$.
\end{example}
