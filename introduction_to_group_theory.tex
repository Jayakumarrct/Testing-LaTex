\documentclass[12pt]{article}
\usepackage{amsmath,amssymb,amsfonts}
\usepackage{xcolor}
\usepackage{hyperref}

\title{From Zero to Lagrange\\\small Lecture Notes on Group Theory}
\author{}
\date{}

\begin{document}
\maketitle

\section*{1. Getting Started}
To study groups we need a set $G$ together with a way to combine any two of its elements.
Throughout these notes the operation will be denoted by $\star$.
Before calling $(G,\star)$ a \textcolor{blue}{group} we must check four requirements: \textcolor{blue}{closure}, \textcolor{blue}{associativity}, the presence of an \textcolor{blue}{identity} element, and the existence of \textcolor{blue}{inverses}.

\section*{2. Definition of a Group}
\textcolor{blue}{\textbf{Definition.}} A pair $(G,\star)$ is a group if for all $a,b,c\in G$:
\begin{enumerate}
  \item \textbf{Closure:} $a\star b$ lies in $G$.
  \item \textbf{Associativity:} $(a\star b)\star c = a\star(b\star c)$.
  \item \textbf{Identity:} there exists $e\in G$ with $e\star a=a\star e=a$ for every $a$.
  \item \textbf{Inverses:} each $a$ has $a^{-1}$ in $G$ such that $a\star a^{-1}=a^{-1}\star a=e$.
\end{enumerate}
If $a\star b=b\star a$ for all $a,b$, the group is called \textcolor{purple}{abelian}.

\section*{3. Verifying a Group}
When presenting a new structure, begin by describing the set and the operation.
Check closure by combining generic elements, appeal to known associativity when possible, identify the element that leaves others unchanged, and finally produce an inverse for a typical element.

\section*{4. Examples and Non-examples}
\subsection*{Example: $(\mathbb{Z},+)$}
Given integers $m$ and $n$, their sum $m+n$ is an integer, so the set is closed under addition.
Addition of integers is associative, $0$ acts as the identity, and $m+(-m)=0$ shows that $-m$ is the inverse of $m$.
The operation is commutative, hence $\mathbb{Z}$ is abelian.

\subsection*{Example: $(\mathbb{Z}_n,+ \text{ mod } n)$}
The set $\mathbb{Z}_n=\{0,1,\ldots,n-1\}$ is closed under addition modulo $n$.
For instance in $\mathbb{Z}_6$,
\[
4+5=9\equiv 3 \pmod 6.
\]
The class $0$ serves as identity and the inverse of $k$ is $n-k$.

\subsection*{Example: $(\mathbb{Q}^*,\times)$ and $(\mathbb{R}^*,\times)$}
Multiplication of nonzero rationals or reals is associative and closed.
The number $1$ is the identity and the inverse of $\frac{a}{b}$ is $\frac{b}{a}$.
Both groups are abelian.

\subsection*{Example: $\mathrm{GL}_2(\mathbb{R})$}
This group consists of all invertible $2\times2$ real matrices under multiplication.
Closure follows because the product of invertible matrices is invertible.
For
\[
A=\begin{bmatrix}1&1\\0&1\end{bmatrix},\quad
B=\begin{bmatrix}1&0\\1&1\end{bmatrix},
\]
we compute
\[
AB=\begin{bmatrix}2&1\\1&1\end{bmatrix},\qquad
BA=\begin{bmatrix}1&1\\1&2\end{bmatrix},
\]
so $AB\neq BA$ and the group is not abelian.

\subsection*{Example: Symmetric Group $S_3$}
$S_3$ contains the six permutations of $\{1,2,3\}$.
Composition of permutations is associative and the identity map fixes each number.
The permutation $(12)(23)$ sends $1\mapsto2\mapsto3$, $2\mapsto3\mapsto1$, and $3\mapsto1\mapsto2$, giving $(123)$.
Because $(12)(23)\neq(23)(12)$ the group is non-abelian.

\subsection*{Example: Dihedral Group $D_4$}
$D_4$ records symmetries of a square: four rotations and four reflections.
Rotation by $90^{\circ}$ composed four times returns the square to its original position, while reflections reverse orientation; the operation is composition of these symmetries.
The group fails to be abelian because, for example, reflecting then rotating is different from rotating then reflecting.

\subsection*{Example: Klein Four Group $V_4$}
The set $\{e,a,b,c\}$ with relations $a^2=b^2=c^2=e$ and $ab=c$, $bc=a$, $ca=b$ forms a group.
Each element is its own inverse and the operation is commutative, but no single element generates the entire group.

\subsection*{Non-examples}
The natural numbers under addition lack inverses, the integers under multiplication lack inverses for most elements, and real numbers with subtraction violate associativity.

\section*{5. Fundamental Notions}
The \textcolor{blue}{order} of an element $a$ is the smallest positive integer $k$ with $a^k=e$ (or $k\cdot a=0$ in additive notation).
A group is \textcolor{blue}{cyclic} if one element generates all others: $G=\langle g\rangle=\{\ldots,g^{-1},e,g,g^{2},\ldots\}$.
The \textcolor{blue}{order of the group} is the number of elements it contains.

\section*{6. Subgroups}
\textcolor{blue}{\textbf{Definition.}} A nonempty subset $H$ of $G$ is a subgroup, written $H\leq G$, if for all $a,b\in H$ the element $ab^{-1}$ also lies in $H$.
In additive groups we test closure under subtraction.

\subsection*{Useful Facts}
\begin{itemize}
  \item The identity of $G$ is always in $H$.
  \item If $a\in H$ then $a^{-1}\in H$.
  \item Intersections of subgroups are subgroups.
  \item The subgroup generated by a set $S$ is the smallest subgroup containing $S$ and is denoted $\langle S\rangle$.
\end{itemize}

\subsection*{Examples}
\paragraph{1. $k\mathbb{Z}$ inside $\mathbb{Z}$.}
For an integer $k$, the set $k\mathbb{Z}=\{\ldots,-2k,-k,0,k,2k,\ldots\}$ is closed under subtraction, hence forms a subgroup.

\paragraph{2. Subgroups of $\mathbb{Z}_6$.}
Every subgroup is generated by a divisor of $6$:
\[
\{0\},\quad \{0,3\},\quad \{0,2,4\},\quad \mathbb{Z}_6.
\]
The subset $\{0,2\}$ is not a subgroup since $2+2=4\notin\{0,2\}$.

\paragraph{3. Subgroups of $S_3$.}
They have sizes $1,2,3,$ or $6$.
Besides the trivial subgroup and $S_3$ itself, there are three subgroups generated by transpositions and a three-element subgroup $\{e,(123),(132)\}$.

\paragraph{4. The center $Z(G)$.}
$Z(G)=\{g\in G: gx=xg \text{ for all } x\in G\}$ is a subgroup because products and inverses of commuting elements also commute.

\paragraph{5. Rotations in $D_4$.}
The set $\{e,r,r^2,r^3\}$ of rotations is a subgroup of order $4$.

\paragraph{6. Upper triangular matrices.}
The invertible upper triangular matrices form a subgroup of $\mathrm{GL}_2(\mathbb{R})$ since the product and inverse of such matrices remain upper triangular.

\section*{7. Cosets and Index}
Given $H\leq G$ and $g\in G$, the \textcolor{blue}{left coset} $gH$ is $\{gh:h\in H\}$ and the \textcolor{blue}{right coset} $Hg$ is $\{hg:h\in H\}$.
All cosets have the same size as $H$; two cosets are either identical or disjoint, and the left cosets partition $G$.
The number of distinct left cosets is called the \textcolor{blue}{index} and is written $[G:H]$.

\subsection*{Examples}
\paragraph{$\mathbb{Z}_6$ with $H=\{0,3\}$.}
The cosets are
\[
0+H=\{0,3\},\quad 1+H=\{1,4\},\quad 2+H=\{2,5\},
\]
so $[\mathbb{Z}_6:H]=3$.

\paragraph{$S_3$ with $H=\langle(12)\rangle$.}
We obtain three cosets:
\[
H=\{e,(12)\},\quad (13)H=\{(13),(132)\},\quad (23)H=\{(23),(123)\}.
\]
Each coset has two elements, hence the index is $3$.

\section*{8. Lagrange's Theorem}
\textcolor{blue}{\textbf{Theorem.}} If $G$ is finite and $H\leq G$, then $|G|=[G:H]\cdot|H|$.
\textcolor{blue}{\textbf{Proof sketch.}} Distinct left cosets of $H$ partition $G$ and each coset has $|H|$ elements, so the order of $G$ is a multiple of the order of $H$.

\subsection*{Corollaries}
\begin{itemize}
  \item The order of an element divides the order of the group; consequently $a^{|G|}=e$ for all $a$ in a finite group.
  \item A group of prime order is cyclic.
  \item Fermat's little theorem: for prime $p$ and $a$ not divisible by $p$, $a^{p-1}\equiv1\pmod p$.
\end{itemize}

\subsection*{Applications}
\paragraph{Subgroup sizes of $S_3$.}
Possible sizes are the divisors of $6$, namely $1,2,3,$ and $6$.

\paragraph{Orders in $\mathbb{Z}_{12}$.}
The element $4$ has order $3$ because $4+4+4=12\equiv0$, and similarly $2$ has order $6$ and $6$ has order $2$.

\paragraph{No element of order $6$ in $S_3$.}
Although $6$ divides $|S_3|$, the elements of $S_3$ have orders $1,2,$ or $3$ only.

\paragraph{Index in $\mathbb{Z}_{12}$.}
With $H=\langle4\rangle=\{0,4,8\}$ we have four distinct cosets
\[
\{0,4,8\},\quad \{1,5,9\},\quad \{2,6,10\},\quad \{3,7,11\},
\]
so $[\mathbb{Z}_{12}:H]=4$.

\section*{9. Common Pitfalls}
\begin{itemize}
  \item Closure means results stay in the set; it is not automatic from non-emptiness.
  \item Many groups are not abelian—never assume commutativity.
  \item Lagrange's theorem gives a necessary divisibility condition but does not guarantee the existence of elements of a given order.
  \item Every subgroup must contain the identity element of the ambient group.
\end{itemize}

\section*{10. Practice Problems}
\begin{itemize}
  \item Is $\{0,3,6,9\}$ a subgroup of $\mathbb{Z}_{12}$ under addition?
  \item List all cosets of $H=\{0,2,4\}$ in $\mathbb{Z}_6$.
  \item In $S_3$, why is $\{(12),(13)\}$ not a subgroup?
  \item Find the order of $2$ in $\mathbb{Z}_8$ and describe $\langle2\rangle$.
  \item Prove that any finite subgroup of $(\mathbb{R}^*,\times)$ is cyclic.
\end{itemize}

\section*{Summary}
Groups combine elements through a binary operation obeying closure, associativity, an identity, and inverses.
Subgroups satisfy the same rules and can be tested with the single condition $ab^{-1}\in H$.
Cosets partition a group into equal-sized pieces, leading to Lagrange's theorem that the order of a subgroup divides the order of the whole group.
These tools let us compute element orders, classify small groups, and apply group theory to number theory.

\end{document}
