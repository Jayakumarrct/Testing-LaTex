\documentclass[12pt]{article}
\usepackage{amsmath,amssymb,amsfonts}
\usepackage{xcolor}
\usepackage{tcolorbox}
\usepackage{hyperref}
\tcbset{colback=white,colframe=black}
\newtcolorbox{definitionbox}[1]{colback=blue!5!white,colframe=blue!75!black,title=\textbf{#1}}
\newtcolorbox{examplebox}[1]{colback=green!5!white,colframe=green!60!black,title=\textbf{#1}}
\newtcolorbox{warnbox}[1]{colback=red!5!white,colframe=red!60!black,title=\textbf{#1}}

\title{From Zero to Lagrange\\\small A Friendly Introduction to Group Theory}
\author{}
\date{}

\begin{document}
\maketitle

\section*{Introduction: What You Need}
\begin{definitionbox}{Starting Point}
To talk about groups, you begin with:
\begin{itemize}
  \item a set $G$,
  \item a binary rule $\star$ that takes $a,b$ in $G$ and returns $a\star b$ in $G$,
  \item four checks: \textcolor{blue}{closure}, \textcolor{blue}{associativity}, \textcolor{blue}{identity}, and \textcolor{blue}{inverses}.
\end{itemize}
\end{definitionbox}

\section*{Definition: Group}
\begin{definitionbox}{What is a Group?}
A pair $(G,\star)$ is a \textcolor{blue}{group} if for all $a,b,c\in G$:
\begin{enumerate}
  \item \textbf{Closure:} $a\star b$ is in $G$.
  \item \textbf{Associativity:} $(a\star b)\star c = a\star(b\star c)$.
  \item \textbf{Identity:} there exists $e$ with $e\star a = a\star e = a$.
  \item \textbf{Inverse:} each $a$ has $a^{-1}$ with $a\star a^{-1} = a^{-1}\star a = e$.
\end{enumerate}
The group is \textcolor{purple}{abelian} if $a\star b = b\star a$ for all $a,b$.
\end{definitionbox}

\section*{How to Prove ``This is a Group"}
\begin{examplebox}{Group Proof Recipe}
\begin{enumerate}
  \item Describe $G$ and the operation $\star$.
  \item Show \textcolor{blue}{closure}.
  \item Use known \textcolor{blue}{associativity} (for $+$ or $\times$) or compute directly.
  \item Give the \textcolor{blue}{identity} element.
  \item For any $a$, find $a^{-1}$ and check it works.
\end{enumerate}
\end{examplebox}

\section*{Worked Examples: Groups and Non-Examples}
\begin{examplebox}{1. $(\mathbb{Z}, +)$}
Closure: $m+n$ is an integer. Identity: $0$. Inverse of $m$ is $-m$. This group is abelian.
\end{examplebox}

\begin{examplebox}{2. $(\mathbb{Z}_n, +\ \text{mod}\ n)$}
Elements: $\{0,1,\dots,n-1\}$. Operation: addition mod $n$. Identity: $0$. Inverse of $k$ is $n-k$. Abelian.
\textit{Example:} in $\mathbb{Z}_6$, $4+5\equiv3$ and the inverse of $4$ is $2$.
\end{examplebox}

\begin{examplebox}{3. $(\mathbb{Q}^*,\times)$ and $(\mathbb{R}^*,\times)$}
Identity: $1$. Inverse of $a/b$ is $b/a$. Both are abelian.
\end{examplebox}

\begin{examplebox}{4. $\mathrm{GL}_2(\mathbb{R})$}
$2\times2$ invertible real matrices under multiplication. Identity is $I=\begin{bmatrix}1&0\\0&1\end{bmatrix}$. Inverses are usual matrix inverses. Generally non-abelian. For example,\[\begin{bmatrix}1&1\\0&1\end{bmatrix}\begin{bmatrix}1&0\\1&1\end{bmatrix}=\begin{bmatrix}2&1\\1&1\end{bmatrix} \neq \begin{bmatrix}1&0\\1&1\end{bmatrix}\begin{bmatrix}1&1\\0&1\end{bmatrix}=\begin{bmatrix}1&1\\1&2\end{bmatrix}.\]
\end{examplebox}

\begin{examplebox}{5. Symmetric Group $S_3$}
Six permutations of $\{1,2,3\}$ with composition. Non-abelian: $(12)\circ(23) \neq (23)\circ(12)$. Example: $(12)(23)=(123)$.
\end{examplebox}

\begin{examplebox}{6. Dihedral Group $D_4$}
Symmetries of a square: four rotations and four reflections. Non-abelian.
\end{examplebox}

\begin{examplebox}{7. Klein Four Group $V_4$}
Elements $\{e,a,b,c\}$ with $a^2=b^2=c^2=e$ and $ab=c$, $bc=a$, $ca=b$. Abelian but not cyclic.
\end{examplebox}

\begin{warnbox}{Non-Examples}
\begin{itemize}
  \item $(\mathbb{N},+)$ fails inverses.
  \item $(\mathbb{Z},\times)$ fails inverses for most elements.
  \item $(\mathbb{R},-)$ fails associativity.
\end{itemize}
\end{warnbox}

\section*{Basic Concepts You Will Use Soon}
\begin{definitionbox}{Key Ideas}
\begin{itemize}
  \item \textbf{Order of an element} $a$: smallest $k>0$ with $a^k=e$ (or $k\cdot a=0$ in additive groups).
  \item \textbf{Cyclic group:} a group generated by one element $g$.
  \item \textbf{Order of a group:} number of elements $|G|$ if finite.
\end{itemize}
\end{definitionbox}

\section*{Subgroups}
\begin{definitionbox}{What is a Subgroup?}
A nonempty subset $H$ of $G$ is a \textcolor{blue}{subgroup} if for all $a,b\in H$, the element $a\star b^{-1}$ is also in $H$. For additive groups replace $a\star b^{-1}$ by $a-b$.
\end{definitionbox}

\subsection*{Quick Subgroup Facts}
\begin{itemize}
  \item The identity $e$ lies in every subgroup.
  \item If $a\in H$ then $a^{-1}\in H$.
  \item Intersections of subgroups are subgroups.
  \item The subgroup generated by $S$ is $\langle S\rangle$.
\end{itemize}

\subsection*{Worked Examples: Subgroups}
\begin{examplebox}{1. $k\mathbb{Z} \leq \mathbb{Z}$}
$k\mathbb{Z}=\{\ldots,-2k,-k,0,k,2k,\ldots\}$ is closed under subtraction, so it is a subgroup. Examples include even integers $2\mathbb{Z}$.
\end{examplebox}

\begin{examplebox}{2. Subgroups of $\mathbb{Z}_6$ under $+$}
Every subgroup is $\langle d\rangle$ where $d$ divides $6$. They are $\{0\}$, $\{0,3\}$, $\{0,2,4\}$, and $\mathbb{Z}_6$. The subset $T=\{0,2\}$ fails because $2+2=4\notin T$.
\end{examplebox}

\begin{examplebox}{3. Subgroups of $S_3$}
Possible sizes: $1,2,3,6$. Subgroups: $\{e\}$; three generated by transpositions $\langle(12)\rangle$, $\langle(13)\rangle$, $\langle(23)\rangle$; the order-3 subgroup $A_3=\langle(123)\rangle$; and $S_3$ itself.
\end{examplebox}

\begin{examplebox}{4. Center $Z(G)$}
$Z(G)=\{g\in G: gx=xg \text{ for all } x\in G\}$ is a subgroup.
\end{examplebox}

\begin{examplebox}{5. Rotation Subgroup of $D_4$}
$R=\{e,r,r^2,r^3\}$ with $r$ a $90^\circ$ rotation is a subgroup of order $4$.
\end{examplebox}

\begin{examplebox}{6. Upper Triangular Matrices}
Invertible upper triangular matrices form a subgroup of $\mathrm{GL}_2(\mathbb{R})$.
\end{examplebox}

\section*{Cosets and Index}
\begin{definitionbox}{Cosets}
For $H\leq G$ and $g\in G$:
\begin{itemize}
  \item Left coset: $gH=\{g\star h: h\in H\}$.
  \item Right coset: $Hg=\{h\star g: h\in H\}$.
\end{itemize}
Cosets are the same size as $H$, either equal or disjoint, and they partition $G$. The \textbf{index} $[G:H]$ counts the left cosets.
\end{definitionbox}

\subsection*{Coset Examples}
\begin{examplebox}{1. $\mathbb{Z}_6$ with $H=\{0,3\}$}
Cosets: $\{0,3\}$, $\{1,4\}$, $\{2,5\}$. Each has two elements, so $[\mathbb{Z}_6:H]=3$.
\end{examplebox}

\begin{examplebox}{2. $S_3$ with $H=\langle(12)\rangle$}
Cosets: $H$, $(13)H=\{(13),(132)\}$, $(23)H=\{(23),(123)\}$. Again $[S_3:H]=3$.
\end{examplebox}

\section*{Lagrange's Theorem}
\begin{definitionbox}{Statement}
If $G$ is finite and $H\leq G$, then $|G|=[G:H]\cdot|H|$. Hence $|H|$ divides $|G|$.
\end{definitionbox}

\begin{examplebox}{Why It Works}
$G$ breaks into disjoint cosets $g_1H,\dots,g_kH$. Each has $|H|$ elements, so $|G|=k\cdot|H|$ with $k=[G:H]$.
\end{examplebox}

\subsection*{Key Corollaries}
\begin{itemize}
  \item \textbf{Order of an element divides $|G|$.} If $a\in G$, then $|\langle a\rangle|$ divides $|G|$, so $a^{|G|}=e$ in finite groups.
  \item \textbf{Groups of prime order are cyclic.}
  \item \textbf{Fermat's little theorem:} in $U_p=\{1,2,\dots,p-1\}$ under multiplication mod $p$, $a^{p-1}\equiv1\pmod p$.
\end{itemize}

\subsection*{Worked Applications}
\begin{examplebox}{1. Subgroup Sizes of $S_3$}
Divisors of $6$ are $1,2,3,6$; no subgroup of size $4$ or $5$ exists.
\end{examplebox}

\begin{examplebox}{2. Orders in $\mathbb{Z}_{12}$ under $+$}
Orders divide $12$: $\operatorname{ord}(1)=12$, $\operatorname{ord}(2)=6$, $\operatorname{ord}(3)=4$, $\operatorname{ord}(4)=3$, $\operatorname{ord}(6)=2$, $\operatorname{ord}(0)=1$.
\end{examplebox}

\begin{examplebox}{3. No element of order $6$ in $S_3$}
Allowed by Lagrange, but actual element orders are only $1,2,$ or $3$.
\end{examplebox}

\begin{examplebox}{4. Index in $\mathbb{Z}_{12}$}
For $H=\langle4\rangle=\{0,4,8\}$, $|H|=3$ and $[\mathbb{Z}_{12}:H]=4$. Cosets are $\{0,4,8\}$, $\{1,5,9\}$, $\{2,6,10\}$, $\{3,7,11\}$.
\end{examplebox}

\section*{Common Pitfalls}
\begin{warnbox}{Be Careful!}
\begin{itemize}
  \item Closure is more than being nonempty.
  \item Many groups are not abelian.
  \item ``Divides" does not guarantee existence.
  \item Every subgroup must contain the identity.
\end{itemize}
\end{warnbox}

\section*{Mini Checklists}
\begin{examplebox}{Testing a Subgroup}
\begin{itemize}
  \item Nonempty?
  \item For all $a,b\in H$, is $ab^{-1}$ in $H$?
\end{itemize}
\end{examplebox}

\begin{examplebox}{Computing Cosets}
\begin{itemize}
  \item Pick a representative $g$.
  \item Form $gH=\{gh:h\in H\}$.
  \item Repeat with elements not yet used.
\end{itemize}
\end{examplebox}

\section*{Quick Practice}
\begin{itemize}
  \item Is $\{0,3,6,9\}$ a subgroup of $\mathbb{Z}_{12}$ under $+$?
  \item List all cosets of $H=\{0,2,4\}$ in $\mathbb{Z}_6$.
  \item In $S_3$, why is $\{(12),(13)\}$ not a subgroup?
  \item Find $\operatorname{ord}(2)$ in $\mathbb{Z}_8$ and list $\langle2\rangle$.
  \item Show every finite subgroup of $(\mathbb{R}^*,\times)$ is cyclic.
\end{itemize}

\section*{Summary}
Groups require \textcolor{blue}{closure}, \textcolor{blue}{associativity}, an \textcolor{blue}{identity}, and \textcolor{blue}{inverses}. Subgroups pass the test $ab^{-1}\in H$. Cosets split the group into equal pieces, leading to \textcolor{purple}{Lagrange's Theorem}: $|G|=[G:H]\cdot|H|$. These tools classify small groups, compute orders, and prove number theory facts.

\end{document}
